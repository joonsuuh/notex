\documentclass[../main.tex]{subfiles}
\graphicspath{{../images/}}

\begin{document}
% lecture 1/29/24
\pagebreak
\subsection*{Lecture 6: \hfill  1/29/24}
\hrule \vspace{10px}
\section{Energy}
\hrule \vspace{10px}

Review: There are two requirements for conservation of angular momentum
\begin{enumerate}
    \item Force is central
    \item External torque is zero
\end{enumerate}

\paragraph{Kinetici Energy:} $T = \frac{1}{2} m v^2 = \frac{1}{2} \vb{v} \cdot \vb{v}$. Taking the
time derivative
\begin{align*}
    \dot T &= \frac{1}{2} (\dot{\vb{v}} \cdot \vb{v} + \vb{v} \cdot \dot{\vb{v}}) \\
    &= m \dot{\vb{v}} \cdot \vb{v} = \vb{F} \cdot \vb{v}
\end{align*}
and integrating over time $t_1$ to $t_2$
\begin{align*}
    \int_{t_1}^{t_2} \dot T \dd{t} = \Delta T = \int_{t_1}^{t_2} \vb{F} \cdot \vb{v} \dd{t}
    = \int_{t_1}^{t_2} \vb{F} \cdot \dd{\vb{r}} = W(1 \to 2)
\end{align*}
since $\vb{v} \cdot \dd{t} = \dd{\vb{r}}$ and $\vb{F} \cdot \dd{\vb{r}}$ hints that this is a line
integral.

\paragraph{Example:}
\begin{align*}
    \vb{F}(x,y) &= \vb{r} = x \vu{x} + y \vu{y} \\
    \dd{vb{r}} &= \dd{x} \vu{x} + \dd{y} \vu{y}
\end{align*}
(a) $y = x$ from $a = (0,0)$ to $b = (1,1)$
\begin{align*}
    \int_a^b \vb{F} \cdot \dd{\vb{r}} &= \int_0^1 (x \dd{x} + y \dd{y}) \\
    &= \int_0^1 x \dd{x} + \int_0^1 x \dd{x} = 1
\end{align*}
(b) $y = x^2$ and $\dd{y} = 2x \dd{x}$
\begin{align*}
    \int_a^b \vb{F} \cdot \dd{\vb{r}} &= \int_0^1 (x \dd{x} + x^2 \dd{y}) \\
    &= \int_0^1 x \dd{x} + \int_0^1 2x^2 \dd{x} = 1
\end{align*}
thus the line integral is independent of the path.
\subsubsection*{Conservative force}
\begin{enumerate}
    \item Given $\vb{F}(\vb{r})$, there is no dependence on $\vb{v}$, $t$.
    \item $\int \vb{F} \cdot \dd{\vb{r}}$ is independent of the path.
\end{enumerate}
\begin{align*}
    U(\vb{r}) = -W (\vb{r}_0 \to \vb{r}) = -\int_{\vb{r}_0}^{\vb{r}} \vb{F} \cdot \dd{\vb{r}}'
\end{align*}
For gravity
\begin{align*}
    U_g (\vb{r}) = - \int_a^b \vb{F}_g \cdot \dd{\vb{r}} = - \int_a^b -mg \dd{y}' = mg(y_a - y_b)
\end{align*}
\paragraph{Work-Kinetic Energy Theorem:}
\begin{align*}
    W(1 \to 2) = W(1 \to \mathcal{O}) + W(\mathcal{O} \to 2) = U(1) - U(2) = \Delta T
\end{align*}
From this we have
\begin{align*}
    \Delta T = T_2 - T_1 = U_1 - U_2
\end{align*}
rearranging terms give the mechanical energy
\begin{align*}
    T_2 + U_2 = T_1 + U_1 = E
\end{align*}
and for $N$ conservative forces in a system
\begin{align*}
    E = T + U_1 + U_2 + \dots + U_N
\end{align*}

\pagebreak
\subsection*{Lecture 7: \hfill  1/31/24}
\hrule \vspace{10px}
\section*{Energy: Part 2}
\hrule \vspace{10px}

\paragraph{Conservative Force: Potential Energy} The mechanical energy
\begin{align*}
    E = T + U
\end{align*}
is made up of the sum of the kinetic and potential energy. This is useful for
\begin{itemize}
    \item obtaining equations of motion (EOM)
\end{itemize}
e.g. finding the EOM for a simple pendulum of mass $m$, length $L$ and initial angle $\theta$. The
kinetic energy is
\begin{align*}
    T = \frac{1}{2} m L^2 \dot{\theta}^2
\end{align*}
where the magnitude of velocity is the tangential component $v = L \omega = L \dot \theta$. The 
potential energy is
\begin{align*}
    U = -mgy = -mg L \cos\theta
\end{align*}
and the conservation of energy tells us that the mechanical energy is constant
\begin{align*}
    T + U = \textrm{constant} &= E \\
    \frac{1}{2} m L^2 \dot{\theta}^2 - mg L \cos\theta &= E
\end{align*}
and in the intial condition we know that the velocity is zero $\dot{\theta} = 0$ and thus 
\begin{align*}
    -mg L \cos\theta_{max} &= E
\end{align*}
taking the time derivative of the energy equation gives
\begin{align*}
    m L^2 \dot{\theta} \ddot{\theta} + mg L \sin\theta \dot{\theta} &= 0 \\
    \ddot{\theta} + \frac{g}{L} \sin\theta &= 0 \\
    \ddot \theta &= -\frac{g}{L} \sin\theta
\end{align*}
taking the time derivative of the energy is a useful trick for finding the EOM when we are trying to
solve for $\dot{v}^2$.

\paragraph{Last time} we found the potential energy for a position $\vb{r}$ in a conservative force
field $\vb{F}(\vb{r})$ is
\begin{align*}
    U(\vb{r}) = -W(\vb{r}_0 \to \vb{r}) = -\int_{\vb{r}_0}^{\vb{r}} \vb{F} \cdot \dd{\vb{r}}'
\end{align*}
from the fundamental theorem of calculus (derivatives and intergrals are inverses) we want to find
a function where the derivative equals the conservative force: First we take an infinitesimal change
in the position $\vb{r} \to \vb{r} + \dd{\vb{r}}$ and the change in potential energy is
\begin{align*}
    U(\vb{r} + \dd{\vb{r}}) &= - \int_{\vb{r}_0}^{\vb{r} + \dd{\vb{r}}} \vb{F} \cdot \dd{\vb{r}}' \\
    &= - \int_{\vb{r}_0}^{\vb{r}} \vb{F} \cdot \dd{\vb{r}}' - \int_{\vb{r}}^{\vb{r} + \dd{\vb{r}}}
    \vb{F} \cdot \dd{\vb{r}}' \\
    &= U(\vb r) - F(\vb r) \cdot \dd{\vb{r}}
\end{align*}
where is know that the force is constant over a small distance. Moving the terms gives
\begin{align*}
    U(\vb{r} + \dd{\vb{r}}) - U(\vb{r}) &= - \vb{F} \cdot \dd{\vb{r}} \\
    &= - (F_x \dd{x} + F_y \dd{y} + F_z \dd{z})
\end{align*}
where we use Cartesian Coordinates, and we know that the gradient of potential is
\begin{align*}
    \grad U &= \vu{x} \pdv{U}{x} + \vu{y} \pdv{U}{y} + \vu{z} \pdv{U}{z} \\
    &= - (F_x \vu{x} + F_y \vu{y} + F_z \vu{z}) = - \vb{F}
\end{align*}

\paragraph{Example 3: 1D motion} If we know what $U$ is as a function of $x$, we can find the force!
At points where $E = U$ we call these classical turning points. At a region of a relative minimum, 
a particle below the threshold of the turning point will oscillate between the two turning points.
And at $E>U_max$ the particle is unbound and will escape the forces that attracted it.

\paragraph{Example 4:}
\begin{align*}
    E &= T + U(x) \textrm{ is constant} \\
    T &= \frac{1}{2} m \dot{x}^2  = E - U(x) \\
    \dot {x}^2 &= \frac{2}{m} (E - U(x)) \\
    \dot{x} &= \pm \sqrt{\frac{2}{m} (E - U(x))}
\end{align*} 
using seperation of variables
\begin{align*}
    \dv{x}{t} &= \sqrt{\frac{2}{m} (E - U(x))} \\
    \sqrt{\frac{m}{2}} \dd{t} &= \frac{\dd{x}}{\sqrt{E - U(x)}} \\
    \int_{t_1}^{t_2} \sqrt{\frac{m}{2}} \dd{t} &= \int_{x_1}^{x_2} \frac{\dd{x}}{\sqrt{E - U(x)}} \\
    (t_2 - t_1) &= \sqrt{\frac{2}{m}} \int_{x_1}^{x_2} \frac{\dd{x}}{\sqrt{E - U(x)}}
\end{align*}
where the sign of the velocity is positive within the oscillating bounds of the turning points and
changes sign as the particle moves past the turning point. 

\pagebreak
\subsection*{Lecture 8: \hfill  2/2/24}
\hrule \vspace{10px}
\section*{Energy: Part 3}
\hrule \vspace{10px}

\paragraph{Last time:} Conservative force as a negative gradient of potential:
\[
    \vb{F} = -\grad U
\]
with classical turning points at $E = U$.

\paragraph{Conditions of a conservative force}
\begin{itemize}
    \item Only depends on position $\vb{r}$ (or just constant)
    \item Work done is path independent (this is sometimes hard to check) $\Leftrightarrow
    \curl{\vb F} = 0$
\end{itemize}

\paragraph{What is curl?} In 3D Cartesian coordinates
\begin{align*}
    \curl{\vb F} &= \mqty| \vu{x} & \vu{y} & \vu z \\
                        \pdv{x} & \pdv{y} & \pdv{z} \\
                        F_x & F_y & F_z | \\
    &= \qt(\pdv{F_z}{y} - \pdv{F_y}{z}\vu{x} + \pdv{F_x}{z} - \pdv{F_z}{x}\vu{y} +
    \pdv{F_y}{x} - \pdv{F_x}{y}\vu{z}\qt)
\end{align*}
Mathematically we know that if the force vector is a gradient of a scalar potential
\begin{align*}
    \vb{F} = \grad{\phi} = - \grad{U} \quad \Leftrightarrow \quad \curl{\vb{F}} = 0
\end{align*}
The curl of a gradient is always zero! Short `proof':
\begin{align*}
    F_x = -\pdv{U}{x} \quad F_y = -\pdv{U}{y} \quad F_z = -\pdv{U}{z}
\end{align*}
and the curl
\begin{align*}
    \mqty| \vu{x} & \vu{y} & \vu z \\
    \pdv{x} & \pdv{y} & \pdv{z} \\
    -\pdv{U}{x} & -\pdv{U}{y} & -\pdv{U}{z} | = 0
\end{align*}
Proving that the line integral is path independent is a little difficult, but in general we know
that the two different paths $a$ and $b$ from points $1$ to $2$ we can write the work as
\begin{align*}
    \int_1^2 \vb{F} \cdot \dd{\vb{r}_2} - \int_1^2 \vb{F} \cdot \dd{\vb{r}_1} 
    = \oint_{a-b} \vb{F} \cdot \dd{\vb{r}} = \int_A (\curl \vb{F}) \cdot \dd{\vb{A}} = 0
\end{align*}
where we invoke Stokes' Theorem to find the integral of the curl over the surface $A$ is zero.

\paragraph{Conservative Force:} \( \vb{F} = x \vu{x} + y \vu{y} \) is conservative, but is
\begin{align*}
    \vb{F}(\vb{r}) = F(\vb{r}) \vu{r} 
\end{align*} 
a central force always conservative? We will need to check the curl of the force to find out. But
first we define spherical coordinates $(r, \theta, \phi)$
\begin{align*}
    x &= r \sin\theta \cos\phi \\
    y &= r \sin\theta \sin\phi \\
    z &= r \cos\theta
\end{align*}
and the central force in spherical coordinates is
\begin{align*}
    \vb{F} = F(\vb{r}) \vu{r} + 0 \vu{\theta} + 0 \vu{\phi}
\end{align*}
the curl in spherical coordinates is
\begin{align*}
    \curl \vb{F} &= \frac{1}{r^2 \sin\theta} \qt[\pdv{\theta} (F_\phi \sin\theta) - \pdv{F_\theta}{\phi}] \vu{r}\\
    &+ \frac{1}{r} \qt[\frac{1}{\sin\theta} \pdv{F_r}{\phi} - \pdv{r} (r F_\phi)] \vu{\theta} \\
    &+ \frac{1}{r} \qt[\pdv{r} (r F_\theta) - \pdv{F_r}{\theta}] \vu{\phi}
\end{align*}
And since
\begin{align*}
    \pdv{F}{\theta} = \pdv{F}{\phi} = 0
\end{align*}
the curl is zero $\curl \vb{F} = 0$ and thus $\vb{F}$ is a conservative central force.

\paragraph{Gravity Conservative?} The force due to gravity
\begin{align*}
    \vb{F}_g = -\frac{G M m}{r^2} \vu{r} = -\frac{G M m}{r^3} \vb{r}
\end{align*}
is a central force as it only depends on $\vb{r}$. e.g. for a a two mass system:
\begin{align*}
    \vb{F}_{12} = - \frac{G m_1 m_2}{\abs{\vb{r}_1 - \vb{r}_2}^3} (\vb{r}_1 - \vb{r}_2)
\end{align*}
Using Translational Invariance, we can shift the origin to the center of $m_2$
\begin{align*}
    \vb{r}_2 &= 0 \quad \vb{F}_{12} = - \frac{G m_1 m_2}{\vb{r}_1^3} \vb{r}_1
\end{align*}
or
\begin{align*}
    F_{12} = - \grad_1 U = - (\pdv{U}{x_1}, \pdv{U}{y_1}, \pdv{U}{z_1})
\end{align*}
where the potential energy due to the interaction between 1 and 2 is
\begin{align*}
    U_{12} = - \frac{G m_1 m_2}{\abs{\vb{r}_1 - \vb{r}_2}}
\end{align*}
from newtons 3rd law, the force on 2 due to 1 is
\begin{align*}
    \vb{F}_{12} = - \vb{F}_{21}
\end{align*}
so
\begin{align*}
    -\grad_1 U_{12} &\to \vb{F}_{21} = \grad_1 U_{12} \\
    \grad_1 U_{12} (\vb{r}_1 0 \vb{r}_2) &= - \grad_2 U_{12} (\vb{r}_2 0 \vb{r}_1) \\
    u_{12}(\vb{x}) &\quad \vb{x} = \vb{r}_1 - \vb{r}_2 \\
    \grad_1 U_{12}(\vb{x}) &= \grad_x U_{12}(\vb{x}) = - \grad_2 U_{12}(\vb{x})
\end{align*}
so
\begin{align*}
    \vb{F}_{12} = - \grad_1 U_{12} \qquad \vb{F}_{21} = - \grad_2 U_{12}
\end{align*}
and for $N$ particles
\begin{align*}
    \vb{F}_i = - \grad_i U \qquad U = \sum_{i,j} U_{ij} + \sum_i U_i^{\text{ext}}
\end{align*}

\end{document}