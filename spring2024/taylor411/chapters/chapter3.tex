\documentclass[../main.tex]{subfiles}
\graphicspath{{../images/}}

\begin{document}
% lecture 1/26/24
\pagebreak
\subsection*{Lecture 5: \hfill  1/26/24}
\hrule \vspace{10px}
\section{Center of Mass \& Conservation of Momentum}
\hrule \vspace{10px}

For $N$ particles, the center of mass is
\begin{align*}
    \vb{R} = \frac{1}{M} \sum_i m_i \vb{r}_i
\end{align*}
where $M = \sum_i m_i$ is the total mass of the system. This is similar to the `weighted' average!
Taking the time derivative of $\vb{R}$ gives the total momentum
\begin{align*}
    \dot{\vb{R}} = \frac{1}{M} \sum_i m_i \dot{\vb{r}}_i = \frac{1}{M} \sum_i \vb{p}_i = \vb{P}
\end{align*}
From Newton's 3rd Law
\begin{align*}
    \sum \dot{\vb{p}}_i = \vb{F}_{ext}
\end{align*}
and from the second Law
\begin{align*}
    M \ddot{\vb{R}} = \vb{F}_{ext}
\end{align*}
in integral form 
\begin{align*}
    \vb{R} = \frac{1}{M} \int \vb{r} \dd{m}
\end{align*}
and using the mass density $\dd{m} = \rho \dd{V}$ we can write
\begin{align*}
    \vb{R} = \frac{1}{M} \int \vb{r} \rho \dd{V} = \frac{1}{M} \int \vb{r} \rho \dd{x} \dd{y} \dd{z}
\end{align*}
For a uniform solid semisphere lying on the $xy$ plane with radius $R=1$ and mass $M = 2\pi/3$, the
CM is 
\begin{align*}
    z &= \frac{1}{M} \int \rho z \dd{m} \\
    &= \frac{1}{M} \int z\pi r^2 \dd{z} \\
    &= \frac{1}{M} \int_0^1 \pi z (1 - z^2) \dd{z} \\
    &= \frac{\pi}{M} \qt[\frac{1}{2} - \frac{1}{4}] = \frac{3}{8}
\end{align*}
\subsubsection*{Angular Momentum}
For the singular particle, the angular momentum is
\begin{align*}
    \vb{l} = \vb{r} \cross \vb{p}
\end{align*}
and the total angular momentum of an multi particle system is
\begin{align*}
    \vb{L} = \sum_i \vb*{\ell}_i = \sum_i \vb{r}_i \cross \vb{p}_i
\end{align*}
and the time derivative of $\vb{L}$ is
\begin{align*}
    \dot{\vb{L}} = \sum_i \dot{\vb{r}}_i \cross \vb{p}_i + \vb{r}_i \cross \dot{\vb{p}}_i
    = \sum_i \vb{r}_i \cross \dot{\vb{p}}_i = \sum_i \vb{r}_i \cross \vb{F}_i = \sum_i \vb{\Gamma}_i
\end{align*}
where $\dot{\vb{r}}_i \cross \vb{p}_i = 0$ since $\dot{\vb{r}}_i$ is parallel to $\vb{p}_i$. Since
$\vb{F}_i$ is the force on the $i$th particle,
\begin{align*}
    \vb{F}_i = \sum_{j\neq i} \vb{F}_{ij} + \vb{F}_i^{ext}
\end{align*}
Plugging into the time derivative of angular momentum
\begin{align*}
    \dot{\vb L} = \sum_i \sum_{j\neq i} \vb{r}_i \cross \vb{F}_{ij} + \sum_i \vb{F}_i^{ext}
\end{align*}
In terms of a matrix, the double sum skips the diagonal elements and thus we can pair the indices
that are reflected on the diagonal
\begin{align*}
    \sum_i \sum_{j > i} (\vb r_i \cross \vb F_{ij} + \vb r_j \cross \vb F_{ji}) = \sum_i \sum_{j > i}
    (\vb r_i - \vb r_j) \cross \vb F_{ij}
\end{align*}
where we use the associativity of the cross product and N3L $\vb{F}_{ij} = -\vb{F}_{ji}$. In
addition the force must be central along the line connecting the two particles. Thus we get
\begin{align*}
    \dot{\vb L} = \sum_i \Gamma_i^{ext}
\end{align*}
The direction of the angular momentum is along the axis of rotation. 

\paragraph{A car} To move a car forward, you exert a torque clockwise on the wheels, and from the 
conservation of angular momentum the car will typically want to rotate counter clockwise which feels
like the weight is being pushed back. The torque on the car will increase the friction on the rear
wheel (increasing traction) and thus RWD are better at high accelerations.

\end{document}