\documentclass[../hw.tex]{subfiles}
\begin{document}
\setcounter{section}{6}
\begin{center}
  \section*{Homework 6} \label{sec:homework6}
  \subsection*{Due 2/28}
\end{center}
\addcontentsline{toc}{section}{\nameref{sec:homework6}}
\hrule \vspace{10px}

\paragraph*{1.} (a) Geometrically we find a constraint
\begin{align*}
    \tan \alpha = \frac{r}{z} \qor z = r \cot \alpha; \qquad \dot z  = \dot r \cot \alpha
\end{align*}
where $z$ is the vertical position. The position vector is a linear
combination of this $z$ position and polar position:
\begin{align*}
    \vb r = r \vu r + z \vu z; \quad \vb v = \dot r \vu r + r\dot \phi \vu*\phi + \dot z \vu z
\end{align*}
so the kinetic energy is
\begin{align*}
    T &= \frac{1}{2} m (\dot r^2 + r^2 \dot \phi^2 + \dot z^2)
        = \frac{1}{2} m (\dot r^2 + r^2 \dot \phi^2 + \dot r^2 \cot^2 \alpha) \\
    \qusing &1 + \tan^2 \alpha = \sec^2 \alpha \implies \cot^2 \alpha = \csc^2 \alpha - 1 \\
    T &= \frac{1}{2} m (\dot r^2 + r^2 \dot \phi^2 + \dot r^2 (\csc^2 \alpha - 1))
        = \frac{1}{2} m (r^2 \dot \phi^2 + \dot r^2 \csc^2 \alpha) \\
\end{align*}
and the potential energy is
\begin{align*}
    U &= mgz = mgr \cot \alpha
\end{align*}
so the Lagrangian is
\begin{align*}
    \lagr &= T - U = \frac{1}{2} m (r^2 \dot \phi^2 + \dot r^2 \csc^2 \alpha) - mgr \cot \alpha
\end{align*}
(b) The EL eqn for $\phi$ is
\begin{align*}
    \pdv{\lagr}{\phi} &= \dv{t} \pdv{\lagr}{\dot \phi} \\
    0 &= \dv{t}(mr^2 \dot \phi) \implies mr^2 \dot \phi = \text{constant} = \ell
\end{align*}
which states the conservation of angular momentum. The EL eqn for $r$ is
\begin{align*}
    \pdv{\lagr}{r} &= \dv{t} \pdv{\lagr}{\dot r} \\
    mr\dot\phi^2 - mgr \cot\alpha &= \dv{t}(m\dot r \csc^2\alpha) \\
    &= m\ddot r \csc^2\alpha
\end{align*}
where the mass cancels out so we can simplify to
\begin{align*}
    r\dot\phi^2 - gr \cot\alpha &= \ddot r \csc^2\alpha \\
    0 &= \ddot r - r \dot\phi^2 \sin^2\alpha + g \frac{\cos\alpha}{\sin\alpha} \sin^2\alpha \\
    0 &= \ddot r - r \dot\phi^2 \sin^2\alpha + g \cos\alpha \sin\alpha
\end{align*}
from the conservation of angular momentum
\begin{align*}
    m r^2 \dot \phi = \ell \implies \dot \phi = \frac{\ell}{mr^2}
\end{align*}
so
\begin{align*}
    0 &= \ddot r - \frac{\ell^2}{m^2 r^3} \sin^2\alpha + g \cos\alpha \sin\alpha 
\end{align*}
and solving for when $r = r_o \implies \ddot r = 0$:
\begin{align*}
    0 &= 0 - \frac{\ell^2}{m^2 r_o^3} \sin^2\alpha + g \cos\alpha \sin\alpha \\
    \frac{\ell^2}{m^2 r_o^3} \sin^2\alpha &= g \cos\alpha \sin\alpha \\
    r_o^3 &= \frac{\ell^2}{m^2 g} \frac{\sin\alpha}{\cos\alpha} = \frac{\ell^2}{m^2 g} \tan\alpha \\
    r_o &= \qt(\frac{\ell^2}{m^2 g} \tan\alpha)^{1/3}
\end{align*}
we can analyze the stability of this solution by assuming a small deviation from the eq point:
\begin{align*}
    r = r_o + \eta 
\end{align*}
and looking at how the second derivative behaves: rewriting the EL eqn,
\begin{align*}
    \ddot r &= \frac{\ell^2}{m^2 r^3} \sin^2\alpha - g \cos\alpha \sin\alpha
\end{align*}
and we find a substitution to directly compare the two terms:
\begin{align*}
    \frac{\ell^2}{m^2 r_o^3} \sin^2\alpha &= \frac{\ell^2}{m^2 \frac{\ell^2}{m^2 g} \tan\alpha} \sin^2\alpha
        = \frac{g}{\tan\alpha} \sin^2\alpha = g \cos\alpha \sin\alpha
\end{align*}
so
\begin{align*}
    \ddot r &= \frac{\ell^2}{m^2 r^3} \sin^2\alpha - \frac{\ell^2}{m^2 r_o^3} \sin^2\alpha \\
    &= \frac{\ell^2}{m^2} \sin^2\alpha \qt(\frac{1}{(r_o + \eta)^3} - \frac{1}{r_o^3})
\end{align*}
where
\begin{align*}
    \frac{1}{(r_o + \eta)^3} < \frac{1}{r_o^3}
\end{align*}
so when $r$ slightly increases ($\eta > 0$), the bead tends back toward the eq point ($\ddot r < 0$)
and when $r$ slightly decreases ($\eta < 0$), the bead tends back toward the eq point i.e.
\begin{align*}
    \frac{1}{(r_o - \eta)^3} > \frac{1}{r_o^3}
\end{align*}
thus $r_o$ is a stable equilibrium point.

\newpage 
\paragraph*{2.} (a) Setting the origin at the circle of radius $R$, the position vector of the bead is
\begin{align*}
    \vb r &= (R\cos(\omega t) + r\cos(\theta + \omega t))\vu x + (R\sin(\omega t) + r\sin(\theta + \omega t))\vu y \\
    \vb v &= (-R\omega\sin(\omega t) - r(\dot \theta + \omega)\sin(\theta + \omega t))\vu x
        + (R\omega\cos(\omega t) + r(\dot \theta + \omega)\cos(\theta + \omega t))\vu y \\
    v^2 &= \color{draculagreen}
                R^2 \omega^2 \sin^2(\omega t) 
        \color{draculapurple} 
            + r^2 (\dot \theta + \omega)^2 \sin^2(\theta + \omega t)
        \color{draculafg} 
            - 2Rr\omega(\dot \theta + \omega)\sin(\omega t)\sin(\theta + \omega t) \\
    &\quad \color{draculagreen} 
                + R^2\omega^2\cos^2(\omega t) 
        \color{draculapurple} 
            + r^2(\dot \theta + \omega)^2 \cos^2(\theta + \omega t)
        \color{draculafg}
            + 2Rr\omega(\dot \theta + \omega)\cos(\omega t)\cos(\theta + \omega t) \\
    &= \color{draculagreen}
            R^2 \omega^2
        \color{draculapurple}
            + r^2 (\dot \theta + \omega)^2 
        \color{draculafg}
            + 2Rr\omega(\dot \theta + \omega)
                (\cos(\omega t) \cos(\theta + \omega t) + \sin(\omega t) \sin(\theta + \omega t))
\end{align*}
where we use the sum identity:
\begin{align*}
    \cos(\alpha + \beta) &= \cos\alpha\cos\beta - \sin\alpha\sin\beta
\end{align*}
so 
\begin{align*}
    \cos(\omega t) \cos(\theta + \omega t) + \sin(\omega t) \sin(\theta + \omega t) &= 
        \cos(\omega t - (\theta + \omega t)) = \cos(-\theta) = \cos\theta
\end{align*}
thus the velocity squared is
\begin{align*}
    v^2 &= R^2 \omega^2 + r^2 (\dot \theta + \omega)^2 + 2R\omega r(\dot \theta + \omega)\cos\theta
\end{align*}
which is equivalent to the square of the sum of vectors:
\begin{align*}
    (\vb a + \vb b)^2 &= a^2 + b^2 + 2\vb a \cdot \vb b = a^2 + b^2 + 2ab\cos\theta
\end{align*}
where $\vb a = (R\omega) \vu a$ and $\vb b = (r(\dot \theta + \omega)) \vu b$ is the velocity 
of the hoop and the bead respectively. The kinetic energy is
\begin{align*}
    T &= \frac{1}{2} m v^2 = \frac{1}{2} m
        (R^2 \omega^2 + r^2 (\dot \theta + \omega)^2 + 2R\omega r(\dot \theta + \omega)\cos\theta)
\end{align*}
where we can assume there is no potential energy i.e. $U = 0$. So the Lagrangian is
\begin{align*}
    \lagr &= \frac{1}{2} m (R^2 \omega^2 + r^2 (\dot \theta + \omega)^2 + 2R\omega r(\dot \theta + \omega)\cos\theta)
\end{align*}
(b) from which we can find the EL eqn for $\theta$:
\begin{align*}
    \pdv{\lagr}{\theta} &= -mR\omega r(\dot \theta + \omega) \sin\theta \\
    \dv{t}(\pdv{\lagr}{\dot \theta}) &= \dv{t} [mr^2(\dot \theta + \omega) + mR\omega r\cos\theta] \\
    &= mr^2 \ddot \theta - mR\omega r \dot \theta \sin\theta
\end{align*}
so 
\begin{align*}
    -mR\omega r(\dot \theta + \omega) \sin\theta &= mr^2 \ddot \theta - mR\omega r \dot \theta \sin\theta \\
    0 &= mr^2 \ddot \theta + mR\omega^2 r \sin\theta 
\end{align*}
and dividing by $mr^2$:
\begin{align*}
    \ddot \theta + \frac{R}{r} \omega^2 \sin\theta = 0
\end{align*}
which is the EOM for a simple pendulum when $R = r$. Finding the eq point(s):
\begin{align*}
    \theta = \theta_0 \implies \ddot \theta = 0
\end{align*}
so
\begin{align*}
    0 + \omega^2 \sin\theta_0 = 0 \implies \sin\theta_0 = 0 \implies \theta_0 = n\pi
\end{align*}
which gives us two eq points: $\theta_0 = 0, \pi$($2\pi \equiv 0$ in this context). 
We can analyze the stability of these eq points by assuming a small deviation from the eq point:
\begin{align*}
    \theta = \theta_0 + \eta
\end{align*}
\paragraph*{}For the case $\theta_0 = 0$: We can use the simple approximation
\begin{align*}
    \sin(0 + \eta) \approx \eta
\end{align*}
and we look at the characteristic of the second derivative:
\begin{align*}
    \ddot \theta &= - \frac{R}{r} \omega^2 \sin\theta \qusing \frac{R}{r} \omega^2 = C_1 \\
    \ddot \theta &= -C_1 \eta
\end{align*}
so when the bead moves slightly counter-clockwise(CCW) i.e. $\eta > 0$, it accelerates clockwise(CW) 
i.e. $\ddot \theta < 0$ and vice versa. When the bead deviates from the eq point $\theta_0 = 0$, it
tends back toward equilibrium thus a stable equilibrium point.
\paragraph*{} For $\theta_0 = \pi$: We will be more careful with the approximation by using Taylor
expansion:
\begin{align*}
    \sin(\pi + \eta) \approx \sin\pi + \eta \cos\pi = -\eta
\end{align*}
so the second derivative is
\begin{align*}
    \ddot \theta &= -C_1 (-\eta) = C_1 \eta
\end{align*}
so when the bead moves slightly CCW($\eta > 0$), it accelerates CCW($\ddot \theta > 0$) and vice
versa. Thus $\theta_0 = \pi$ is an unstable equilibrium point.

\newpage 
\paragraph*{3.} (a) The position and velocity of mass $M$ is
\begin{align*}
    \vb r_M &= (x + L \sin\phi) \vu x + (L\cos\phi) \vu y \\
    \vb v_M &= (\dot x + L\dot \phi \cos\phi) \vu x + (-L\dot \phi \sin\theta) \vu y \\ 
\end{align*}
and the velocity squared is
\begin{align*}
    v_M^2 &= \dot x^2 + 2L\dot x \dot \phi \cos\phi + L^2 \dot \phi^2 \cos^2\phi
        + L^2 \dot \phi^2 \sin^2\phi \\
    &= \dot x^2 + 2L\dot x \dot \phi \cos\phi + L^2 \dot \phi^2
\end{align*}
The kinetic energy is
\begin{align*}
    T &= \frac{1}{2} (Mv_M^2 + mv_m^2) \qquad v_m^2 = \dot x^2 \\
    &= \frac{1}{2} [M(\dot x^2 + 2L\dot x \dot \phi \cos\phi + L^2 \dot \phi^2) + m\dot x^2] \\
    T &= \frac{1}{2} (M + m) \dot x^2 + \frac{1}{2}M(2L\dot x \dot\phi \cos\phi + L^2 \dot\phi^2)
\end{align*}
and the potential energy is the gravitational potential energy + the spring potential energy:
\begin{align*}
    U &= Mgy + \frac{1}{2}kx^2 = -MgL\cos\phi + \frac{1}{2}kx^2
\end{align*}
so the Lagrangian is
\begin{align*}
    \lagr &= T - U \\
    &= \frac{1}{2} (M + m) \dot x^2 + \frac{1}{2}M(2L\dot x \dot\phi \cos\phi + L^2 \dot\phi^2)
        + MgL\cos\phi - \frac{1}{2}kx^2
\end{align*}
The EL eqn for $x$:
\begin{align*}
    \pdv{\lagr}{x} &= \dv{t}(\pdv{\lagr}{\dot x})\\
    - kx &= \dv{t}((M + m)\dot x + ML\dot \phi \cos\phi) \\
    -kx &= (M + m) \ddot x + ML \ddot \phi \cos\phi - ML\dot\phi^2 \sin\phi
\end{align*}
The EL eqn for $\phi$:
\begin{align*}
    \pdv{\lagr}{\phi} &= \dv{t}(\pdv{\lagr}{\dot \phi}) \\
    -ML\dot x \dot\phi \sin\phi - MgL\sin\phi &= ML \dv{t}(\dot x \cos\phi + L \dot \phi) \\
    -\dot x \dot\phi \sin\phi - g \sin\phi &= \ddot x \cos\phi - \dot x \dot\phi \sin\phi + L\ddot \phi \\
    -g\sin\phi &= \ddot x \cos\phi + L\ddot \phi
\end{align*}
(b) For small $\phi$ we can approximate
\begin{align*}
    \cos\phi &\approx 1,\qquad \sin\phi \approx \phi 
\end{align*}
which simplfies the two EL eqns to
\begin{align*}
    -kx &= (M + m) \ddot x + ML(\ddot \phi - \dot\phi^2 \phi)\\
    -g\phi &= \ddot x + L\ddot \phi
\end{align*}
and throwing away some terms
\begin{align*}
    -kx &= (M + m)\ddot x \implies \ddot x = -\frac{k}{M + m} x \\
    -g\phi &= L\ddot \phi \implies \ddot \phi = -\frac{g}{L}\phi
\end{align*}

\newpage
\paragraph*{4.} (a) From HW 3:
\begin{align*}
    \vb r &= (r\cos\phi) \vu x + (r\sin\phi) \vu y \\
    \vb v &= (\dot r \cos\phi - r\dot\phi\sin\phi) \vu x + (\dot r \sin\phi + r\dot\phi\cos\phi) \vu y
\end{align*}
so the kinetic energy is
\begin{align*}
    T &= \frac{1}{2} m v^2 = \frac{1}{2} m (\dot r^2 + r^2 \dot\phi^2)
\end{align*}
and the potential energy is
\begin{align*}
    U &= \frac{1}{2} k(r - a)^2
\end{align*}
So the Lagrangian is
\begin{align*}
    \lagr &= T - U = \frac{1}{2} m (\dot r^2 + r^2 \dot\phi^2) - \frac{1}{2} k(r - a)^2
\end{align*}
The EL eqn for $r$:
\begin{align*}
    \pdv{\lagr}{r} &= \dv{t}(\pdv{\lagr}{\dot r}) \\
    mr\dot\phi^2 - k(r - a) &= \dv{t}(m\dot r) \\
    mr\dot\phi^2 - k(r - a) &= m\ddot r \\
    \implies -k(r - a) &= m\ddot r - mr\dot\phi^2
\end{align*}
The EL eqn for $\phi$:
\begin{align*}
    \pdv{\lagr}{\phi} &= \dv{t}(\pdv{\lagr}{\dot \phi}) \\
    0 &= \dv{t}(m r^2\dot\phi) = m(2r \dot r \dot\phi + r^2 \ddot\phi) \\
    \implies 0 &= m(r\ddot \phi + 2 \dot r \dot\phi)
\end{align*}
which is what we found in HW 3.

\paragraph*{}(b) The kinetic eqn is the same as before:
\begin{align*}
    T &= \frac{1}{2} m (\dot r^2 + r^2 \dot\phi^2)
\end{align*}
but the potential energy is adds gravitational potential energy:
\begin{align*}
    U &= \frac{1}{2} k(r - a)^2 - mgr\cos\phi
\end{align*}
so the Lagrangian is
\begin{align*}
    \lagr &= T - U = \frac{1}{2} m (\dot r^2 + r^2 \dot\phi^2) - \frac{1}{2} k(r - a)^2 + mgr\cos\phi
\end{align*}
(c) The EL eqn for $r$:
\begin{align*}
    \pdv{\lagr}{r} &= \dv{t}(\pdv{\lagr}{\dot r}) \\
    mr\dot\phi^2 -k(r - a) + mg\cos\phi &= \dv{t}(m\dot r) \\
    r\dot\phi^2 -\frac{k}{m}(r - a) + g\cos\phi &= \ddot r \\
\end{align*}
so
\begin{align*}
    \ddot r - r \dot\phi^2 + \frac{k}{m}(r - a) - g\cos\phi = 0
\end{align*}
and the EL eqn for $\phi$:
\begin{align*}
    \pdv{\lagr}{\phi} &= \dv{t}(\pdv{\lagr}{\dot \phi}) \\
    -mgr\sin\phi &= \dv{t}(m r^2\dot\phi) \\
    &= m(2r \dot r \dot\phi + r^2 \ddot\phi) \\
    -g \sin\phi &= 2 \dot r \dot\phi + r \ddot\phi
\end{align*}
so \begin{align*}
    \ddot \phi + \frac{2}{r} \dot r \dot\phi + \frac{g}{r}\sin\phi = 0
\end{align*}
assuming $\phi$ and $(r - a) = \epsilon$ are small, we can use the approximation around the 
angular eq point $\phi = 0$:
\begin{align*}
    \sin\phi &\approx \phi, \qquad \cos\phi \approx 1
\end{align*}
and getting rid of some higher order terms:
\begin{align*}
    \ddot r - \cancel{r \dot\phi^2} + \frac{k}{m}(r - a ) - g(1) = 0 \\
    \ddot r = g - \frac{k}{m}(r - a) \\
    \ddot \phi + \cancel{\frac{2}{r} \dot r \dot\phi} + \frac{g}{r}\phi = 0 \\
    \ddot \phi =- \frac{g}{r}\phi
\end{align*}
the eq point for $r$ is when $\ddot r = 0$:
\begin{align*}
    0 = g - \frac{k}{m}(r - a) \implies r_o = a + \frac{mg}{k}
\end{align*}
and expanding for small values around eqn point $r = r_o + \epsilon$ where we define $\epsilon = r - a$:
\begin{align*}
    \ddot \epsilon &= g - \frac{k}{m}(r_o + \epsilon - a) \\
    &= g - \frac{k}{m}(\frac{mg}{k} + \epsilon) \\
    &= g - g - \frac{k}{m}\epsilon \\
\end{align*}
thus we get the two simple harmonics oscillators
\begin{align*}
    \ddot \epsilon &= -\frac{k}{m}\epsilon \qquad \omega_1^2 = \frac{k}{m} \\
    \ddot \phi &= -\frac{g}{r_o}\phi \qquad \omega_2^2 = \frac{g}{r_o}
\end{align*}
\end{document}