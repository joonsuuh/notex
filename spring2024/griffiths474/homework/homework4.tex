\documentclass[../main.tex]{subfiles}

\graphicspath{{../images/}}
% shortcut command for fraction 3/2

\begin{document}

\setcounter{section}{1}
\begin{center}
    \addcontentsline{toc}{section}{Homework 1}
    \section*{Homework 4}
    \subsection*{Due 2/14}
\end{center}
\hrule \vspace{10px}

\paragraph*{1.} (a) Imagining the electron as solid sphere, the moment of inertia is 
$I = \frac{2}{5}m_e r^2$. The speed at a point on its `equator' is given by the tangential
velocity $v = \omega r \implies \omega = \frac{v}{r}$. And if this electron is spinning with angular
momentum $\ell = \hbar / 2$, the speed of this point is
\begin{align*}
    \ell &= I \omega \\
    \frac{\hbar}{2} &= \frac{2}{5}m_e r^2 \frac{v}{r} \\
    v &= \frac{5\hbar}{4m_e r}
\end{align*}
(b) If we have probed down to $\num{e-18}$ and still haven't found any structure, then we would 
think that the radius of this electron solid sphere is $r < \num{e-18}$. Then the speed of the
point on the equator is roughly
\begin{align*}
    v > \frac{\num{5e-34}}{4\times \num{9e-31} \times \num{e-18}} \approx \qty{e14}{m/s}
\end{align*}
which is much faster than the speed of light $c = \qty{3e8}{m/s}$. So an electron is probably not
spinning.

\paragraph*{2.} Given that the neutron, proton, and electron are all spin $S=1/2$ particles, the 
total spin of the beta decay
\begin{align*}
    n \to p + e^-
\end{align*}
on the left side is $S_L = 1/2$ and on the right side we have $S_R = \ohf \otimes \ohf = 0 \oplus 1$.
So the angular momentum is not conserved in this process. For the correct
conservation of angular momentum, we need to include the neutrino in the decay
\begin{align*}
    n \to p + e^- + \bar{\nu}_e
\end{align*}
if we were to suppose the electron antineutrino had spin $S=1/2$, then the total spin states could
be $1/2$ or $3/2$ and the angular momentum would be conserved. We could also suppose it has spin
$S=3/2$ and the possible total spin states could be $1/2, 3/2$ or, $5/2$ which also conserves angular
momentum. This means that any half integer spin would conserve angular momentum in the beta decay.

\paragraph*{3.} If the $J_i$'s are Hermitian, then
\begin{align*}
    J_i^\dagger = J_i, \quad J_j^\dagger= J_i, \quad J_k^\dagger = J_k
\end{align*}
and the commutator is defined as
\begin{align*}
    [J_i, J_j] = J_i J_j - J_j J_i
\end{align*}
taking the Hermitian conjugate of the commutator
\begin{align*}
    [J_i, J_j]^\dagger &= (J_i J_j - J_j J_i)^\dagger \\
    &= (J_i J_j)^\dagger - (J_j J_i)^\dagger \\
    &= J_j^\dagger J_i^\dagger - J_i^\dagger J_j^\dagger \\
    &= J_j J_i - J_i J_j \\
    &= -(J_i J_j - J_j J_i) = -[J_i, J_j]
\end{align*}
where on the 3rd step we know that the Hermitian adjoint (conjugate \emph{transpose}) of a product of
matrices is the product of the Hermitian adjoints in reverse order i.e.
$(AB)^\dagger = B^\dagger A^\dagger$ because transposes do this. taking the Hermitian adjoint of the
right hand side:
\begin{align*}
    (if_{ijk} J_k)^\dagger = -i f_{ijk}^\dagger J_k
\end{align*}
so
\begin{align*}
    -[J_i, J_j] = -i f_{ijk}^\dagger J_k \to [J_i, J_j] = i f_{ijk}^\dagger J_k
\end{align*}
and in order for the commutator relation to be true, the structure constants must be real i.e.
$f_{ijk} = f_{ijk}^\dagger$ (real numbers are Hermitian).

\paragraph*{4.} $f$ as a sum:
\begin{align*}
    f(x, y, z) = f_+(x, y, z) + f_-(x, y, z)
\end{align*}
Parities are:
\begin{align*}
    P(f_+) = + f_+, \quad P(f_-) = -f_- \quad P(f(x,y,z)) = f(-x,-y,-z)
\end{align*}
where the parity of $f$ is just the inversion(reflection + 180 degree rotation). Thus the parity of
the RHS:
\begin{align*}
    P(f(x,y,z)) &= P(f_+(x,y,z)) + P(f_-(x,y,z)) \\
    f(-x,-y,z) &= (+f_+(x,y,z)) + (-f_-(x,y,z))
\end{align*}
solving for $f_+$ and substituting $f_- = f - f_+$:
\begin{align*}
    f_+(x,y,z) &= f(-x,-y,z) + f_-(x,y,z) \\
    &= f(-x,-y,z) + f(x,y,z) - f_+(x,y,z) \\
    2f_+(x,y,z) &= f(-x,-y,z) + f(x,y,z) \\
    f_+(x,y,z) &= \frac{1}{2} \qt(f(x,y,z) + f(-x,-y,z))
\end{align*}
and similarly by substituting the result back into $f_- = f - f_+$:
\begin{align*}
    f_-(x,y,z) &= \frac{1}{2} \qt(f(x,y,z) - f(-x,-y,z))
\end{align*}
we can see that
\begin{align*}
    f_+(x,y,z) + f_-(x,y,z) = f(x,y,z)
\end{align*}
where $f_+$ and $f_-$ is an eigenfunction of the parity operator with eigenvalues $+ 1$ and $-1$
respectively.

\paragraph*{5.} (a) Given  that for EM \& Strong interactions, the parity must be conserved. For the
decay
\begin{align*}
    \eta \to 2\pi
\end{align*}
the parity of $\eta$ is $P(\eta) = -1$ and the parity of $2\pi$ is $P_{tot} = (P(\pi))^2 = (-1)^2 = 1$.
So the parity is not conserved in this decay and thus forbidden for EM \& Strong interactions. 

(b) For the decay
\begin{align*}
    \eta \to 3\pi
\end{align*}
we can see that the parity is conserved: $P(\eta) = P_{tot} = (P(\pi))^3 = -1$. Since a G-parity 
violation forbids decay under Strong interactions, the G-parity of the two sides are:
\begin{align*}
    G(\eta) = (-1)^0 C = 1(+1) = +1, \quad G(3\pi) = (-1)^3 = -1
\end{align*}
so G-parity conservation is violated and the decay is forbidden under Strong interactions, but
allowed for EM interactions due  to Parity conservation.
\end{document}
