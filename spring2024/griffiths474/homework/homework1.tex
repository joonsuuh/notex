\documentclass[../main.tex]{subfiles}

\graphicspath{{../images/}}

\begin{document}

\setcounter{section}{1}
\begin{center}
    \addcontentsline{toc}{section}{Homework 1}
    \section*{Homework 1}
    \subsection*{Due 1/24}
\end{center}
\hrule \vspace{10px}

\paragraph{1.} Gravity vs. E\&M

Given the force of gravitational attraction
\begin{align*}
    \vb{F}_g = -\frac{G m_1 m_2}{r^2} \vu{r}
\end{align*}
and the force of electrostatic repulsion
\begin{align*}
    \vb{F}_e = \frac{1}{4 \pi \epsilon_0} \frac{q_1 q_2}{r^2} \vu{r}
\end{align*}
The ratio of the two forces between two electrons is
\begin{align*}
    \frac{F_g}{F_e} = \frac{4 \pi \epsilon_0 G m_e^2}{q_e^2}
\end{align*}
Using the values for the constants
\begin{align*}
    G &= \qty{6.67e-11}{\N.\m^2/\kg^2}, \quad \epsilon_0 = \qty{8.85e-12}{\C^2/\N.\m^2} \\
    m_e &= \qty{9.11e-31}{\kg}, \quad q_e = \qty{1.60e-19}{\C}
\end{align*}
we find
\begin{align*}
    \boxed{\frac{F_g}{F_e} = \num{2.40e-43}}
\end{align*}
this tells us that the denominator (the electrostatic force) is \emph{much} larger than the
numerator (the gravitational force), which convinces us that gravitaional forces are negligible
for elementary particles.


\paragraph{2.} Mesons and Baryons
\begin{itemize}
\item [(a)] For mesons, you can have $n$ possible quarks and $n$ possible antiquarks, thus there are
$\boxed{n^2}$ combinations.

For baryons order does not matter we have to make sure not to double count instances such as $(uud)$
and $(udu)$: This is essentially a combination problem with the solution
\begin{align*}
    \binom{n}{3} = \frac{n!}{3!(n-3)!} = \boxed{\frac{n(n-1)(n-2)}{6}}
\end{align*}

\item[(b)] Given the 6 flavors of quarks, we would expect $6^2 = 36$ mesons and $\binom{6}{3} = 20$ baryons.

\item[(c)] We haven't found all of them because of energy required to observe the heavier particles. In
the Particle Data Group website, the heaviest baryon is in the order of $\qty{6000}{\MeV}$ and the 
LHC has a beam energy of $\qty{6.5}{\TeV}$ and the energy consumption is about $\qty{1.3}{TWh}$ per
year compared to the global energy consumption of $\qty{20000}{TWh}$ per year 
\href{https://home.cern/resources/faqs/facts-and-figures-about-lhc}{(source)}. In addition to the
enourmous energy required to produce these particles, they are also very unstable and decay very
quickly thus detecting them require us to measure at very small time scales.

\end{itemize}

\paragraph{3.} Global Conservation Laws

\begin{itemize}
    \item [(a)] $n \rightarrow \bar p + e^+ + \nu_e$ 
    
    not valid: violates Baryon number conservation
    \item [(b)] $\nu_e + n \rightarrow p + e^-$
    
    valid
    \item [(c)] $\mu^- \rightarrow e^- + \bar{\nu}_e + \nu_\mu$
    
    valid
    \item [(d)] $\mu^- \rightarrow e^- + \gamma$
    
    not valid because it violates electron and muon lepton number conservation
    \item [(e)] $e^+ + e^- \rightarrow \gamma$
    
    valid
\end{itemize}

\paragraph{4.} Nuclear $\beta$-decay
\begin{itemize}
    \item [(a)]
\begin{align*}
    _Z^A X \rightarrow _{Z+1}^A Y + e^-
\end{align*}
From the conservation of momentum
\begin{align*}
    p^\mu_X &= p^\mu_Y + p^\mu_e \qor p^\mu_Y = p^\mu_X - p^\mu_e
\end{align*}
squaring both sides
\begin{align*}
    p^2_Y &= p^2_X + p^2_e - 2p_X \cdot p_e
\end{align*}
and since
\begin{align*}
    p^2_X = m_X^2 c^2, \quad
    p^2_Y = m_Y^2 c^2, \quad p^2_e = m_e^2 c^2
\end{align*}
and
\begin{align*}
    p_X p_e = \frac{E_X}{c} \frac{E_e}{c} - \vb{p}_X \cdot \vb{p}_e
\end{align*}
but we know that particle $X$ has momentum $\vb{p}_X = 0$ and rest mass $E_X = m_X c^2$ so
\begin{align*}
    m_Y^2 c^2 = m_X^2 c^2 + m_e^2 c^2 - 2 m_X E_e
\end{align*}
solving for the Energy of the outgoing particle is
\begin{align*}
    \boxed{E_e = \frac{m_X^2 + m_e^2 - m_Y^2}{2 m_X} c^2}
\end{align*}
to find the momentum of the outgoing electron we use energy-momentum relation
\begin{align*}
    E_e^2 = \abs{\vb{p}_e}^2 c^2 + m_e^2 c^4
\end{align*}
or
\begin{align*}
    \abs{\vb{p}_e}^2 = \frac{E_e^2}{c^2} - m_e^2 c^2
\end{align*}
using the energy of the outgoing electron we found earlier:
\begin{align*}
    \abs{\vb{p}_e}^2 &= c^2 \qt(\frac{m_X^2 + m_e^2 - m_Y^2}{2 m_X})^2 - m_e^2 c^2 \\
    &= \frac{c^2}{4 m_X^2} (m_X^2 + m_e^2 - m_Y^2)^2 - \qt(\frac{c^2}{4m_X^2}) 4m_X^2 m_e^2 \\
    &= \frac{c^2}{4m_X^2} (m_X^4 + m_e^4 + m_Y^4 + 2 m_X^2 m_e^2 - 2 m_X^2 m_Y^2 - 2 m_e^2 m_Y^2 -
    \mathcolor{draculagreen}{4m_X^2 m_e^2})\\
    &= \frac{c^2}{4m_X^2} (m_X^4 + m_e^4 + m_Y^4 - \mathcolor{draculagreen}{2 m_X^2 m_e^2}
    - 2 m_X^2 m_Y^2 - 2 m_e^2 m_Y^2) \\
\end{align*}
and therefore the momentum of the outgoing electron is
\begin{align*}
\boxed{
    \abs{\vb{p}_e} = \frac{c}{2m_X} \sqrt{m_X^4 + m_e^4 + m_Y^4 - 2 m_X^2 m_e^2 - 2 m_X^2 m_Y^2
    - 2 m_e^2 m_Y^2}
}
\end{align*}

\item[(b)] For the decay including an anti-neutrino
\begin{align*}
    _Z^A X \rightarrow _{Z+1}^A Y + e^- + \bar{\nu}_e
\end{align*}

%%%
For the massless neutrino, the energy is
\begin{align*}
    E_\nu = \abs{\vb{p}_\nu} c
\end{align*}
or from planck's relation
\begin{align*}
    E_\nu = h \nu = \frac{h c}{\lambda}
\end{align*}
so the energy of the neutrino is
\begin{align*}
    \boxed{
    E_\nu = \frac{\qty{6.63e-34}{\J.\s} \cdot \qty{3e8}{\m.\s^{-1}}}{\qty{e-15}{\m}}
    = \qty{1.99e-10}{\J} = \qty{1240}{\MeV}
    }
\end{align*}
and the momentum of the neutrino is
\begin{align*}
    \boxed{\abs{\vb{p}_\nu} = \frac{E_\nu}{c} = \frac{\qty{1.99e-10}{\J}}{\qty{3e8}{\m.\s^{-1}}}
    = \qty{6.63e-19}{\kg.\m.\s^{-1}} \qor \qty{1240}{\frac{\MeV}{c}}
    }
\end{align*}
This is much larger compared to the typical neutrino energy (keV). This means that the neutrino could
not have come from inside the nucleus.

%%%
Using the Heisenberg uncertainty principle
\begin{align*}
    \Delta p \Delta x \geq \frac{h}{4\pi} \qor \Delta p \geq \frac{h}{4\pi \Delta x}
\end{align*}
and the typical size of a nucleus is $\Delta x \approx \qty{1e-15}{\m}$ so
\begin{align*}
    \Delta p \geq \frac{\qty{6.63e-34}{\J.\s}}{4\pi \cdot \qty{e-15}{\m}}
    = \qty{0.53}{\frac{J}{m/s}}
\end{align*}
or in more convenient units $\qty{1}{\eV} = \qty{1.6e-19}{\J}$ and $\unit{c} = \qty{3e8}{m/s}$:
\begin{align*}
    &\Delta p \geq \qty{0.53}{\frac{J}{m/s}} \frac{\qty{1}{\eV}}{\qty{1.6e-19}{\J}}
    \frac{\qty{3e8}{m/s}}{\unit{c}} \\
    &\boxed{
    \Delta p \geq \qty{99}{\frac{\MeV}{c}}
    }
\end{align*}
and the energy of the neutrino (a massless particle) is
\begin{align*}
    \boxed{
    E_\nu = \abs{\vb{p}_\nu} c \geq \qty{99}{\MeV}
    }
\end{align*}
Compared to the typical neutrino energy of $\qty{1}{\keV}$, this is much larger and thus the
neutrino could not have come from inside the nucleus.

\end{itemize} 
\end{document}