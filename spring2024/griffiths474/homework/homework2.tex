\documentclass[../main.tex]{subfiles}

\graphicspath{{../images/}}

\begin{document}

\setcounter{section}{1}
\begin{center}
    \addcontentsline{toc}{section}{Homework 1}
    \section*{Homework 2}
    \subsection*{Due 1/31}
\end{center}
\hrule \vspace{10px}

\paragraph{1.} Muon Decay

The minimum energy of the electron would be equivalent to the rest mass
\begin{align*}
    E_{\text{min}} = m_e c^2 = \qty{9.11e-31}{\kg} \cdot (\qty{3e8}{\m/\s})^2
    = \qty{8.2e-14}{\joule} * \frac{1 \si{\eV}}{\qty{1.6e-19}{\joule}} = 
    \boxed{\qty{0.512}{\mega\eV}}
\end{align*}
The maximum energy of the electron would be when the electron moves in one direction and both 
neutrinos move together in the opposing direction. Thus we can treat this like a two body decay
with the energy of the electron:
\begin{align*}
    E_e &= \frac{m_\mu^2 + m_e^2 - M^2}{2 m_u} c^2 
\end{align*}
where $M$ is the sum of the masses of the neutrinos $M = 0$ (neutrinos are massless lol). Using
the mass of the muon and electron:
\begin{align*}
    m_e &= \qty{9.11e-31}{\kg} \cdot \frac{(\qty{3e8}{\m/\s})^2}{\qty{1.6e-19}{\joule/eV}} 
    = \qty{0.51}{\MeV} \\
    m_\mu &= \qty{1.88e-28}{\kg} = \qty{106}{\MeV}
\end{align*}
The maximum energy of the electron is then
\begin{align*}
    E_{\text{max}} &= \frac{m_\mu^2 + m_e^2}{2 m_\mu} c^2 = \boxed{\qty{53}{\MeV}}
\end{align*}

\paragraph{2.} (a) For the head-on collision, we know that the initial energy is the sum of the rest
masses of the protons and the minimum energy required to produce the antiproton. Thus
\begin{align*}
    E_i &= E_f \\
    2m_p c^2 + E_{\text{min}} &= 4m_p c^2 \\
    E_{\text{min}} &= 2m_p c^2 = \boxed{\qty{1.88}{\GeV}}
\end{align*}
(b) For the fixed target, we first look at the total momentum four vector before the collision:
The zeroth component of the total momentum is
\begin{align*}
    p^0 = \frac{E}{c} + \frac{E_{rest}}{c} = \frac{E}{c} + m_p c
\end{align*}
thus the four-vector before the collision is
\begin{align*}
    p^\mu = \qt(\frac{E}{c} + m_p c, \abs{\vb{p}})
\end{align*}
to find the four-vector after the collision, we can use the center of momentum frame (CM) where the
two protons are viewed as going towards each other at the same speed like part (a), so the 
total three-vector momentum is zero. Thus the zeroth component of the four-vector is jus the sum of
the rest masses of the protons and the antiproton
\begin{align*}
    p^{\mu'} = \qt(4mc, 0)
\end{align*}
we can then exploit the invariant dot product of the four-vectors to find the minimum energy
\begin{align*}
    p^\mu p_\mu &= p^{\mu'} p_\mu' \\
    \qt(\frac{E}{c} + m_p c)^2 - \abs{\vb{p}}^2 &= (4mc)^2
\end{align*}
where the can use the energy momentum relation
\begin{align*}
    E^2 &= \abs{\vb{p}}^2 c^2 + m^2 c^4 \\
    \abs{\vb{p}}^2 &= \frac{1}{c^2} (E^2 - m^2 c^4)
\end{align*}
to solve for the minimum energy
\begin{align*}
    \qt(\frac{E}{c} + m_p c)^2 - \frac{1}{c^2} (E^2 - m_p^2 c^4) &= (4mc)^2 \\
    \frac{E^2}{c^2} + 2m_p E + m_p^2 c^2 - \frac{E^2}{c^2} + m_p^2 c^2 &= 16m^2 c^2 \\
    2m_p E &= 14m_p c^2 \\
    E &= 7m_p c^2 = \boxed{\qty{6.6}{\GeV}}
\end{align*}
Thus the head-on collision requires less energy to produce the antiproton.

\paragraph{3.} (a) Given
\begin{align*}
    s \equiv \frac{(p_A + p_B)^2}{c^2}, \quad t \equiv \frac{(p_A - p_C)^2}{c^2},
    \quad u \equiv \frac{(p_A - p_D)^2}{c^2}
\end{align*}
the sums of the Mandelstroms variables are
\begin{align*}
    s + t + u &= \frac{1}{c^2} \qt[(p_A + p_B)^2 + (p_A - p_C)^2 + (p_A - p_D)^2]
\end{align*}
where we expand the squares
\begin{align*}
    [\;] &= p_A^2 + p_B^2 + 2p_A \cdot p_B + p_A^2 + p_C^2 - 2p_A \cdot p_C 
        + p_A^2 + p_D^2 - 2p_A \cdot p_D \\
    &= 3p_A^2 + p_B^2 + p_C^2 + p_D^2 - 2p_A \cdot p_B - 2p_A \cdot p_C - 2p_A \cdot p_D
\end{align*}
spliting the $3p_A^2$ into $p_A^2 + 2p_A^2$ we can factor out the dot products
\begin{align*}
    [\;] &= p_A^2 + p_B^2 + p_C^2 + p_D^2 - 2 p_A \cdot (p_A + p_B - p_C - p_D)
\end{align*}
and from the conservation of momentum 
\begin{align*}
    p_A + p_B &= p_C + p_D
\end{align*}
so we are left with 
\begin{align*}
    [\;] &= p_A^2 + p_B^2 + p_C^2 + p_D^2 \\
    &= c^2 (m_A^2 + m_B^2 + m_C^2 + m_D^2)
\end{align*}
thus plugging $[\;]$ back into the sum of the Mandelstroms variables gives
\begin{align*}
    s + t + u = m_A^2 + m_B^2 + m_C^2 + m_D^2
\end{align*}

(b) In the CM frame, the total momentum is zero, so the four-vector of the total momentum is
and the total energy is
\begin{align*}
    E_T = E_A + E_B
\end{align*}
we know that the momentum of the two particles are zero:
\begin{align*}
    \vb p_A + \vb p_B = 0
\end{align*}
so from the first Mandelstroms variable
\begin{align*}
    s = \frac{(p_A + p_B)^2}{c^2} &= \frac{1}{c^2} \qt[\qt(\frac{E_A}{c} + \frac{E_b}{c})^2 
        + (\vb p_A + \vb p_B)^2] \\
    s &= \frac{E_T^2}{c^4} 
\end{align*}
so
\begin{align*}
    E_T = c^2 \sqrt{s}
\end{align*}
(c) Since $\vb p_A + \vb p_B = 0$ and $E_A = E_B = E = \sqrt{\vb p^2 c^2 + m^2 c^4}$, the
Mandelstroms variable $s$ is
\begin{align*}
    \frac{1}{c^2} (p + p)^2 &= \frac{1}{c^2} \qt[\frac{(E_A + E_B)^2}{c^2} - (\vb p_A + \vb p_B)^2] \\
    &= \frac{1}{c^2} \qt[\frac{4E^2}{c^2}] \\
    &= \frac{1}{c^2} \qt[\frac{\vb p^2 c^2 + m^2 c^4}{c^2}] \\
    s &= \frac{\vb p^2 c^2 + m^2 c^4}{c^2}
\end{align*}
for $t$ we assume the angle between $\vb p_A$ and $\vb p_C$ is $\theta$ so
\begin{align*}
    t &= \frac{1}{c^2} (p_A - p_C)^2 \\
    &= \frac{1}{c^2} \qt[\qt(\frac{E_A}{c} - \frac{E_C}{c})^2 - (\vb p_A - \vb p_C)^2]
\end{align*}
and we know that $E_A = E_C$ so the first term is zero and the second term is
\begin{align*}
    (\vb p_A - \vb p_C)^2 &= \vb p_A^2 + \vb p_C^2 - 2 \vb p_A \cdot \vb p_C 
\end{align*}
where $\vb p_A^2 = \vb p_C^2 = \vb p^2$ and $\vb p_A \cdot \vb p_C = \vb p^2 \cos \theta$ so
\begin{align*}
    (\vb p_A - \vb p_C)^2 &= 2 \vb p^2 - \vb p ^2 \cos \theta = 2 \vb p^2 (1 - \cos \theta)
\end{align*}
thus
\begin{align*}
    t &= \frac{1}{c^2} [0 - 2 \vb p^2 (1 - \cos \theta)] = \frac{-2 \vb p^2(1 - \cos \theta)}{c^2} 
\end{align*}
for $u$ everything is the same but $\vb p_A \cdot \vb p_D = - \vb p^2 \cos \theta$ so
\begin{align*}
    (\vb p_A - \vb p_D)^2 &= 2 \vb p^2 + \vb p^2 \cos \theta = 2 \vb p^2 (1 + \cos \theta)
\end{align*}
and thus
\begin{align*}
    u &= \frac{-2 \vb p^2 (1 + \cos \theta)}{c^2}
\end{align*}













\paragraph{4.} First we know that the energy and momentum of a photon $\gamma$ are
\begin{align*}
    E_\gamma = h\nu = pc \quad p = \frac{h\nu}{c}
\end{align*}
from the planck relation and energy-mass relation. Using conservation of energy we know that the 
before the collision it is the energy of the photon plus the rest mass of the electron:
\begin{align*}
    E_i &= E_f \\
    E_\gamma + m_e c^2 &= E_\gamma' + E_e
\end{align*}
and using the energy momentum relation for the electron
\begin{align} \label{eq:1} \tag{1}
    h\nu + m_e c^2 &= h\nu' + \sqrt{p_e^2c^2 + (m_e c^2)^2}
\end{align}
and from the conservation of momentum we know that 
\begin{align*}
    \vb p = \vb p' + \vb p_e \qor \vb p_e = \vb p - \vb p'
\end{align*}
where the momentum of the electron is initially zero, squaring both sides
\begin{align*}
    \vb p_e^2 &= (\vb p - \vb p')^2 \\
    p_e^2 &= p^2 + p'^2 - 2 \vb p \cdot \vb p'
\end{align*}
where we know that the dot product of the two momenta is
\begin{align*}
    \vb p \cdot \vb p' = p p' \cos \theta
\end{align*}
so
\begin{align} \label{eq:2} \tag{2}
    p_e^2 &= p^2 + p^2 - 2 p p' \cos \theta
\end{align}
now we relate the two equations by first solving \eqref{eq:1} for $p_e^2 c^2$
\begin{align*}
    p_e^2 c^2 &= (h\nu^2 - h\nu'^2 + m_e c^2)^2 - (m_e c^2)^2
\end{align*}
and then multiplying \eqref{eq:2} by $c^2$
\begin{align*}
    p_e^2 c^2 &= (pc)^2 + (p' c)^2 - 2 p p' c^2 \cos \theta
\end{align*}
and substituting the momentum of the photon from the energy relation $p = \frac{h\nu}{c}$
\begin{align*}
    p_e^2 c^2 = (h\nu)^2 + (h\nu')^2 - 2 h^2 \nu \nu' \cos \theta 
\end{align*}
thus we can set the two equations equal to each other
\begin{align*}
    (h\nu^2 - h\nu'^2 + m_e c^2)^2 - (m_e c^2)^2 &= 
        (h\nu)^2 + (h\nu')^2 - 2 h^2 \nu \nu' \cos \theta \\
    (h\nu)^2 + (h\nu')^2 + (m_e c^2)^2 + 2(h^2 \nu \nu' - h\nu' m_e c^2 + h\nu m_e c^2) - (m_ec^2)^2
        &= (h\nu)^2 + (h\nu')^2 - 2 h^2 \nu \nu' \cos \theta
\end{align*}
where the first three terms cancel out
\begin{align*}
    2(h^2 \nu \nu' - h\nu' m_e c^2 + h\nu m_e c^2) &= - 2 h^2 \nu \nu' \cos \theta
\end{align*}
dividing both sides by $2h$ and rearranging terms gives
\begin{align*}
    m_e c^2 (\nu - \nu') &= h \nu \nu' (1 - \cos \theta)
\end{align*}
dividing both sides again but by $m_e c \nu \nu'$ gives
\begin{align*}
    \qt(\frac{c}{\nu'} - \frac{c}{\nu}) &= \frac{h}{m_e c} (1 - \cos \theta)
\end{align*}
and since the wavelength is $\lambda = \frac{c}{\nu}$ we can solve for the outgoing wavelength
\begin{align*}
    \lambda' - \lambda = \frac{h}{m_e c} (1 - \cos \theta) \\
    \boxed{
        \lambda' = \lambda + \frac{h}{m_e c} (1 - \cos \theta)
    }
\end{align*}

\end{document}