\documentclass[../main.tex]{subfiles}

\graphicspath{{../images/}}
% shortcut command for fraction 3/2

\begin{document}
\lhead{Homework 5}
\rhead{Due 2/21}
\setcounter{section}{5}

\paragraph*{1.} (a) From the Gell-Mann-Nishijima formula
\begin{align*}
    Q = I_3 + \frac{1}{2}(A + S)
\end{align*}
For the $uds$ quarks, the isospin is $I_3 = \ohf, -\ohf, 0$, the baryon number is $A = \frac{1}{3}$
and the strangeness is $S = 0, 0, -1$ respectively. The charges are then
\begin{align*}
    Q_u &= \ohf + \frac{1}{2}\qt(\frac{1}{3} + 0)= \frac{2}{3} \\
    Q_d &= -\ohf + \frac{1}{2}\qt(\frac{1}{3} + 0)= -\frac{1}{3} \\
    Q_s &= 0 + \frac{1}{2}\qt(\frac{1}{3} - 1)= -\frac{1}{3}
\end{align*}
(b) The antiparticle will have the opposite charge $Q_{\bar{q}} = -Q_q$, baryon number
$A_{\bar q} = -\frac{1}{3}$ and strangeness $S_{\bar q} = 0, 0, 1$, so the isospin states are
\begin{align*}
    Q_{\bar{u}} &= -\frac{2}{3} = I_3 + \frac{1}{2}\qt(-\frac{1}{3} + 0)
        \implies I_3 = -\ohf \\
    Q_{\bar{d}} &= \frac{1}{3} = I_3 + \frac{1}{2}\qt(-\frac{1}{3} + 0)
        \implies I_3 = \ohf \\
    Q_{\bar{s}} &= \frac{1}{3} = I_3 + \frac{1}{2}\qt(-\frac{1}{3} + 1)
        \implies I_3 = 0
\end{align*}
so the isospin assignments $\ket{I, I_3}$ are
\begin{align*}
    \bar u = \ket{\ohf, -\ohf}, \quad \bar d = \ket{\ohf, \ohf}, \quad \bar s = \ket{0, 0}
\end{align*}

\paragraph*{2.} (a) For a the charged kaon
\begin{align*}
    K^- \Leftrightarrow K^+
\end{align*}
the charge is not conserved, so they cannot interconvert, so only the neutral mesons can mix.
(b) We don't observe baryon-antibaryon interconversion because it violates baryon number conservation.
(c) There is no mixing of neutral strange vector mesons because the $K^{*0}$ and $\bar{K}^{*0}$ have
different strangeness $S= +1, -1$, so they cannot mix due to strangeness conservation.

\paragraph*{3.} From the Schr\"odinger equation
\begin{align*}
    i\hbar\pdv{t}\ket{\psi(t)} &= H \ket{\psi(t)}
\end{align*}
and the time reversal operator $\ket{\psi(t)} = T \ket{\psi(-t)}$, so the equation is
\begin{align*}
    i\hbar\pdv{(t)}T\ket{\psi(-t)} &= H T\ket{\psi(-t)}
\end{align*}
and the time derivative of the time-reversed state using chain rule
\begin{align*}
    i\hbar\pdv{(t)}\ket{\psi(-t)} &= -i\hbar\pdv{t}\ket{\psi(-t)}
\end{align*}
for the right side, since $T$ and $H$ commute
\begin{align*}
    -i\hbar\pdv{t}\ket{\psi(-t)} &= T H\ket{\psi(-t)} \\
    -i\hbar\pdv{t}T\ket{\psi(t)} &= T HT\ket{\psi(t)} = T T H \ket{\psi(t)}
\end{align*}
and since $T^2 = 1$
\begin{align*}
    -i\hbar\pdv{t}T\ket{\psi(t)} &= H \ket{\psi(t)}
\end{align*}
or
\begin{align*}
    T c = c^* T
\end{align*}

\paragraph*{4.} Given the Hamiltonian
\begin{align*}
    H = - \frac{1}{\abs{\vb J}} (\mu \vb J \cdot \vb B + d \vb J \cdot \vb E)
\end{align*}
(a) From Maxwells equations
\begin{align*}
    \div \vb E &= 4\pi \rho \\
    \curl \vb E &= -\pdv{\vb B}{t}
\end{align*}
under time reversal $t \to -t$ the electric field is T-even $E \to E$ and the magnetic field is T-odd
$B \to -B$. From angular momentum
\begin{align*}
    \vb L = \vb r \times \vb p = \vb r \times m \dv{\vb r}{t}
\end{align*}
so under time reversal $t \to -t$ angular momentum is T-odd $\vb L \to -\vb L$, and since spin 
angular momentum is T-odd by the right hand rule, the total angular momentum is T-odd $\vb J \to -\vb J$.

\paragraph*{}For the parity, the magnetic field and angular momentum are even under parity since 
they are pseudovectors, and the electric field is odd under parity as a vector.

\paragraph*{} The charge conjugation of the electric field is $E \to -E$ and the magnetic field is
$B \to -B$ since the antiparticle will have the opposite charge. The total angular momentum is
invariant under charge conjugation $\vb J \to \vb J$ since the antiparticle will have the same spin.
\begin{align*}
    C&: \vb E \to -\vb E, \quad \vb B \to -\vb B, \quad \vb J \to \vb J \\
    P&: \vb E \to -\vb E, \quad \vb B \to \vb B, \quad \vb J \to \vb J \\
    T&: \vb E \to \vb E, \quad \vb B \to -\vb B, \quad \vb J \to -\vb J
\end{align*}
(b) Since the Hamiltonian is invariant under time reversal so $\mu$ is T-even (odd times odd is even
and even times even is even) and $d$ is T-odd. The Hamiltonian is also invariant under parity so
$\mu$ is P-even and $d$ is P-odd. For charge conjugation, the Hamiltonian is invariant so $\mu$ is
C-odd and $d$ is C-odd.
\begin{align*}
    C&: \mu \to \mu, \quad d \to -d \\
    P&: \mu \to \mu, \quad d \to -d \\
    T&: \mu \to -\mu, \quad d \to -d
\end{align*}
\end{document}