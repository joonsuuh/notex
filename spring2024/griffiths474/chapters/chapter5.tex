\documentclass[../main.tex]{subfiles}
\usepackage[compat=1.1.0]{tikz-feynman}
\graphicspath{{../images/}}

\begin{document}
\hrule
\section{Decay and Scattering}
\hrule \vspace{10px}

\lhead{Lecture 15: 3/18/24}
\chead{Decay and Scattering}
\rhead{PHYS 474}

\paragraph*{Decay rate} $\Gamma$
\begin{itemize}
    \item Probability per unit time for the decay to happen
\end{itemize}
For a decay process the change in the number of particles (amount of stuff that decayed)
\begin{align*}
    -N(t) \Gamma \dd t &= \dd N
\end{align*}
we can solve this differential equation to find
\begin{align*}
    \int \frac{\dd N}{N} &= -\int \Gamma \dd t \\
    \ln N &= -\Gamma t + C \\
    \implies N(t) &= N_0 e^{-\Gamma t}
\end{align*}
we can find the mean lifetime $\tau = \frac{1}{\Gamma}$ so
\begin{align*}
    N(t) &= N_0 e^{-t/\tau}
\end{align*}
\paragraph*{Half time}
and the half-life is when
\begin{align*}
    N(t_{1/2}) &= \frac{N_0}{2} = N(0) e^{-\Gamma t_{1/2}} \\
    \implies e^{\Gamma t_{1/2}} &= 2 \\
    \Gamma t_{1/2} &= \ln 2
\end{align*}
or 
\begin{align*}
    t_{1/2} &= \tau \ln 2
\end{align*}
Example
\begin{align*}
    \pi^+ \to &\mu^+ + \nu_\mu \qquad \Gamma_1 \gg \Gamma_2 \\
    &e^+ + \nu_e \qquad \Gamma_2
\end{align*}
and
\begin{align*}
    \Gamma_{tot} = \sum_i \Gamma_i \qquad \tau_{tot} = \frac{1}{\Gamma_{tot}}
\end{align*}
we have a branching ratio (or fraction)
\begin{align*}
    \text{Br}_i = \frac{\Gamma_i}{\Gamma_{tot}} \qquad [0,1]
\end{align*}
and we find the branching ratio of the pion decay is experimentally
\begin{align*}
    \text{Br}_1 &= 0.999877 \\
    \text{Br}_2 &= 0.000123
\end{align*}
Insert Griffiths Figure 6.1 here

\paragraph*{Scattering} From the impact parameter $b$ and scattering angle $\theta$ we can find the
cross section, or the probability of scattering. We have an infinitesimal area of
\begin{align*}
    \dd \sigma &= \abs{\dd b \cdot b\dd \phi}
\end{align*}
which is like the area of a rectangle made by the differential impact parameter. The solid angle is
\begin{align*}
    \dd \Omega &= \sin\theta \dd \theta \dd \phi
\end{align*}
like the theta and phi part of spherical coordinates. The differential cross section is
\begin{align*}
    \dv{\sigma}{\Omega} &= \abs{\frac{b}{\sin\theta} \cdot \dv{b}{\theta}}
\end{align*}
\paragraph*{Hard Sphere Scattering} We have a hard sphere of radius $R$ and we send a particle 
toward the sphere and it scatters on the surface. Thus the cross section is expected to be
\begin{align*}
    \sigma &= \pi R^2
\end{align*}
or the area of a circle that cuts the sphere. From the law of inflection we have an inflection 
\begin{align*}
    2\alpha + \theta = \pi
\end{align*}
and the trigonometry shows that the impact parameter is
\begin{align*}
    b &= R \sin\alpha
\end{align*}
or
\begin{align*}
    b &= R \sin(\frac{\pi}{2} - \frac{\theta}{2}) \\
    &= R \cos(\frac{\theta}{2})
\end{align*}
so the differential cross section is
\begin{align*}
    \dv{b}{\theta} &= - \frac{R}{2} \sin{\frac{\theta}{2}} \\
    \dv{\sigma}{\Omega} &= \abs{\frac{R\cos\frac{\theta}{2}}{\sin\theta} \cdot \frac{R}{2} \sin\frac{\theta}{2}} 
        \qquad \sin\theta = 2\sin\frac{\theta}{2} \cos\frac{\theta}{2}\\
    &= \frac{R^2}{4}
\end{align*}
and
\begin{align*}
    \int \dd{\sigma} &= \int \frac{R^2}{4} \dd{\Omega} \\
    \sigma &= \frac{R^2}{4} \cdot 4\pi = \pi R^2
\end{align*}
\paragraph*{Rutherford Scattering} In the experiment we can find the impact parameter
\begin{align*}
    b &= \frac{q_1 q_2}{2E} \cot\frac{\theta}{2}
\end{align*}
so
\begin{align*}
    \dv{b}{\theta} &= -\frac{q_1 q_2}{4E} \csc^2\frac{\theta}{2}
\end{align*}
and
\begin{align*}
    \dv{\sigma}{\Omega} &= \abs{\frac{b}{\sin\theta} \cdot \dv{b}{\theta}} \\
    &= \abs{\frac{q_1 q_2}{2E} \cot\frac{\theta}{2} \frac{1}{2\sin\frac{\theta}{2} \cos\frac{\theta}{2}} 
    \cdot -\frac{q_1 q_2}{4E} \csc^2\frac{\theta}{2}} \\
    &= \frac{q_1^2 q_2^2}{16E^2} \csc^4\frac{\theta}{2}
\end{align*}
so the cross section is
\begin{align*}
    \sigma &= \int \dv{\sigma}{\Omega} \dd{\Omega} \\
    &= \frac{q_1^2 q_2^2}{16E^2} \int \csc^4\frac{\theta}{2} \sin\theta \dd \theta \dd \phi \\
    &= 2\pi \frac{q_1^2 q_2^2}{16E^2} \int \frac{\sin\theta}{\sin^4\frac{\theta}{2}} \dd \theta \\
    &= 2\pi \frac{q_1^2 q_2^2}{16E^2} \int \frac{2\sin\frac{\theta}{2}\cos\frac{\theta}{2}}{\sin^4\frac{\theta}{2}} \dd \theta
\end{align*}
and substituting
\begin{align*}
    x &= \sin\frac{\theta}{2} \implies \dd x = \frac{1}{2} \cos\frac{\theta}{2} \dd \theta \\
\end{align*}
so
\begin{align*}
    2\pi \int_0^1 \frac{2x}{x^4} \dd x &= 2\pi \qt(\frac{1}{2x^2})\eval_0^1 \to \infty
\end{align*}
\paragraph*{Fermi Golden Rule} For nonrelativistic system
\begin{align*}
    \text{Transition probability} = \text{phase space} \times \abs{\text{amplitude}}^2
\end{align*}
or
\begin{align*}
    \rho \cdot \abs{\bra f 0 \ket i}^2
\end{align*}
where $\rho $ is the density of states.
\subparagraph*{Relativistic System} 
\begin{align*}
    \dd \Gamma \propto \abs{\mathcal M}^2 \dd \Pi \\
    \dd \sigma \propto \abs{\mathcal M}^2 \dd \Pi
\end{align*} 
where $\dd \Pi$ is the phase space. For the two body decay
\begin{align*}
    1 \to 2 + 3 \\
    m_1 > m_2 + m_3
\end{align*}
\paragraph*{Wigner-Eckart Theorem} For spherically symmetric systems we can split the amplitude into
two parts: the symmetric and dynamic parts.
\begin{align*}
    \bra f 0 \ket \propto \text{symmetric} \times \text{dynamic}
\end{align*}

\newpage
\lhead{Lecture 16: 3/20/24}
Quiz Review
\begin{itemize}
    \item The decay formula gives us 
    \begin{align*}
        N(t) = N_0 e^{-t/\tau} = 10^6 e^{-10} \approx 45
    \end{align*}
    \item The probability of 1 particle still being there after 10 average lifetimes is directly 
    equal to
    \begin{align*}
        e^{-t/\tau} = e^{-10} \approx 4.5 \times 10^{-5}
    \end{align*}
    \item Dirac Delta Function
    \begin{align*}
        \int_{-\infty}^\infty \delta(x) \dd x = 1
    \end{align*}
    or
    \begin{align*}
        \delta (x) = \begin{cases}
            \infty & x = 0 \\
            0 & x \neq 0
        \end{cases}
    \end{align*}
    We can also think of a rectangle with area 1 at $x = 0$ and we keep shortening the width and
    increasing the height to keep the area 1. As the width gets infinitesimally small, the height
    gets infinitely large. 
    \item From the heaviside step function
    \begin{align*}
        \int_{-\infty}^\infty \dv{x} \theta(x) \dd x &= \theta(x) \eval_{-\infty}^\infty = 1 \\
        &= \int_{-\infty}^\infty \delta(x) \dd x \\
        \implies \delta(x) &= \dv{x} \theta(x)
    \end{align*}
\end{itemize}
\paragraph*{Fermi Golden Rule (again)} We know that the phase space is dependent of the kinematics
i.e. it only depends on the number of paritcles involved. The amplitude $\mathcal{M}$ is dependent
on the dynamics or the type of interaction. 

\paragraph*{Decay} $1 \to 2 + 3 + \dots + n$ 
\begin{align*}
    \Gamma = \frac{S}{2m_1 \hbar} \int \abs{\mathcal{M}}^2 &(2\pi)^4 \delta^4(p_1 - p_2 - p_3 - \dots - p_n) \\
    &\times \prod_{j = 2}^n 2\pi \delta(p_j^2 - m_j^2 c^2) \theta(p_j^0) 
    \cdot \frac{\dd^4 p_j}{(2\pi)^4}
\end{align*}
Is the decay rate where $S$ is the symmetry factor
\begin{align*}
    S = \frac{1}{\prod_i k_i!}
\end{align*}
e.g. $a \to b + b + c + c +c$
\begin{align*}
    S = \frac{1}{2!3!} = \frac{1}{12}
\end{align*}
and we also have the phase space part which is in a 4-dimensional component i.e.
\begin{align*}
    \delta^3(\vb r) = \delta(x) \delta(y) \delta(z) \\
    \delta^4(p) = \delta(p^0) \delta^3(\vb p)
\end{align*}
\paragraph*{Phase space parts} 
\begin{enumerate}
    \item In the first part
    \begin{align*}
        \delta^4(p_1 - p_2 - p_3 - \dots - p_n)
    \end{align*}
    we have a non-zero value \emph{only} when
    \begin{align*}
        p_1 - p_2 - p_3 - \dots - p_n = 0 \\
        \implies \vb p_1 = \vb p_2 + \vb p_3 + \dots + \vb p_n
    \end{align*}
    or the Energy-momentum conservation. 
    \item In the second part
    \begin{align*}
        \delta(p_j^2 - m_j^2 c^2)
    \end{align*}
    we have a non-zero value \emph{only} when
    \begin{align*}
        p_j^2 &- m_j^2 c^2 = 0 \\
        \implies p_j^2 &= m_j^2 c^2 \qquad \forall j = 2, 3, \dots, n
    \end{align*}
    which is true for all real particles (on-shell condition). If this is not true i.e.
    $p_j^2 \neq m_j^2 c^2$ we have a virtual particle. 
    \item In the third part
    \begin{align*}
        \theta(p_j^0)
    \end{align*}
    is non-zero \emph{only} when $p_j^0 > 0$ or $E_j >0$ (positivity of energy). So from the
    energy momentum relation
    \begin{align*}
        E_j^2 &= \vb p_j^2 c^2 + m_j^2 c^4 \\
        \implies E_j &= \pm \sqrt{\vb p_j^2 c^2 + m_j^2 c^4} > 0
    \end{align*}
\end{enumerate}
\paragraph*{Evaluating the integral}
From the delta function
\begin{align*}
    \int \dd{x} \delta(x^2 - a^2) &= \frac{1}{2a} [\delta(x - a) + \delta(x + a)]
\end{align*}
so
\begin{align*}
    \delta(p_j^2 - \vb p_j^2 - m_j^2 c^2) &= \delta(p_j^0 - a^2) \quad a = \sqrt{\vb p_j^2 + m_j^2 c^2} \\
    &= \frac{1}{2a} [\delta(p_j^0 - a) + \delta(p_j^0 + a)] \\
    &= \frac{1}{2\sqrt{\vb p_j^2 + m_j^2 c^2}} \qt[\delta \qt(p_j^0 - \sqrt{\vb p_j^2 + m_j^2 c^2}) 
        + \delta \qt(p_j^0 + \sqrt{\vb p_j^2 + m_j^2 c^2})]
\end{align*}
the second term does not contribute so we are left with
\begin{align*}
    \int \dd{p_j^0} \frac{1}{2\sqrt{\vb p_j^2 + m_j^2 c^2}} \delta \qt(p_j^0 - \sqrt{\vb p_j^2 + m_j^2 c^2}) 
    = \frac{\dd^3 \vb p_j}{2\sqrt{\vb p_j^2 + m_j^2 c^2}}
\end{align*}
so we have removed one of the integrals. Now we are left with the integral
\begin{align*}
    \Gamma = \frac{S}{2m_1 \hbar} \int &\abs{\mathcal{M}}^2 (2\pi)^4 
        \delta(m_1  c - p_2^0 - p_3^0 - \dots - p_n^0) \\
        &\delta^3 (\vb 0 - \vb p_2 - \vb p_3 - \dots - \vb p_n) \\
        & \times \prod_{j=2}^n \frac{\dd^3 \vb p_j}{(2\pi)^3} \frac{1}{2\sqrt{\vb p_j^2 + m_j^2 c^2}}
\end{align*}
and from the energy-momentum relation
\begin{align*}
    \frac{E_j}{c} = \sqrt{\vb p_j^2 + m_j^2 c^2}
\end{align*}

\paragraph*{Example} Two-body decay $1 \to 2 + 3$
\paragraph*{} Sidenote: we cannot have $1 \to 2$ as it would violate the conservation of 4-momentum 
Since the delta function is even, $\delta(\vb x) = \delta(-\vb x)$, so
\begin{align*}
    \Gamma = \frac{S}{2m_1 \hbar} \int &\abs{\mathcal{M}}^2 (2\pi)^4 \delta(m_1 c - E_2/c - E_3/c) 
    \delta^3 (\vb p_2 + \vb p_3) \\
    &\times \frac{\dd^3 \vb p_2}{(2\pi)^3} \frac{1}{2\sqrt{\vb p_2^2 + m_2^2 c^2}} 
    \frac{\dd^3 \vb p_3}{(2\pi)^3} \frac{1}{2\sqrt{\vb p_3^2 + m_3^2 c^2}}
\end{align*}
We have nonzero values when $\vb p_2 = -\vb p_3$ and $E_2 = E_3 = \frac{m_1 c}{2}$. We can use the
delta function to remove the integral over $\vb p_3$ and we are left with
\begin{align*}
    = \frac{S}{2m_1 \hbar} \int &\abs{\mathcal{M}}^2 (2\pi)^4 \delta(m_1 c - (E_2 + E_3)/c) 
    \frac{\dd^3 \vb p_2}{(2\pi)^6} \frac{1}{2\sqrt{\vb p_2^2 + m_2^2 c^2}}
    \frac{1}{2\sqrt{\vb p_2^2 + m_3^2 c^2}}
\end{align*}
and now we can remove one more integral using
\begin{align*}
    \dd^3 \vb p_2 = \abs{\vb p_2}^2 \dd{p_2} \dd{\Omega} \qquad \dd{\Omega} = \sin\theta \dd{\theta} \dd{\phi}
\end{align*}
and we also know that
\begin{align*}
    E_2 = c\sqrt{\abs{\vb p_2}^2 + m_2^2 c^2} \qquad E_3 = c\sqrt{\abs{\vb p_2}^2 + m_3^2 c^2}
\end{align*}
so
\begin{align*}
    \Gamma = \frac{S}{2m_1\hbar} \frac{1}{4(2\pi)^2} \int &\abs{\mathcal{M}}^2 
    \delta(m_1 c - (E_2 + E_3)/c) 
    \frac{\abs{\vb p_2}^2 \dd{\abs{\vb p_2}} \dd{\Omega}}{\sqrt{\abs{\vb p_2}^2 + m_2^2 c^2} \sqrt{\abs{\vb p_2}^2 + m_3^2 c^2}}
\end{align*}
we know the momentums are
\begin{align*}
    p_1 = (m_1 c, \vb 0) \qquad p_2 = (E_2/c, \vb p_2) \qquad p_3 = (E_3/c, -\vb p_2)
\end{align*}
we can construct a scalar out of two vectors using the dot product which is always dependent on
$\abs{\vb p_2}^2$ (there is no angular dependence) so
\begin{align*}
    \abs{\mathcal{M}}^2 (\vb p_2) = f(\abs{\vb p_2}^2)
\end{align*}
so we are left with one integral and one delta function
\begin{align*}
    \Gamma = \frac{S}{2m_1\hbar} \frac{1}{4\pi^2 4} (4\pi) \int_0^\infty \abs{\mathcal{M}}^2
    \delta(m_1 c - (E_2 + E_3)/c) \abs{\vb p_2}^2 \frac{\dd{\abs{\vb p_2}}}
    {\sqrt{\abs{\vb p_2}^2 + m_2^2 c^2} \sqrt{\abs{\vb p_2}^2 + m_3^2 c^2}}
\end{align*}
using a change of variables we can use
\begin{align*}
    u = \sqrt{\abs{\vb p_2}^2 + m_2^2 c^2} + \sqrt{\abs{\vb p_2}^2 + m_3^2 c^2} \\
    \dd{u} = \frac{2\abs{\vb p_2} \dd{\abs{\vb p_2}}}{2\sqrt{\abs{\vb p_2}^2 + m_2^2 c^2}} 
    + \frac{2\abs{\vb p_2} \dd{\abs{\vb p_2}}}{2\sqrt{\abs{\vb p_2}^2 + m_3^2 c^2}}
\end{align*}
and thus we get
\begin{align*}
    \Gamma &= \frac{S}{8m_1 \pi \hbar} \int_{(m_2 + m_3)c}^\infty \abs{\mathcal{M}}^2 \delta(m_1 c - u) \dd{u}
    \frac{\abs{\vb p_2}^2}{u} \\
    &= \frac{S \abs{\vb p}}{8\pi \hbar m_1^2 c} \abs{\mathcal{M}}^2
\end{align*}

\newpage
\lhead{Lecture 17: 3/25/24}
\paragraph*{Quiz review}
\begin{itemize}
    \item A simple delta function integral tells us
    \begin{align*}
        \int_{a - e}^{a + e} f(x) \delta(x - a) \dd{x} = f(a)
    \end{align*}
    \item If the the non zero term is out of bounds of the integral, then the integral is zero!
    \item From the theta function (step function) we know that
    \begin{align*}
        \theta(x) = \begin{cases}
            1 & x > 0 \\
            0 & x < 0
        \end{cases}
    \end{align*}
    and thus
    \begin{align*}
        \theta(2x - 4) = \begin{cases}
            1 & x > 2 \\
            0 & x < 2
        \end{cases}
    \end{align*}
    so we can split the integral from $-1 \to 2$ and $2 \to 5$ and we get
    \begin{align*}
        \int_{-1}^2 0 e^{-3x} \dd{x} &= 0 \\
        \int_2^5 \theta(2x - 4) e^{-3x} \dd{x} &= \int_2^5 e^{-3x} \dd{x} \\
        &= -\frac{1}{3} e^{-3x} \eval_2^5
    \end{align*}
    \item For integration over a sphere we can just find if the magnitude of distance is less than
    the radius of the sphere $1.5$:
    \begin{align*}
        \abs{(2,2,2) - (3,2,1,)} = \sqrt{2} \approx 1.4 < 1.5
    \end{align*}
    so we find the function
    \begin{align*}
        \oint \dd{V} \vb r \cdot (\vb a - \vb r) \delta^3(\vb r - \vb b) 
        &= \int \dd{V} f(\vb r) \delta^3(\vb r - \vb b) \\
        &= f(\vb b)
    \end{align*}
    which is
    \begin{align*}
        f(\vb b) &= \vb b \cdot (\vb a - \vb b) \\
        &= (3,2,1) \cdot [(1,2,3) - (3,2,1)] \\
        &= -4
    \end{align*}
    \item The decay rate using dimensional analysis from last time
    \begin{align*}
        \Gamma = \frac{1}{[\unit{J.s.kg^2.m\per s}]} \cdot \unit{kg.m\per s} \cdot \mathcal{M}
    \end{align*}
    and since $\unit{J} = \unit{kg.m^2\per s^2}$ we can see that the amplitude has units of
    $\unit{kg.m\per s}$ or momentum. Thus the number of particles involved is the only thing that
    is dependent on the number of particles involved.
\end{itemize}

\subsection*{Scattering} 
($2 \to n$ Scattering)
\begin{align*}
    1 + 2 \to 3 + 4 + \dots + n
\end{align*}
the cross section is given by
\begin{align*}
    \sigma = \frac{S\hbar^2}{4\sqrt{(p_1 \cdot p_2)^2} - (m_1 m_2 c^2)^2}
    \int \abs{\mathcal{M}}^2 (2\pi)^4 \delta^4(p_1 + p_2 - p_3 - \dots - p_n) \\
    \times \prod_{j=3}^n \frac{\dd^4 \vb p_j}{(2\pi)^4} (2\pi) \delta(p_j^2 - m_j^2 c^2) \theta(p_j^0) 
\end{align*}
From momentum conservation we have
\begin{align*}
    p^2 = (p^0)^2 - \vb p^2
\end{align*}
so the delta function can be rewritten as
\begin{align*}
    \delta(p_j^2 - m_j^2 c^2) = \delta((p_j^0)^2 - \vb p_j^2 - m_j^2 c^2)
\end{align*}
and using the same trick as last time we can split
\begin{align*}
    \delta(x^2 - a^2) = \frac{1}{2a} [\delta(x - a) + \delta(x + a)]
\end{align*}
or in the general form
\begin{align*}
    \delta(g(x)) = \sum_i \frac{1}{\abs{g'(x_i)}} \delta(x - x_i)
\end{align*}
so defining =
\begin{align*}
    x = p_j^0 \qquad a = \sqrt{\vb p_j^2 + m_j^2 c^2}
\end{align*}
we can rewrite the delta function as
\begin{align*}
    \frac{1}{2\sqrt{\vb p_j^2 + m_j^2 c^2}} \qt[\delta \qt(p_j^0 - \sqrt{\vb p_j^2 + m_j^2 c^2})
    + \delta \qt(p_j^0 + \sqrt{\vb p_j^2 + m_j^2 c^2})]
\end{align*}
and we can remove the second term becase the theta function removes negative energies! So we are left
with
\begin{align*}
    \frac{1}{2\sqrt{\vb p_j^2 + m_j^2 c^2}} \delta(p_j^0 - \sqrt{\vb p_j^2 + m_j^2 c^2})
\end{align*}
Now we we are left with an integral
\begin{align*}
    \int \frac{\dd{p_j^0}}{\cancel{(2\pi)}} \cancel{(2\pi)} \delta(p_j^2 - m_j^2 c^2) \theta(p_j^0)
    f(p_j^0) = f(\sqrt{\vb p_j^2 + m_j^2 c^2})
\end{align*}
which removes the zeroth component of the 4-momentum in the original integral which leaves us with
\begin{align*}
    \sigma = \frac{S \hbar^2}{4\sqrt{(p_1 \cdot p_2)^2 - (m_1 m_2 c^2)^2}} \int \abs{\mathcal{M}}^2
    (2\pi)^4 \delta^4(p_1 + p_2 - p_3 - \dots - p_n) \\
    \prod_{j=3}^n \frac{\dd^3 \vb p_j}{(2\pi)^3} \frac{1}{2\sqrt{\vb p_j^2 + m_j^2 c^2}}
\end{align*}
with
\begin{align*}
    p_j^0 = \sqrt{\vb p_j^2 + m_j^2 c^2} = \frac{E_j}{c}
\end{align*}

\newpage
\paragraph*{2 - 2 Scattering} $1 + 2 \to 3 + 4$

In the center of mass frame the total 3-momentum is zero (HW In the lab frame with one particle at rest initially i.e. $p_2 = (m_2 c, \vb 0)$). 
We have two momenta of the \emph{beam} of particles (LHC)
\begin{align*}
    p_1 = (E_1/c, \vb p_1) \qquad p_2 = (m_2 c, \vb p_2)
\end{align*}
where
\begin{align*}
    p_1 + p_2 = \vb 0 \implies \vb p_1 = -\vb p_2
\end{align*}
which means
\begin{align*}
    \sqrt{(p_1 \cdot p_2)^2 - (m_1 m_2 c^2)^2} = \frac{\abs{\vb p_1}^2}{c} \sqrt{S} \qquad S = (E_1 + E_2)^2
\end{align*}
where $S$ is the Mandelstam variable. So the cross section is
\begin{align*}
    \sigma = \frac{S \hbar^2}{4\frac{\abs{\vb p_1}^2}{c} \sqrt{S}} \int \abs{\mathcal{M}}^2 (2\pi)^4
    \delta^4(p_1 + p_2 - p_3 - p_4) \\
    \frac{\dd^3 \vb p_3}{(2\pi)^3}
    \frac{\dd^3 \vb p_4}{(2\pi)^3}
    \frac{1}{2\sqrt{\vb p_3^2 + m_3^2 c^2}} 
    \frac{1}{2\sqrt{\vb p_4^2 + m_4^2 c^2}}
\end{align*}
and we can remove the delta function by using the energy-momentum relation
\begin{align*}
\delta^4(p_1 + p_2 - p_3 - p_4) = \delta \qt(\frac{E_1 + E_2}{c} - \frac{E_3 + E_4}{c}) 
\delta^3(\vb p_1 + \vb p_2 - \vb p_3 - \vb p_4)
\end{align*}
but since the total momentum is zero i.e. 
\begin{align*}
    \vb p_1 + \vb p_2 = 0
\end{align*}
we can replace the $\dd^3 \vb p_4$ with the $\dd^3{\vb p_3 + \vb p_4}$ and we are left with
\begin{align*}
    \sigma =  \frac{S \hbar^2}{4\frac{\abs{\vb p_1}^2}{c} \sqrt{S}} \int \abs{\mathcal{M}}^2 (2\pi)^4
    \delta \qt(\frac{E_1 + E_2}{c} - \frac{E_3 + E_4}{c}) \\
    \frac{\dd^3 \vb p_3}{(2\pi)^3} \frac{1}{(2\pi)^3} \frac{1}{2\sqrt{\vb p_3^2 + m_3^2 c^2}}
    \frac{1}{2\sqrt{\vb p_3^2 + m_4^2 c^2}}
\end{align*}
and since
\begin{align*}
    \dd^3 \vb p_3 = \abs{\vb p_3}^2 \dd{\abs{\vb p_3}} \dd{\Omega} \qquad \dd{\Omega} = \sin\theta \dd{\theta} \dd{\phi}
\end{align*}
We know that
\begin{align*}
    \vb p_4 = -\vb p_3 \\
    \implies E_4 = \sqrt{\vb p_4^2 c^2 + m_4^2 c^4} = \sqrt{\vb p_3^2 c^2 + m_4^2 c^4}
\end{align*}
so we can represent
\begin{align*}
    \abs{\mathcal{M}}^2 (p_1, p_2, p_3, p_4) &= \abs{\mathcal{M}}^2 (p_3, p_4) \\
    &= \abs{\mathcal{M}}^2 (\vb p_3, \theta, \phi)
\end{align*}
which can't be written as a function of $\abs{\vb p_3}$ so we must use the differential cross section
\begin{align*}
    \dv{\sigma}{\Omega} = \frac{S\hbar^2}{4\abs{\vb p_1^0} \sqrt{S}} \frac{1}{(2\pi)^4} \frac{1}{4}
    \int \abs{\mathcal{M}}^2 \delta \qt(\frac{E_1 + E_2}{c} - \frac{E_3 + E_4}{c}) \\
    \abs{\vb p_3}^2
    \dd{\abs{\vb p_3}} \frac{1}{\sqrt{\vb p_3^2 + m_3^2 c^2} \sqrt{\vb p_3^2 + m_4^2 c^2}}
\end{align*}
using the change of variables we can use
\begin{align*}
    u &= \frac{E_3 + E_4}{c} \\
    &= \sqrt{\vb p_3^2 + m_3^2 c^2} + \sqrt{\vb p_3^2 + m_4^2 c^2} \\
    \dd{u} &= \frac{2\abs{\vb p_3} \dd{\abs{\vb p_3}}}{2\sqrt{\vb p_3^2 + m_3^2 c^2}} 
    + \frac{2\abs{\vb p_3} \dd{\abs{\vb p_3}}}{2\sqrt{\vb p_3^2 + m_4^2 c^2}} \\
    &= \abs{\vb p_3} \dd{\abs{\vb p_3}} \frac{u}{\sqrt{\vb p_3^2 + m_3^2 c^2} \sqrt{\vb p_3^2 + m_4^2 c^2}}
\end{align*}
which is the last part of the integral So
\begin{align*}
    \dv{\sigma}{\Omega} &= \frac{S\hbar^2}{4\abs{\vb p_1^0} \sqrt{S}} \frac{1}{(2\pi)^4} \frac{1}{4}
    \int \abs{\mathcal{M}}^2 \delta \qt(\frac{E_1 + E_2}{c} - \frac{E_3 + E_4}{c}) \dd{u} \frac{1}{u} \abs{\vb p_3} \\
    &= \frac{S\hbar^2 c}{64 \pi^2 \abs{\vb p_1} (E_1 + E_2)} \frac{\abs{\mathcal{M}}^2 \abs{\vb p_3}}{\frac{E_1 + E_2}{c}} \\
    &= \qt(\frac{hc}{8\pi})^2 \frac{S \abs{\mathcal{M}}^2}{(E_1 + E_2)^2} \frac{\abs{\vb p_3}}{\abs{\vb p_1}}
\end{align*}
We find that this cross section is proportional to many things:
\begin{align*}
    \sigma \propto \frac{1}{S}, \qquad
    \sigma \propto \frac{\abs{p_f}}{{\abs{p_i}}}
\end{align*}
But why use the collider like this? 
\begin{itemize}
    \item In the past we used $\sqrt{S} = \qty{91}{GeV}$ (LEP)
    \item $\sqrt{S} = \qty{1.96}{TeV}$ (Tevatron) 
    \item $\sqrt{S} = \qty{13.6}{TeV}$ (LHC)
    \item $\sqrt{S} = \qty{100}{TeV}$ (FCC/SPPC)
\end{itemize}
But we can only find the cross section to grow with $S$ if $\abs{\mathcal{M}}^2$ is independent of $S$.`'

\newpage
\lhead{Lecture 18: 3/27/24}
\paragraph*{Quiz Review}
\subsection*{Feynman Rules}
QED: $e^{\pm}, \gamma$

% feynman diagram of e^+ e^- -> gamma using tikz package
\begin{figure}[ht]
    \begin{center}
        \feynmandiagram [horizontal=a to b] {
      i1 [particle=\(e^{+}\)] -- [anti fermion] a -- [anti fermion] i2 [particle=\(e^{-}\)],
      a -- [photon, edge label=\(\gamma\), momentum'=Time] b,
    };
    \end{center}
    \caption{Not allowed}
\end{figure}
\begin{figure}[ht]
    \centering
    \feynmandiagram[vertical=a to b] {
        i1 [particle=\(e^{-} (p_1)\)] -- [fermion] a -- [fermion] f1 [particle=\(e^{-} (p_3)\)],
        a -- [photon, edge label=\(\gamma\)] b,
        i2 [particle=\(e^{-} (p_2)\)] -- [anti fermion] b -- [anti fermion] f2 [particle=\(e^{-} (p_4)\)],
    };
    \caption{Allowed}
\end{figure}
\begin{figure}[ht]
    \centering
    \feynmandiagram[vertical=a to b] {
        i1 [particle=\(e^{-} (p_1)\)] -- [fermion] a -- [fermion] f1 [particle=\(e^{-} (p_3)\)],
        a -- [photon, edge label=\(\gamma\)] b,
        i2 [particle=\(e^{-} (p_2)\)] -- [anti fermion] b -- [anti fermion] f2 [particle=\(e^{-} (p_4)\)],
    };
    \caption{Allowed}
\end{figure}
is not allowed, but (diagram 3) is allowed only if the initial electron has some KE. But also there
is (diagram 4) due to the symmetry.

\paragraph*{Notation}
\begin{itemize}
    \item Fermion: solid line with a forward arrow for the direction of the particle, and a backward
    arrow for the antiparticle.
    \item Photon: wavy line
    \item Gluon: springy line
    \item $W/Z$ boson: triangle wave
    \item Higgs: dashed line
\end{itemize}
\paragraph*{Rules}
\begin{itemize}
    \item Label the external momenta as $p_i$ and interntal momenta as $q_i$
    \item For the Vertex, insert a factor of $-ig$
    \item Propogator: For each internal line write a factor of 
    \begin{align*}
        \frac{i}{q_j^2 - m_j^2c^2}
    \end{align*}
    (For Virtual particles $q_j^2 \neq m_j^2 c^2$)
    \item 4-momentum conservation: For each vertex, write a factor of \((2\pi)^4 \delta^4(k_1 + k_2 + k_3)\),
    where $k_i$'s are momenta flowing into the vertex. (Total momentum is zero)
    \item Integrate over all interal momenta
    \begin{align*}
        \prod_i \int \frac{\dd[4]{q_i}}{(2\pi)^4}
    \end{align*}
    \item Drop \((2\pi)^4 \delta^4(p_1 + p_2 + \dots - p_m - \dots - p_n)\)
    \item Multiply the final result by $i$
\end{itemize}
\paragraph*{Example Decay} $A \to B + C$
\begin{figure}[ht]
    \centering
    \feynmandiagram[horizontal=a to b] {
        i1 [particle=\(B\)] -- [fermion] a -- [fermion] f1 [particle=\(A\)],
        a -- [fermion] f2 [particle=\(C\)],
    };
    \caption{Decay}
\end{figure}
\begin{align*}
    \mathcal{M} = i (-ig) \cancel{(2\pi)^4 \delta^4(p_1 - p_2 - p_3)} = g
\end{align*}
The decay rate is
\begin{align*}
    \Gamma &= \frac{S}{8\pi m_1^2 \hbar c} \abs{\mathcal{M}}^2 \abs{\vb p_B} \\
    &= \frac{S}{8\pi m_A^2 \hbar c} g^2 \abs{\vb p_B} \\
    \abs{\vb p_B} = \frac{c}{2m_A} \lambda^{1/2}(m_A^2, m_B^2, m_C^2)
\end{align*}
\paragraph*{Example: 2-2 Scattering} $A + A \to B + B$
For the first diagram 
\begin{figure}[ht]
    \centering
    \feynmandiagram[vertical=a to b] {
        i1 [particle=\(A (p_1)\)] -- [fermion] a -- [fermion] f1 [particle=\(B (p_3)\)],
        a -- [edge label=$C(q)$] b,
        i2 [particle=\(A (p_2)\)] -- [anti fermion] b -- [anti fermion] f2 [particle=\(B (p_4)\)],
    };
    \caption{Diagram 1}
\end{figure}
we can take the virtual particle direction to be upwards, we get the amplitude
\begin{align*}
    \mathcal{M}_1 &= i \int (-ig)^2 \frac{i}{q^2 - m_C^2 c^2} (2\pi)^4 \delta^4(p_1 + q - q_3)
    (2\pi)^4 \delta^4(p_2 - q - p_4) \frac{\dd[4]{q}}{(2\pi)^4}
\end{align*}
getting rid of the integral with the first delta function
\begin{align*}
    &= g^2 \frac{1}{(p_3 - p_1)^2 - m_C^2 c^2} (2\pi)^4 \delta^4(p_2 - (p_3 - p_1) - p_4)
\end{align*}
we can drop the factors using rule 6:
\begin{align*}
    = \frac{g^2}{(p_3 - p_1)^2 - m_C^2 c^2}
\end{align*}
From the second diagram we get
\begin{align*}
    \mathcal{M}_2 = \frac{g^2}{(p_4 - p_2)^2 - m_C^2 c^2}
\end{align*}
and the total amplitude is
\begin{align*}
    \mathcal{M} = \mathcal{M}_1 + \mathcal{M}_2 = \frac{g^2}{t^2 - m_C^2 c^2} + \frac{g^2}{u^2 - m_C^2 c^2}
\end{align*}
where
\begin{align*}
    t = (p_1 - p_3)^2,\qquad u = (p_1 - p_4)^2, \qquad s = (p_1 + p_2)^2
\end{align*}
And to find the cross section we use 
\begin{align*}
    \dv{\sigma}{\Omega} = \qt(\frac{\hbar c}{8\pi})^2 \frac{S \abs{\mathcal{M}}^2}{(E_1 + E_2)^2} \frac{\abs{\vb p_f}}{\abs{\vb p_i}}
\end{align*}
and in the center of mass frame we have
\begin{align*}
    m_A = m_B = m, \quad m_C = 0
\end{align*}
and from the conservation of momentum and energy momentum relation
\begin{align*}
    (p_1 - p_3)^2 &= p_1^2 + p_3^2 - 2p_1 p_3 \\
        &= m_A^2 c^2 + m_B^2 c^2 - 2 \qt(\frac{E_1 E_3}{c^2} - \vb p_1 \cdot \vb p_3) \\
        &= (m_A^2 + m_B^2)c^2 - 2 \qt(\frac{E_1 E_3}{c^2} - \abs{\vb p_1} \abs{\vb p_3} \cos\theta)
\end{align*}
where
\begin{align*}
    \vb p_1 + \vb p_2 = \vb 0 \implies \abs{\vb p_1} = \abs{\vb p_2} \\
    \vb p_3 + \vb p_4 = \vb 0 \implies \abs{\vb p_3} = \abs{\vb p_4}
\end{align*}
so from the energy conservation
\begin{align*}
    E_1 + E_2 = E_3 + E_4
\end{align*}
\begin{align*}
    \implies \sqrt{m^2 c^4 + \abs{\vb p_1}^2 c^2} + \sqrt{m^2 c^4 + \abs{\vb p_2}^2 c^2} 
    = \sqrt{m^2 c^4 + \abs{\vb p_3}^2 c^2} + \sqrt{m^2 c^4 + \abs{\vb p_4}^2 c^2}
\end{align*}
so the energies are equivalent and thus we can simplify
\begin{align*}
    (p_1 - p_3)^2 &= 2m^2 c^2 - 2 \qt(\frac{E^2}{c^2} - \abs{\vb p}^2 \cos\theta)
\end{align*}
and using
\begin{align*}
    E^2 = m^2 c^4 + \abs{\vb p}^2 c^2 \\
    \implies \frac{E^2}{c^2} = m^2 c^2 + \abs{\vb p}^2
\end{align*}
we finally get
\begin{align*}
    (p_1 - p_3)^2 = -2\abs{\vb p}^2 (1 - \cos\theta)
\end{align*}
and for the second diagram we get
\begin{align*}
    (p_1 - p_4)^2 = -2\abs{\vb p}^2 (1 + \cos\theta)
\end{align*}
Back to the total amplitude
\begin{align*}
    \mathcal{M} &= \frac{g^2}{-2\abs{\vb p}^2 } \qt(\frac{1}{1 - \cos\theta} + \frac{1}{1 + \cos\theta}) \\
    &= - \frac{g^2}{\abs{\vb p}^2 \sin^2 \theta} 
\end{align*}
and now we can find the cross section:
\begin{align*}
    \dv{\sigma}{\Omega} = \qt(\frac{\hbar c}{8\pi})^2 \frac{1}{2} \frac{1}{4E^2} \frac{g^4}{\abs{\vb p}^4 \sin^4 \theta}
\end{align*}
integrating
\begin{align*}
    \sigma \propto \int_0^\pi \frac{\sin\theta \dd{\theta}}{\sin^4 \theta} \to \infty
\end{align*}
The mediator of the force is the photon $C$, a massless mediator, which is why the cross section is
infinite. In the homework we will see that this will go to $\propto \frac{1}{m_C^4}$.
\paragraph*{Vacuum Polarization} We can have multiple loops in the diagrams and we would get
\begin{align*}
    \mathcal{M} = \int \frac{\dd[4]{q}}{q^4} = \int_0^\infty \frac{q^3 \dd{q}}{q^4} = \infty
\end{align*}
and to get rid of this we use \emph{Regularization} i.e.
\begin{align*}
    \int^M \frac{\dd{q}}{q} = \ln(M)
\end{align*}
known as the Cut-off scale. We will get finite quantities
\begin{align*}
    m = m_0 + \delta m \\
    g = g_0 + \delta g
\end{align*}

\end{document}