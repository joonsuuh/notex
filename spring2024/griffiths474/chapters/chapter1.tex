\documentclass[../main.tex]{subfiles}

\graphicspath{../images/}

\begin{document}

\section{Lecture (1/17/24)}
\barh 

\paragraph{Four Fundamental Forces}

\begin{itemize}
    \item Strong (gluon)
    \item Weak (W, Z)
    \item Electromagnetic (photon)
    \item Gravity (graviton?)
\end{itemize}

The `Standard Model' describe the first three forces and unifies the Strong and Weak Forces known as
the `Electroweak' force. So, the Standard Model does not include gravity.

\paragraph{The Standard Model (SM)}
\begin{itemize}
    \item Basic building blocks: spin 1/2 particles (fermions)

    \item Interaction between then are mediated by force carriers:
    spin 1 particles (vector bosons)
    
    \item How particles get mass? $\rightarrow$ Higgs Boson (spin 0)
\end{itemize}

The Range of Forces:
\begin{itemize}
    \item Strong: $10^{-15}$ m
    \item Weak: $10^{-18}$ ~ $10^{-16}$ m
    \item EM: $1/r^2$
    \item Gravity: $1/r^2$
\end{itemize}

The ranges of forces are related by
\begin{align*}
    R ~ \frac{e^{-r/a}}{r^2}
\end{align*}
where $a \approx 10^{-15}$ m for the Strong and Weak forces.

\paragraph{The Rise of Quantum Field Theory (QFT)}

Relativity + Quantum Mechanics $\rightarrow$ QFT

% 3 x 3 table
\begin{center}
    \begin{tabular}{c|c|c}
        & Macroscopic & Micro \\
        \hline
        SLOW & CM & Quantum Mechanics \\
        \hline
        FAST & Special Relativity & QFT \\
    \end{tabular}
\end{center}

\paragraph{QFT Discoveries}
\begin{itemize}
    \item Existence of anti-particles
    \item Spin-statistics theorem
    \item CPT Theorem (Charge conjugation, Parity, Time reversal)
\end{itemize}

\subsection*{Units!}
\begin{itemize}
    \item Mass: (kg) $\rightarrow$ (eV) from $E = mc^2$
\begin{align*}
    m_e &= \qty{0.5e6}{\electronvolt/c^2} \quad E_n = \frac{\qty{-13.6}{\electronvolt}}{n^2} \\
    m_p &= \qty{1}{\giga\electronvolt/c^2} \quad \qty{1}{\electronvolt} = \qty{1.6e-19}{\joule}
\end{align*}
    \item Momentum: $\frac{eV}{C} \rightarrow p = \frac{E}{c}$
    \item Energy: eV
\end{itemize}

\paragraph{Matter Fermions} are divided into two groups:
\begin{itemize}
    \item Leptons (electrons, muon, tau, neutrinos): Doesn't have the strong force
    \item Quarks (up, down, charm, strange, top, bottom): Feels the strong force 
\end{itemize}
e.g. the proton is made of 2 up quarks and 1 down quark (uud) and the Neurtron is (udd).

\paragraph{Quarks} make up composite subparticles (Hadrons) are held together by the strong force.
\begin{itemize}
    \item Mesons: 1 quark + 1 anti-quark $(q\bar q)$ e.g. pion, kaon...
    \item Baryons: 3 quarks $(qqq)$ e.g proton, neutron
\end{itemize}

Quark charges:
\begin{itemize}
    \item $Q = +2/3$ (up, charm, top)
    \item $Q = -1/3$ (down, strange, bottom)
\end{itemize}

\paragraph{Leptons} are fundamental particles
\begin{itemize}
    \item Charged electrically (-1)
    \begin{itemize}
        \item electron $(\qty{0.5}{\mega\electronvolt})$
        \item muon $(\qty{105}{\mega\electronvolt})$
        \item tau $(\qty{1.8}{\giga\electronvolt})$
    \end{itemize}
    \item Neutral (neutrinos)
    \begin{itemize}
        \item electron neutrino $\nu_e$
        \item mueon neutrino $\nu_\mu$
        \item tau neutrino $\nu_\tau$
    \end{itemize}
\end{itemize}

\paragraph{Crossing Symmetry}
\begin{align*}
    A + B &\rightarrow C + D \quad \textrm{Scattering} \\
    A &\rightarrow B + C + D \quad \textrm{Decay} \\
    A + \bar C &\rightarrow \bar B + D
\end{align*}
e.g. Neutron Decay
\begin{align*}
    n &\rightarrow p + e^- + \bar \nu_e 
\end{align*}
Sum rules to think about:
\begin{itemize}
    \item Baryon Number Conservation
    \item Lepton Number Conservation
    \item Electric Charge Conservation
\end{itemize}
another example:
\begin{align*}
    n + e^+ &\rightarrow p + \bar \nu_e \\
    p + e^- &\rightarrow n + \nu_e
\end{align*}

\href{https://phys.libretexts.org/Bookshelves/University_Physics/University_Physics_(OpenStax)/University_Physics_III_-_Optics_and_Modern_Physics_(OpenStax)/11%3A_Particle_Physics_and_Cosmology/11.03%3A_Particle_Conservation_Laws}{Particle Conservation Laws}

\newpage
\subsection*{Lecture 2: \hfill  1/22/24}
\hrule \vspace{10px}
\section{Relativistic Kinematics}
\hrule \vspace{10px}

\paragraph{Quiz 2 Review}

\begin{enumerate}
    \item The Baryon, Lepton, and Electric Charge are conserved in the Standard Model. 
    \item The Baryon and Lepton number ensure the stability of the proton.
    \item In Neutron Decay $n \rightarrow p + e^- + \bar \nu_e$, the weak force is responsible for
    the decay.
% 5x3 table
\begin{center}
    \begin{tabular}{c|c|c|c|c}
        & Strong & EM & Weak & Gravity \\
        \hline
        Strength & 1 & $10^{-2}$ & $10^{-7}$ & $10^{-40}$ \\
        \hline
        Time scale & $10^{-23}$ sec & $10^{-16}$ & $10^{-10}$ & $>$ yr\\
    \end{tabular}
\end{center}
    The decay rate is proportional to the coupling strength of the force $\Gamma \propto \alpha^2$.
    For the time scale is $\tau$ it is inversely proportional:
    \begin{align*}
        \tau \propto \frac{1}{\Gamma}
    \end{align*}
    \item The strong force is responsible for holding the nucleus together.
    \item 
\end{enumerate}

\paragraph{Experimental Discoveries}

To discover and observe particles, there are typically three ways:
\begin{enumerate}
    \item Scattering (cross section) 
    \item Decay (decay rate or lifetime)
    \item Bound states (binding energy/mass)
\end{enumerate}

\paragraph{Relativistic Kinematics} 4-vectors
\begin{align*}
    x^\mu = (ct, x, y, z) \quad \textrm{space-time} \\
    p^\mu = (E/c, p_x, p_y, p_z) \quad \textrm{momentum}
\end{align*}
where $x^\mu$ and $p^\mu$ are the space-time (position) four-vector and energy-momentum four-vector.

\subparagraph*{NOTE:} Totak four-momentum is conserved in all interactions.

Starting with the lorentz invariant
\begin{align*}
    p^\mu p_\mu = p^2
\end{align*}
using the Einstein-summation convention
\begin{align*}
    p^\mu p_\mu = \sum_{\mu = 0}^3 p^\mu p_\mu = p^2
\end{align*}
and the metric tensor
\begin{align*}
    g^{\mu\nu} = \begin{pmatrix}
        1 & 0 & 0 & 0 \\
        0 & -1 & 0 & 0 \\
        0 & 0 & -1 & 0 \\
        0 & 0 & 0 & -1 \\
    \end{pmatrix}
\end{align*}
we can write the lower momentum vector as
\begin{align*}
    p_\mu = p^\nu g_{\mu\nu} 
\end{align*}
thus
\begin{align*}
    p^\mu p_\mu &= p^\mu p^\nu g_{\mu\nu} \\
    &= \qt(\frac{E}{c})^2 + \vb{p} \cdot \vb{p} (-1) \\
    &= \qt(\frac{E}{c})^2 - \abs{\vb{p}}^2 \\
    &= m^2 c^2
\end{align*}
Using
\begin{align}
    E = \sqrt{\abs{\vb{p}}^2 + m^2 c^4}
\end{align}

\paragraph{Lorentz Transformation} At rest $\vb{p} = 0$ and $E = mc^2$.

In the Galilean transformation in the $x$ direction:
\begin{align*}
    x' &= x - vt \\
    y' = y \\
    z' = z \\
    t' = t
\end{align*}
where we assume absolute time, but in the Lorentz transformation:
\begin{align*}
    x' &= \gamma(\beta ct + x) \quad \beta = \frac{v}{c} \\
    ct' &= \gamma(t - \beta x) \quad \gamma = \frac{1}{\sqrt{1 - \beta^2}} \\
    y' &= y \\
    z' &= z
\end{align*}
In matrix form:
\begin{align*}
    \Lambda = 
    \begin{pmatrix}
        ct' \\
        x' \\
        y' \\
        z' \\
    \end{pmatrix}
    &= \begin{pmatrix}
        \gamma & -\beta\gamma & 0 & 0 \\
        -\beta\gamma & \gamma & 0 & 0 \\
        0 & 0 & 1 & 0 \\
        0 & 0 & 0 & 1 \\
    \end{pmatrix}
    \begin{pmatrix}
        ct \\
        x \\
        y \\
        z \\
    \end{pmatrix}
\end{align*}
and thus $p^\mu p_\mu$ is invariant under Lorentz transformation.

\paragraph{Massless particle:} From the energy momentum relation
\begin{align*}
    E^2 = \abs{\vb{p}}^2 c^2 + m^2 c^4
\end{align*}
The massless particle has energy $E = \abs{\vb{p}}c$. But we have to include the frequency (Planck)
relation from quantum mechanics as well:
\begin{align*}
    E = h \nu = \hbar \omega
\end{align*}
And in the SM photons and neutrinos are massless thus
\begin{align*}
    p^2 = p^\mu p_\mu = m^2 c^2 = 0
\end{align*}

\paragraph{Collisions} Non-relativistic vs. Relativistic

Non-relativistic:
\begin{itemize}
    \item Elastic (KE conserved)
    \item Inelastic (KE not conserved)
\end{itemize}
Relativistic:
\begin{itemize}
    \item Elastic (KE conserved) e.g. particle splitting into two
    \item Inelastic (KE not conserved) or Rest energy and mass e.g. colliding two particles to form
    a new particle
    \begin{itemize}
        \item KE increases (Explosive)
        \item KE decreases (Sticky)
    \end{itemize}
\end{itemize}
In the extreme case:
\begin{align*}
    A &+ B \rightarrow C \quad \textrm{inverse decay} \\
    A &\rightarrow B + C \quad \textrm{decay}
\end{align*}

\paragraph{Example} $\pi^+ \rightarrow \mu^+ + \nu_\mu$ (decay)

The Rest energies are $m_{\pi^+} = \qty{135}{\mega\electronvolt/c^2}$,
$m_{\mu^+} = \qty{105}{\mega\electronvolt/c^2}$, and $m_{\nu_\mu} = 0$. But this energy is lost
through the kinetic energy of the muon and muon-neutrino.

The momentum before is just the momentum of the pion
\begin{align*}
    p_i = p_{\pi} = 0
\end{align*}
since it is startionary. Afterward the momentum is split between the muon and neutrino
\begin{align*}
    p_f = p_{\mu} + p_{\nu_\mu}
\end{align*}
where energy and momentum is conserved:
\begin{align*}
    \vb{p}_\mu &= -\vb{p}_{\nu} \\
    m_{\pi} c^2 &= E_{\mu} + E_{\nu_\mu} 
\end{align*}

\paragraph{4-momentum conservation}
\begin{align*}
    p_{before} &= p_{after} \\
    p_{\pi} &= p_{\mu} + p_{\nu_\mu}
\end{align*}
since the massless particle has no momentum from the energy momentum relation
\begin{align*}
    p_\nu &= p_\pi - p_\mu \\
    p_\nu^2 &= (p_\pi - p_\mu)^2 \\
    &= p_\pi^2 - 2p_\pi p_\mu + p_\mu^2 \\
    0 &= m_\pi^2 c^2 + m_\mu^2 c^2 - 2\frac{m_\pi c^2}{c} \frac{E_\mu}{c} \\
    2E_\mu m_\pi &= (m_\pi^2 + m_\mu^2) c^2 \\
    E_\mu &= \frac{m_\pi^2 + m_\mu^2 - m_\nu^2}{2m_\pi} c^2 \\
\end{align*}

Another way is finding
\begin{align*}
    p_\pi = p_\mu + p_\nu
\end{align*}
rewritten as
\begin{align*}
    p_\mu = p_\pi - p_\nu
\end{align*}
squaring both sides gives
\begin{align*}
    p_\mu^2 = p_\pi^2 - 2p_\pi p_\nu + p_\nu^2
\end{align*}
and since $p_\nu^2 = 0$ we have
\begin{align*}
    p_\mu^2 = p_\pi^2 - 2p_\pi p_\nu
\end{align*}
which implies
\begin{align*}
    m_\mu^2 c^2 = m_\pi^2 c^2 - 2m_\pi E_\nu
\end{align*}
the Planck relation tells us
\begin{align*}
    E_\nu = \abs{\vb{p}_\nu} c = \abs{\vb{p}_\mu} c
\end{align*}
thus
\begin{align*}
    2 m_\pi \abs{\vb{p}_\mu} c = (m_\pi^2 - m_\mu^2) c^2
\end{align*}
and
\begin{align*}
    \abs{\vb{p}_\mu} = \frac{m_\pi^2 - m_\mu^2}{2m_\pi} c
\end{align*}

\paragraph{Scattering experiments}

\begin{itemize}
    \item Head-on collision: (LHC)
    \item Fixed target collision: Beam of protons hitting a target (e.g. Carbon) (SLAC)
\end{itemize}  
From momentum conservation, the head-on collision is more energy efficient as it loses the minimum
amount of energy. The created particle is at rest, thus the energy is the rest energy. But the Fixed
target collision has a higher energy loss since the particle loses energy since the created particle
has kinetic energy.

e.g. The Anti-proton Discovery is due to the Bevatron colliding two protons to create an anti-proton
\begin{align*}
    p + p \rightarrow p + p + p + \bar p
\end{align*}

HW HINT: $E_{cm} < E_{fixed}$

\newpage
\subsection*{Lecture 3: \hfill  1/24/24}
\hrule \vspace{10px}
\section{Symmetries}
\hrule \vspace{10px}

Quiz review:

\paragraph{3.} The Energy of the large mass is 
\begin{align*}
    Mc^2 = E_1 + E_2 = 2\gamma m c^2
\end{align*}
where the energy of the smaller masses are
\begin{align*}
    E = \sqrt{\abs{\vb{p}}^2 c^2 + m^2 c^4} 
\end{align*}
where $\abs{\vb{p}} = \gamma mv$ and $\gamma = \frac{1}{\sqrt{1 - \beta^2}}$. Thus the mass $M >2m$.

\paragraph{4.} Using the same thought from 3. we know that the rest mass of $M$ is greater.

\paragraph{Lorentz Invariant}
\begin{align*}
    p^2 = m^2 c^2
\end{align*}
From \href{https://en.wikipedia.org/wiki/Minkowski_space}{Wikipedia}: this is the lightlike vector.
For the timelike $p^2 > 0$ and spacelike $p^2 < 0$.

\subsection*{Symmetries}

Equilateral triangles are symmetric under 3 axes where we can flip the triangle and it is still the
same. For the square, we have 4 axes, and so and so forth. All of these objects are studied in
\href{https://en.wikipedia.org/wiki/Group_theory}{Group Theory}.

\paragraph{Group Theory} Group is a set of objects satisfying certain properies under an operation.

\paragraph{Properties}
\begin{enumerate}
    \item Closure: For $a, b \in G, \quad a \cdot b \in G$
    \item Identity: For any $a \in G, \quad a \cdot I = I \cdot a = a$
    \item Inverse: For each $a \in G, \quad a \cdot a^{-1} = a^{-1} \cdot a = 1$
    \item Associativity: For $a, b, c \in G, \quad (a \cdot b) \cdot c = a \cdot (b \cdot c)$
    \item (optional) Commutativity: For $a, b \in G, \quad a \cdot b = b \cdot a$ AKA Abelian Group.
    Not all groups are commutatiive and thus are called non-Abelian groups.
\end{enumerate}

\paragraph{Two Types of Groups}
\begin{enumerate}
    \item Finite: Finite number of elements. e.g. $Z_2 = \qt{1, -1} = \qt{I, r}$ where $r^2 = I$
    \item Infinite: Descrete or continuous. e.g. set of integers under addition (discrete), set of
    real numbers under multiplication (continuous), $U(1)$ (continuous)
\end{enumerate}

\paragraph{Examples} For an isoscale triangle $Z_2 = \qt{1, -1}$ and for an equilateral triangle
$Z_3 = \qt{0, 1, 2}$ or the operation mod 3. Which is isomorphic to
\begin{align*}
    \equiv \qt{1, \omega, \omega^2}, \quad  \omega = e^{2\pi i/3}
\end{align*}
For the square
\begin{align*}
    Z_4 = \qt{0, 1, 2, 3,} \equiv \qt{1, i, -1, -i} \qor \qt{1, \omega, \omega^2, \omega^3}
\end{align*}
Thus for $n$ elements.
\begin{align*}
    Z_n = \qt{e^{i2\pi j/n}}, \quad j = 0, 1, \dots, n-1
\end{align*}
where all of these groups are Abelian.

\paragraph{For $n \rightarrow \infty$} We get a circle as it has an infinite number of symmetries.

In addition $j \rightarrow \infty$
\begin{align*}
    \frac{2\pi j}{n} = \theta 
\end{align*}
we get
\begin{align*}
    U = e^{i\theta} = \cos \theta + i \sin \theta
\end{align*}
where $\theta \in \qt[0, 2\pi]$, and we have the $U(1)$ group.
\begin{align*}
    U^\dagger U = I \qquad U^\dagger = (U^*)^T
\end{align*}
where the dagger is the transpose of the complex conjugate (conjugate transpose).

\subsection*{Standard Model}

\begin{align*}
    SU(3)_C \otimes SU(2)_L \otimes U(1)_y
\end{align*}
$U(N)$ set of unitary $N \times N$ matrices (non-Abelian in general except for $N>1$). Taking the 
determinant of the matrix
\begin{align*}
    \det (U^\dagger U) = \det I = 1 
\end{align*}
and 
\begin{align*}
    \det (U^\dagger) \det (U) = 1 \qquad \det (U^{*T}) = \det (U^*) = (\det U)
\end{align*}
and
\begin{align*}
    \abs{\det U}^2 &= 1 \\
    \det U &= e^{i\alpha} \quad \alpha \in [0, 2\pi]
\end{align*}
Choosing the phase angle $\alpha = 0$ we get
\begin{align*}
    \det U = 1 \qquad SU(N) C U(N)
\end{align*}
$\otimes$ is a direct product: Two groups $F$ and $G$. For $f \in F$ and $g \in G$ we have
\begin{align*}
    (f, g) \in F \otimes G
\end{align*}
The $U(1)$ group is related to the photon $\gamma$, the $SU(2)$ group is related to the weak force
$W^\pm, Z^0$, and the $SU(3)$ group is related to the strong force $g$ (gluon).

\paragraph{SU(2)} A set of $2 \times 2$ matrices with a determinant of 1.

Given the theorem
\begin{align*}
    U = e^{iH}
\end{align*}
for the hermitian matrix $H$ where
\begin{align*}
    U^\dagger U = 1 \rightarrow e^{-iH^\dagger} e^{iH} = 1
\end{align*}
thus
\begin{align*}
    H^\dagger = H
\end{align*}
we take the determinant of $U$:
\begin{align*}
    \det U = \det(e^{iH}) = e^{i\Tr H} = 1 = e^0
\end{align*}
thus $\Tr H = 0$. This means that the Hermitian $H$ is traceless.

\subsection*{Pauli Matrices} traceless matrices
\begin{align*}
    \sigma_1 &= \begin{pmatrix}
        0 & 1 \\
        1 & 0 \\
    \end{pmatrix} \\
    \sigma_2 &= \begin{pmatrix}
        0 & -i \\
        i & 0 \\
    \end{pmatrix} \\
    \sigma_3 &= \begin{pmatrix}
        1 & 0 \\
        0 & -1 \\
    \end{pmatrix} \\
\end{align*}
thus we can write the Hermitian matrix as
\begin{align*}
    H = \frac{1}{2} \sum_i \theta_i \sigma_i = \frac{1}{2} \vb{\theta} \cdot \vb{\sigma}
\end{align*}
where we have the group element of $SU(2)$
\begin{align*}
    U = e^{i \vb\theta \cdot \vb\sigma/2}
\end{align*}

\paragraph{From QM}
\begin{align*}
    \vb{S} = \frac{\hbar}{2} \vb{\sigma}
\end{align*}
\begin{align*}
    \qt[S_y, S_z] &= i S_x \\
    \qt[S_z, S_x] &= i S_y \\
    \qt[S_x, S_y] &= i S_z
    \qt[\sigma_i, \sigma_j] = 2i \epsilon_{ijk} \sigma_k
\end{align*}
where $\epsilon_{ijk}$ is the Levi-Civita symbol.
\begin{align*}
    \epsilon_{ijk} = \begin{cases}
        1 & \textrm{if } (i, j, k) \textrm{ is an even permutation of } (1, 2, 3) \\
        -1 & \textrm{if } (i, j, k) \textrm{ interchange any two indices } (3, 2, 1) \\
        0 & \textrm{otherwise any index is repeated}
    \end{cases}
\end{align*}
thus
\begin{align*}
    [S_i, S_j] = i \epsilon_{ijk} S_k
\end{align*}
The Lie Algebra for $SU(2)$ is $SO(3)$ where both groups are isomorphic.
\begin{align*}
    [L_i, L_j] = i \epsilon_{ijk} L_k \qquad \vb{L} = \vb{r} \times \vb{p}
\end{align*}
the generators of $SU(2)$ is $\vb{\sigma}/2$. For $SU(3)$
\begin{align*}
    U = e^{i \vb\theta \cdot \vb\lambda/2}
\end{align*}
where we have the Gell-Mann matrices $\vb\lambda$. In general for $SU(N)$

\paragraph{Addition of Angular Momenta}
\begin{align*}
    \vb{J} = \vb{J}_1 + \vb{J}_2
\end{align*}
\begin{align*}
    [J_i, J_j] = i \epsilon_{ijk} J_k
\end{align*}
and
\begin{align*}
    [J^2, J_i] = 0 \qquad J^2 = J_x^2 + J_y^2 + J_z^2T &= 4\sqrt{\frac{l}{2g}}
    \int_0^{\theta_o} \frac{1}{\sqrt{\cos\theta - \cos\theta_o}} \dd{\theta}
\end{align*}
where $J^2$ is the Casimir operator. Since we have simultaneous eigenstates of $J^2$ and $J_z$ we
can write
\begin{align*}
    \ket{j,j_z}
\end{align*}

% lecture 1/29/24
\newpage
\subsection*{Lecture 4: \hfill  1/29/24}
\hrule \vspace{10px}
\section{Symmetries}
\hrule \vspace{10px}

\paragraph{Quiz 3 Review}
SU(2) is the group of 2x2 unitary matrices with determinant 1. Using the basisc vectors
$\mqty(1 \\ 0)$ and $\mqty(0 \\ 1)$ we can write the group element as
\begin{align*}
    \mqty(a \\ b) = a \mqty(1 \\ 0) + b \mqty(0 \\ 1)
\end{align*}
or the linear combination of the basis vectors. Thus the transformation is
\begin{align*}
    \mqty(a' \\ b') = U(\theta) \mqty(a \\ b) = e^{i\vb{\theta} \cdot \vb{\sigma} / 2} \mqty(a \\ b)
\end{align*}
THe Lie Algebra for SU(2) is
\begin{align*}
    [J_i, J_j] = i \epsilon_{ijk} J_k
\end{align*}
and
\begin{align*}
    [J^2, J_i] = 0
\end{align*}
for simultaneous eigenstates of $\ket{j,m}$.
\begin{align*}
    J_z \ket{j,m} = m\hbar \ket{j,m} \qquad
    J^2 \ket{j,m} = j(j+1)\hbar^2 \ket{j,m}
\end{align*}
from the ladder operators
\begin{align*}
    J_\pm = J_x \pm i J_y
\end{align*}
where these are not Hermitian (does not commute). Thus
\begin{align*}
    J^2 &= J_x^2 + J_y^2 + J_z^2 \\
    &= J_+ J_- + J_+ J_- J_z^2
\end{align*}
furthermore
\begin{align*}
    J_\pm \ket{j,m} = \hbar \sqrt{(j \mp m)(j \pm m)} \ket{j, m \pm 1}
\end{align*}
where going up the ladder $m \rightarrow m + 1$ and going down the ladder $m \rightarrow m - 1$.
For fixed $j$ there is a maximum and minimum $m$ value
\begin{align*}
    m_{max} = j \qquad m_{min} = -j
\end{align*}
so for example
\begin{align*}
    J_+ \ket{j,j} = 0 \qquad J_- \ket{j,-j} = 0
\end{align*}

\subsubsection*{Spin}
\begin{align*}
    j \equiv s = 1/2, \qquad m \equiv m_s = \pm 1/2
\end{align*}
The basis states are
\begin{align*}
    (1/2, 1/2) &= \mqty(1 \\ 0) = \ket{\uparrow}  \qquad m_s = 1/2 \\
    (1/2, -1/2) &= \mqty(0 \\ 1) = \ket{\downarrow} m_s = -1/2
\end{align*}
For the addition of spin
\begin{align*}
    \frac{1}{2} \otimes \frac{1}{2} = ? \qquad \vb{S} = \vb{S}_1 + \vb{S}_2 \qquad 
    S_{tot} = (S_1 + S_2), ... , (S_1 - S_2) = 1, 0 \qquad
    m_{s, tot} = 1, 0, -1, 0
\end{align*}
\subsubsection*{General Addition of Angular Momentum}
\begin{align*}
    \ket{1,1} = \ket{\uparrow \uparrow} \qquad \ket{1,0} = \frac{1}{\sqrt{2}} \ket{\uparrow \downarrow} + \frac{1}{\sqrt{2}} \ket{\downarrow \uparrow} \qquad \ket{1,-1} = \ket{\downarrow \downarrow}
\end{align*}
finding the linear combination through basis transformation by using the resolution of the identity
\begin{align*}
    \ket{j,m} &\to \ket{j_1, m_1} \otimes \ket{j_2, m_2} \\
    &= \sum_{m_1, m_2} \ket{j_1, m_1, j_2, m_2} \bra{j_1, m_1, j_2, m_2} \ket{j,m}
\end{align*}
where the bra-ket is the Clebsch-Gordan coefficient. thus
\begin{align*}
    = \sum_{m_1, m_2} c_{m, m1, m2}^{j, j_1, j_2} \ket{j_1, m_1, j_2, m_2}
\end{align*}
where $m = m_1 + m_2$ and $c_{m, m1, m2}^{j, j_1, j_2}$ is the Clebsch-Gordan coefficient.
\paragraph{Example} For the $S=1$ state $m = 1$
\begin{align*}
    \ket{1,1,} &= \ket{1/2, 1/2} \otimes \ket{1/2, 1/2} \\
    &= \ket{1/2, 1/2, 1/2, 1/2} \\
    &= \ket{\uparrow \uparrow}
\end{align*}
For $m = 0$ we have a linear combination of the basis states
\begin{align*}
    J_- \ket{1,1} &= \hbar \sqrt{2} \ket{1,0} \\
    \qor \ket{1,0} &= \frac{1}{\hbar \sqrt{2}} J_- \ket{1,1}
\end{align*}
the sum of the basis states is
\begin{align*}
    J_-(\ket{1/2, 1/2} \otimes \ket{1/2, 1/2}) &= \hbar \sqrt{(1/2 + 1/2) (1/2 - 1/2 + 1)} 
        \ket{1/2, -1/2} \otimes \ket{1/2, 1/2} \\
        &+ \ket{1/2, 1/2} \otimes \ket{1/2, -1/2} \\
    &= \hbar(\ket{\uparrow \downarrow} + \ket{\downarrow \uparrow})
\end{align*}
or 
\begin{align*}
    \ket{1,0} = \frac{1}{\sqrt{2}} (\ket{\uparrow \downarrow} + \ket{\downarrow \uparrow})
\end{align*}
for $m = -1$ we have
\begin{align*}
    J_- \ket{1,0} &= \hbar \sqrt{2} \ket{1,-1}
\end{align*}
where
\begin{align*}
    \ket{1,-1} &= \ket{1/2, -1/2} \otimes \ket{1/2, -1/2}
    &= \ket{\downarrow \downarrow}
\end{align*}
Now for $S = 0$, $m = 0$ we have
\begin{align*}
    \ket{0,0} &= \frac{1}{\sqrt{2}} (\ket{\uparrow \downarrow} - \ket{\downarrow \uparrow})
\end{align*}
since it is the way to make it orthogonal to $\ket{1,0}$. Therefore
\begin{align*}
    \frac{1}{2} \otimes \frac{1}{2} = 1 \oplus 0
\end{align*}
Thus the there are 3 triplet states $m_s = 1, 0, -1$ and 1 singlet state $m_s = 0$.

\subsection*{Isospin}
\begin{align*}
    m_p = 938.3 \textrm{ MeV/c}^2 \qquad m_n = 939.6 \textrm{ MeV/c}^2
\end{align*}
why are they so close? Heisenberg postulated an isospin state of a nucleon $N$ as
\begin{align*}
    N = \mqty(\alpha \\ \beta) = \alpha \ket{p} + \beta \ket{n}
\end{align*}
with
\begin{align*}
    p = \mqty(1 \\ 0) \qquad n = \mqty(0 \\ 1)
\end{align*}
the isospin state of the proton and neutron are
\begin{align*}
    \ket{p} = \ket{\frac{1}{2}, \frac{1}{2}} \qquad \ket{n} = \ket{\frac{1}{2}, -\frac{1}{2}}
\end{align*}
% shortcut for \frac{1}{2} to \half
\newcommand{\half}{\frac{1}{2}}

\begin{enumerate}
    \item Strong interactions preserve isospin symmetry
    \item EM \& Weak interactions do not preserve isospin symmetry
\end{enumerate}

\subsubsection*{Examples}
Pions: $\pi^+$, $\pi^0$, $\pi^-$ where the approximate symmetry is a triplet state
\begin{align*}
    \pi^+ &= \ket{1, 1} \qquad I = 1, \quad I_3 = 1 \\
    \pi^- &= \ket{1, 0} \qquad I = 1, \quad I_3 = 0 \\ 
    \pi^0 &= \ket{1, -1} \qquad I = 1, \quad I_3 = -1
\end{align*}
$\Delta$-baryons:
\begin{align*}
    \Delta^{++} &= \ket{3/2, 3/2} \qquad I = 3/2, \quad I_3 = 3/2 \\
    \Delta^{+} &= \ket{3/2, 1/2} \qquad I = 3/2, \quad I_3 = 1/2 \\
    \Delta^{0} &= \ket{3/2, -1/2} \qquad I = 3/2, \quad I_3 = -1/2 \\
    \Delta^{-} &= \ket{3/2, -3/2} \qquad I = 3/2, \quad I_3 = -3/2 \\
\end{align*}
where $\Delta^{--}$ is an antiparticle of $\Delta^{++}$. We write from the highest to lowest
from the empirical Gellman-Nishijima formula
\begin{align*}
    Q = I_3 + \frac{1}{2} (B + S)
\end{align*}
where $Q$ is the charge, $I_3$ is the third component of isospin, $B$ is the baryon number, and $S$
is the strangeness.

\subsubsection*{Pions}
Since a Pion is a \emph{meson} and not a baryon, it has a baryon number of 0. Thus with no
strangeness
\begin{align*}
    S = 0 \qquad B = 0
\end{align*}
\subsubsection*{Nucleons}
\begin{align*}
    S = 0 \qquad B = 1
\end{align*}
\begin{align*}
    Q = \begin{cases}
        1/2 + 1/2(1 + 0)= 1 & \textrm{proton} \\
        -1/2 + 1/2(1 + 0)= 0 & \textrm{neutron}
    \end{cases}
\end{align*}
For all elementary particles there is a general formula
\begin{align*}
    Q = I_3 + \frac{Y}{2}
\end{align*}
where $Y$ is the hyper charge $U(1)_Y$.

\subsubsection*{Power of Symmetry: Applications}
\begin{enumerate}
    \item Deuteron (neutron of deuterium): Two-Nucleon system
\begin{align*}
    I = 1 \qor 0 \qquad I_3 = 1, 0, -1 \qor 0 \textrm{ (singlet)}
\end{align*}
\begin{align*}
    \ket{1,1} &= \ket{p, p} \\
    \ket{1,0} &= \frac{1}{\sqrt{2}} (\ket{p, n} + \ket{n, p}) \\
    \ket{1,-1} &= \ket{n, n} \\
    \ket{0,0} &= \frac{1}{\sqrt{2}} (\ket{p, n} - \ket{n, p})
\end{align*}
experimentally, we only see the singlet state because we see only one deuteron state. Thus we can
only see a isospin state of $I = 0$.

\paragraph{Two-nucleon potential} $\propto \vb{I}_1 \cdot \vb{I}_2$ where we hae the total isospin
\begin{align*}
    \vb{I}^2 = (\vb{I}_1 + \vb{I}_2)^2 &= \vb{I}_1^2 + \vb{I}_2^2 + 2 \vb{I}_1 \cdot \vb{I}_2
\end{align*}
where the $s^2$ term is
\begin{align*}
    s^2 = 1/2 (1/2 + 1) \hbar^2 = \frac{3}{4} \hbar^2
\end{align*}
Thus
\begin{align*}
    \vb{I}_1^2 + \vb{I}_2^2 = \frac{3}{2}
\end{align*}
and
\begin{align*}
    \vb{I}_1 \cdot \vb{I}_2 &= \frac{1}{2} \qt(\vb{I}^2 - 3/2)^{3/2} \\
    &= \begin{cases}
        1/2(1(1+1) -3/2) &= 1/4 \quad \textrm{triplet} \\
        1/2(0(0+1) -3/2) &= -3/4 \quad \textrm{singlet}
    \end{cases}
\end{align*}
\end{enumerate}

\pagebreak
\subsection*{Lecture 5: \hfill  1/31/24}
\hrule \vspace{10px}
\section{Symmetries}
\hrule \vspace{10px}

\paragraph{Quiz 5 Review}
For $j$
\begin{align*}
    \frac{1}{2} \otimes \frac{1}{2} = 1 \oplus 0
\end{align*}
For $2j + 1$
\begin{align*}
    2 \otimes 2 = 3 \oplus 1
\end{align*}
Isospins of particles
\begin{enumerate}
    \item pion: 1
    \item deuteron: 0
    \item $\Delta$-baryons: 3/2
    \item nucleons: 1/2
\end{enumerate}
The strong ineteraction preserves $I$ and $I_3$, and the weak interactions do not preserve $I$ and
$I_3$ (e.g. in beta decay the iso spin of the neutron (-1/2) go to an iso spin of the proton (1/2)).
In E\&M the isospin preserves only $I$ and not $I_3$ (e.g. $\pi_o$ decay to two photons
$\gamma \gamma$: $I_3 = 0$ for the $\pi_o$ and $I_3 = 0$ for the two photons).

\paragraph{Applications of Isospin:} Nucleon-nucleon Scattering
\begin{align*}
    p + p &\rightarrow D + \pi^+ \\
    p + n &\rightarrow D + \pi^0 \\
    n + n &\rightarrow D + \pi^-
\end{align*}
The relative probabilities of these processes: we get this from the amplitude $A$ where the 
probability $\abs{A}^2$ is proportional to the cross section $\sigma = \pi r^2$ (the cross section
of a sphere, but this is not a solid sphere and rather a `fuzzy' sphere). With the fact that `strong
interactions preserve isospin' we have the the ratio of the cross sections
\begin{align*}
    \sigma_a : \sigma_b : \sigma_c
\end{align*}
For all three processes the RHS the isospin is
\begin{align*}
    I_{tot} = 0 \otimes 1 = 1
\end{align*}
on the left hand side 
\begin{align*}
    I_{tot} = \frac{1}{2} \otimes \frac{1}{2} = 0 \qor 1
\end{align*}
(a) The ratio of getting an isospin of 1 on the left hand side for the first process
\begin{align*}
    \ket{pp} = \ket{11}
\end{align*}
(c) for the third proccess
\begin{align*}
    \ket{nn} = \ket{1, -1}
\end{align*}
(b) The second is the linear combination of $\ket{10}$ and $\ket{00}$
\begin{align*}
    \ket{pn} = \frac{1}{\sqrt{2}} (\ket{10} + \ket{00})
\end{align*}
the $\ket{00}$ does not contribute to the isospin of 1. Thus the ratio of the probability is
\begin{align*}
    A_a : A_b : A_c = 1 : \frac{1}{\sqrt{2}} : 1
\end{align*}
and the ratio of the cross sections is
\begin{align*}
    \sigma_a : \sigma_b : \sigma_c = 1 : \frac{1}{2} : 1
\end{align*}

\paragraph{Example 3} Pion-nucleon Scattering
\begin{align*}
    \mqty(\pi^+ \\ \pi^0 \\ \pi^-) \quad I = 1, \qquad
    \mqty(p \\ n) \quad I = 1/2
\end{align*}
So the total isospin $j$ is 
\begin{align*}
    I_{tot} = 1/2 \otimes 1 = 3/2 \oplus 1/2
\end{align*}
and for $2j + 1$
\begin{align*}
    3 \otimes 2 = 4 \oplus 2
\end{align*}
The elastic processes are (from $a \to f$)
\begin{align*}
    \pi^+ + p &\rightarrow \pi^+ + p \\
    \pi^0 + p &\rightarrow \pi^0 + p \\
    \pi^- + p &\rightarrow \pi^- + p \\
    \pi^+ + n &\rightarrow \pi^+ + n \\
    \pi^0 + n &\rightarrow \pi^0 + n \\
    \pi^- + n &\rightarrow \pi^- + n
\end{align*}
and the charge-exchange processes are (from $g \to j$)
\begin{align*}
    \pi^+ + n &\to \pi^0 + p \\
    \pi^0 + p &\to \pi^+ + n \\
    \pi^- + p &\to \pi^0 + n \\
    \pi^0 + n &\to \pi^- + p \\
\end{align*}
The states of the $3/2$ isospin are
\begin{align*}
    \ket{3/2, 3/2}, \ket{3/2, 1/2}, \ket{3/2, -1/2}, \ket{3/2, -3/2}
\end{align*}
and the states of the $1/2$ isospin are
\begin{align*}
    \ket{1/2, 1/2}, \ket{1/2, -1/2}
\end{align*}
so we have the following states
\begin{align*}
    \ket{\pi^+ p} &= \ket{1,1} \otimes \ket{1/2, 1/2} = \ket{3/2, 3/2} \\
    \ket{\pi^- n} &= \ket{1,-1} \otimes \ket{1/2, -1/2} = \ket{3/2, -3/2} \\
\end{align*}
for the obvious highest and lowest isospin states. Carrying on\dots
\begin{align*}
    \ket{\pi^+ n} &= \ket{1,1} \otimes \ket{1/2, -1/2}
\end{align*}
this is the linear combination of $\ket{3/2, 1/2}$ and $\ket{1/2, 1/2}$, and so on. To find the 
proportional cross sections we know that
\begin{align*}
    \braket{i}{f} \propto A \qquad \abs{\braket{i}{f}}^2 \propto \sigma
\end{align*}
We can use the Clebsch-Gordan coefficients to find the linear combination of the states. For example
\begin{align*}
    \ket{\pi^+ n} &= \ket{3/2, 1/2} + \ket{1/2, 1/2}
\end{align*}
where the Clebsch-Gordan coefficient is
\begin{align*}
    \braket{3/2, 1/2, 1/2, 1/2}{3/2, 1/2} = \sqrt{\frac{2}{3}}
\end{align*}
e.g. for the $\pi^+ p$ state
\begin{align*}
    \ket{\pi^+ p} &= \ket{3/2, 3/2} \\
    \braket{3/2, 3/2, 1/2, 1/2}{3/2, 3/2} &= 1
\end{align*}
using the lowering operator
\begin{align*}
    J_- \ket{j,m} = \hbar \sqrt{(j + m)(j - m + 1)} \ket{j, m - 1}
\end{align*}
so
\begin{align*}
    J_- \ket{3/2, 3/2} &= \hbar \sqrt{3} \ket{3/2, 1/2}
\end{align*}
applying the lower operator to $J_{1-} + J_{2-}$ we get
\begin{align*}
    J_- \qt(\ket{11} \otimes \ket{1/2, 1/2}) &= \hbar \sqrt{2} \ket{10} \otimes \ket{1/2, 1/2} 
    + \hbar \sqrt{1} \ket{11} \otimes \ket{1/2, -1/2}  \\
    &= \hbar \sqrt{2} \ket{10} \otimes \ket{1/2, 1/2} + \hbar \ket{11} \otimes \ket{1/2, -1/2}
\end{align*}
we then get
\begin{align*}
    \ket{3/2, 1/2} &= \sqrt{2/3} \ket{11} \otimes \ket{1/2, 1/2}
        + \sqrt{1/3} \ket{10} \otimes \ket{1/2, 1/2} \\
    &= \sqrt{2/3} \ket{\pi^+ p} + \sqrt{1/3} \ket{\pi^+ n}
\end{align*}
and the orthogonal state is
\begin{align*}
    \ket{1/2, 1/2} &= \sqrt{2/3} \ket{\pi^+ p} - \sqrt{1/3} \ket{\pi^+ n}
\end{align*}
and so on for the other states. At the end we will find that the ratio of the total cross sections
(adding up the matching elastic and exchange processes) is 3.

The amplitude has a factor
\begin{align*}
    \braket{\pi^+ p}{\pi^+ p} = \braket{3/2, 3/2}{3/2, 3/2} = M_3
\end{align*}
where for example 
\begin{align*}
    (\sqrt{2}{3} \bra{3/2, 1/2} - 1/\sqrt{3} \bra{1/2, 1/2})
    &(\sqrt{2}{3} \ket{3/2, 1/2} - 1/\sqrt{3} \ket{1/2, 1/2}) = \\
    &= 2/3 \braket{3/2, 1/2}{3/2, 1/2} - 1/3 \braket{1/2, 1/2}{1/2, 1/2} \\
    &= 2/3 M_3 - 1/3 M_1
\end{align*}
for $M_3 >> M_1$ the ratio is $4/9$, and for $M_3 << M_1$ the ratio is $1/3$.

\subsubsection*{$SU(3)$}
\begin{align*}
    \mqty(p \\ n) \quad SU(2) \textrm{doublet}
\end{align*}
where the spins are
\begin{align*}
    p&: uud \quad Q_u = 2/3 \\
    n: udd \quad Q_d = -1/3
\end{align*}
For the two spins
\begin{align*}
    \mqty(u \\ d)
\end{align*}
the isospins are
\begin{align*}
    I = 1/2, \quad I_3 = 1/2 \qor -1/2
\end{align*}
for the up and down quarks respectively. In reality we have six quarks
\begin{itemize}
    \item Light quarks: $u, d, s$
    \item Heavy quarks: $c, b, t$
\end{itemize}
For the light quaks we have a $SU(3)$ symmetry
\begin{align*}
    \mqty(u \\ d \\ s)
\end{align*}
the masses are all different:
\begin{align*}
    m_u \approx \qty{2}{\MeV / c^2} \qquad 
    m_d \approx \qty{4}{\MeV / c^2} \qquad 
    m_s \approx \qty{95}{\MeV / c^2}
\end{align*}
so we have to add a flavor symmetry to the $SU(2)$ isospin symmetry:
\begin{align*}
    \mqty(u \\ d \\ s) \to \mqty(u \\ d) \oplus s
\end{align*}
or the $SU(3)$ symmetry
\begin{align*}
    SU(3)_f \to SU(2)_I \otimes U(1)_y
\end{align*}
From $SU(2)$ algebra:
\begin{align*}
    [J_i, J_j] = i \epsilon_{ijk} I_k \qquad J_i = \sigma_i / 2
\end{align*}
For the three pauli matrices
\begin{align*}
    \sigma_1 = \mqty(0 & 1 \\ 1 & 0) \qquad 
    \sigma_2 = \mqty(0 & -i \\ i & 0) \qquad
    \sigma_3 = \mqty(1 & 0 \\ 0 & -1)
\end{align*}
Now for $SU(3)$: We know that the generators
\begin{align*}
    U = e^{i \vb{\theta} \cdot \vb{\lambda}/2}
\end{align*}
For $SU(N)$ we have $N^2 - 1$ generators. For $SU(3)$ we have 8 generators. The Gell-Mann matrices
are 
\begin{align*}
    \lambda_1 &= \mqty(0 & 1 & 0 \\ 1 & 0 & 0 \\ 0 & 0 & 0) \qquad
    \lambda_2 = \mqty(0 & -i & 0 \\ i & 0 & 0 \\ 0 & 0 & 0) \qquad
    \lambda_3 = \mqty(1 & 0 & 0 \\ 0 & -1 & 0 \\ 0 & 0 & 0) \\
    \lambda_4 &= \mqty(0 & 0 & 0 \\ 0 & 0 & 1 \\ 0 & 1 & 0) \qquad
    \lambda_5 = \mqty(0 & 0 & 0 \\ 0 & 0 & -i \\ 0 & i & 0) \qquad
    \lambda_6 = \mqty(1 & 0 & 0 \\ 0 & 0 & 0 \\ 0 & 0 & -1)
\end{align*}
\end{document}