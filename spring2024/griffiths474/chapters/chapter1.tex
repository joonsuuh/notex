\documentclass[../main.tex]{subfiles}

\graphicspath{{../images/}}
\usepackage{ytableau}
\begin{document}

\lhead{Lecture 1: 1/17/24}
\chead{Intro to Particle Physics}
\rhead{PHYS 474}
\hrule
\section{Intro to Particle Physics}
\hrule \vspace{10px}

\paragraph{Four Fundamental Forces}

\begin{itemize}
    \item Strong (gluon)
    \item Weak (W, Z)
    \item Electromagnetic (photon)
    \item Gravity (graviton?)
\end{itemize}

The `Standard Model' describe the first three forces and unifies the Strong and Weak Forces known as
the `Electroweak' force. So, the Standard Model does not include gravity.

\paragraph{The Standard Model (SM)}
\begin{itemize}
    \item Basic building blocks: spin 1/2 particles (fermions)

    \item Interaction between then are mediated by force carriers:
    spin 1 particles (vector bosons)
    
    \item How particles get mass? $\rightarrow$ Higgs Boson (spin 0)
\end{itemize}

The Range of Forces:
\begin{itemize}
    \item Strong: $10^{-15}$ m
    \item Weak: $10^{-18}$ ~ $10^{-16}$ m
    \item EM: $1/r^2$
    \item Gravity: $1/r^2$
\end{itemize}

The ranges of forces are related by
\begin{align*}
    R ~ \frac{e^{-r/a}}{r^2}
\end{align*}
where $a \approx 10^{-15}$ m for the Strong and Weak forces.

\paragraph{The Rise of Quantum Field Theory (QFT)}

Relativity + Quantum Mechanics $\rightarrow$ QFT

% 3 x 3 table
\begin{center}
    \begin{tabular}{c|c|c}
        & Macroscopic & Micro \\
        \hline
        SLOW & CM & Quantum Mechanics \\
        \hline
        FAST & Special Relativity & QFT \\
    \end{tabular}
\end{center}

\paragraph{QFT Discoveries}
\begin{itemize}
    \item Existence of anti-particles
    \item Spin-statistics theorem
    \item CPT Theorem (Charge conjugation, Parity, Time reversal)
\end{itemize}

\subsection*{Units!}
\begin{itemize}
    \item Mass: (kg) $\rightarrow$ (eV) from $E = mc^2$
\begin{align*}
    m_e &= \qty{0.5e6}{\electronvolt/c^2} \quad E_n = \frac{\qty{-13.6}{\electronvolt}}{n^2} \\
    m_p &= \qty{1}{\giga\electronvolt/c^2} \quad \qty{1}{\electronvolt} = \qty{1.6e-19}{\joule}
\end{align*}
    \item Momentum: $\frac{eV}{C} \rightarrow p = \frac{E}{c}$
    \item Energy: eV
\end{itemize}

\paragraph{Matter Fermions} are divided into two groups:
\begin{itemize}
    \item Leptons (electrons, muon, tau, neutrinos): Doesn't have the strong force
    \item Quarks (up, down, charm, strange, top, bottom): Feels the strong force 
\end{itemize}
e.g. the proton is made of 2 up quarks and 1 down quark (uud) and the Neurtron is (udd).

\paragraph{Quarks} make up composite subparticles (Hadrons) are held together by the strong force.
\begin{itemize}
    \item Mesons: 1 quark + 1 anti-quark $(q\bar q)$ e.g. pion, kaon...
    \item Baryons: 3 quarks $(qqq)$ e.g proton, neutron
\end{itemize}

Quark charges:
\begin{itemize}
    \item $Q = +2/3$ (up, charm, top)
    \item $Q = -1/3$ (down, strange, bottom)
\end{itemize}

\paragraph{Leptons} are fundamental particles
\begin{itemize}
    \item Charged electrically (-1)
    \begin{itemize}
        \item electron $(\qty{0.5}{\mega\electronvolt})$
        \item muon $(\qty{105}{\mega\electronvolt})$
        \item tau $(\qty{1.8}{\giga\electronvolt})$
    \end{itemize}
    \item Neutral (neutrinos)
    \begin{itemize}
        \item electron neutrino $\nu_e$
        \item mueon neutrino $\nu_\mu$
        \item tau neutrino $\nu_\tau$
    \end{itemize}
\end{itemize}

\paragraph{Crossing Symmetry}
\begin{align*}
    A + B &\rightarrow C + D \quad \textrm{Scattering} \\
    A &\rightarrow B + C + D \quad \textrm{Decay} \\
    A + \bar C &\rightarrow \bar B + D
\end{align*}
e.g. Neutron Decay
\begin{align*}
    n &\rightarrow p + e^- + \bar \nu_e 
\end{align*}
Sum rules to think about:
\begin{itemize}
    \item Baryon Number Conservation
    \item Lepton Number Conservation
    \item Electric Charge Conservation
\end{itemize}
another example:
\begin{align*}
    n + e^+ &\rightarrow p + \bar \nu_e \\
    p + e^- &\rightarrow n + \nu_e
\end{align*}

\href{https://phys.libretexts.org/Bookshelves/University_Physics/University_Physics_(OpenStax)/University_Physics_III_-_Optics_and_Modern_Physics_(OpenStax)/11%3A_Particle_Physics_and_Cosmology/11.03%3A_Particle_Conservation_Laws}{Particle Conservation Laws}

\newpage

\end{document}