\documentclass[../main.tex]{subfiles}

\graphicspath{{../images/}}

\begin{document}
\hrule
\section{Relativistic Kinematics}
\hrule \vspace{10px}

\lhead{Lecture 2: 1/22/24}
\chead{Relativistic Kinematics}
\rhead{PHYS 474}

\paragraph{Quiz 2 Review}

\begin{enumerate}
    \item The Baryon, Lepton, and Electric Charge are conserved in the Standard Model. 
    \item The Baryon and Lepton number ensure the stability of the proton.
    \item In Neutron Decay $n \rightarrow p + e^- + \bar \nu_e$, the weak force is responsible for
    the decay.
% 5x3 table
\begin{center}
    \begin{tabular}{c|c|c|c|c}
        & Strong & EM & Weak & Gravity \\
        \hline
        Strength & 1 & $10^{-2}$ & $10^{-7}$ & $10^{-40}$ \\
        \hline
        Time scale & $10^{-23}$ sec & $10^{-16}$ & $10^{-10}$ & $>$ yr\\
    \end{tabular}
\end{center}
    The decay rate is proportional to the coupling strength of the force $\Gamma \propto \alpha^2$.
    For the time scale is $\tau$ it is inversely proportional:
    \begin{align*}
        \tau \propto \frac{1}{\Gamma}
    \end{align*}
    \item The strong force is responsible for holding the nucleus together.
    \item 
\end{enumerate}

\paragraph{Experimental Discoveries}

To discover and observe particles, there are typically three ways:
\begin{enumerate}
    \item Scattering (cross section) 
    \item Decay (decay rate or lifetime)
    \item Bound states (binding energy/mass)
\end{enumerate}

\paragraph{Relativistic Kinematics} 4-vectors
\begin{align*}
    x^\mu = (ct, x, y, z) \quad \textrm{space-time} \\
    p^\mu = (E/c, p_x, p_y, p_z) \quad \textrm{momentum}
\end{align*}
where $x^\mu$ and $p^\mu$ are the space-time (position) four-vector and energy-momentum four-vector.

\subparagraph*{NOTE:} Totak four-momentum is conserved in all interactions.

Starting with the lorentz invariant
\begin{align*}
    p^\mu p_\mu = p^2
\end{align*}
using the Einstein-summation convention
\begin{align*}
    p^\mu p_\mu = \sum_{\mu = 0}^3 p^\mu p_\mu = p^2
\end{align*}
and the metric tensor
\begin{align*}
    g^{\mu\nu} = \begin{pmatrix}
        1 & 0 & 0 & 0 \\
        0 & -1 & 0 & 0 \\
        0 & 0 & -1 & 0 \\
        0 & 0 & 0 & -1 \\
    \end{pmatrix}
\end{align*}
we can write the lower momentum vector as
\begin{align*}
    p_\mu = p^\nu g_{\mu\nu} 
\end{align*}
thus
\begin{align*}
    p^\mu p_\mu &= p^\mu p^\nu g_{\mu\nu} \\
    &= \qt(\frac{E}{c})^2 + \vb{p} \cdot \vb{p} (-1) \\
    &= \qt(\frac{E}{c})^2 - \abs{\vb{p}}^2 \\
    &= m^2 c^2
\end{align*}
Using
\begin{align}
    E = \sqrt{\abs{\vb{p}}^2 + m^2 c^4}
\end{align}

\paragraph{Lorentz Transformation} At rest $\vb{p} = 0$ and $E = mc^2$.

In the Galilean transformation in the $x$ direction:
\begin{align*}
    x' &= x - vt \\
    y' = y \\
    z' = z \\
    t' = t
\end{align*}
where we assume absolute time, but in the Lorentz transformation:
\begin{align*}
    x' &= \gamma(\beta ct + x) \quad \beta = \frac{v}{c} \\
    ct' &= \gamma(t - \beta x) \quad \gamma = \frac{1}{\sqrt{1 - \beta^2}} \\
    y' &= y \\
    z' &= z
\end{align*}
In matrix form:
\begin{align*}
    \Lambda = 
    \begin{pmatrix}
        ct' \\
        x' \\
        y' \\
        z' \\
    \end{pmatrix}
    &= \begin{pmatrix}
        \gamma & -\beta\gamma & 0 & 0 \\
        -\beta\gamma & \gamma & 0 & 0 \\
        0 & 0 & 1 & 0 \\
        0 & 0 & 0 & 1 \\
    \end{pmatrix}
    \begin{pmatrix}
        ct \\
        x \\
        y \\
        z \\
    \end{pmatrix}
\end{align*}
and thus $p^\mu p_\mu$ is invariant under Lorentz transformation.

\paragraph{Massless particle:} From the energy momentum relation
\begin{align*}
    E^2 = \abs{\vb{p}}^2 c^2 + m^2 c^4
\end{align*}
The massless particle has energy $E = \abs{\vb{p}}c$. But we have to include the frequency (Planck)
relation from quantum mechanics as well:
\begin{align*}
    E = h \nu = \hbar \omega
\end{align*}
And in the SM photons and neutrinos are massless thus
\begin{align*}
    p^2 = p^\mu p_\mu = m^2 c^2 = 0
\end{align*}

\paragraph{Collisions} Non-relativistic vs. Relativistic

Non-relativistic:
\begin{itemize}
    \item Elastic (KE conserved)
    \item Inelastic (KE not conserved)
\end{itemize}
Relativistic:
\begin{itemize}
    \item Elastic (KE conserved) e.g. particle splitting into two
    \item Inelastic (KE not conserved) or Rest energy and mass e.g. colliding two particles to form
    a new particle
    \begin{itemize}
        \item KE increases (Explosive)
        \item KE decreases (Sticky)
    \end{itemize}
\end{itemize}
In the extreme case:
\begin{align*}
    A &+ B \rightarrow C \quad \textrm{inverse decay} \\
    A &\rightarrow B + C \quad \textrm{decay}
\end{align*}

\paragraph{Example} $\pi^+ \rightarrow \mu^+ + \nu_\mu$ (decay)

The Rest energies are $m_{\pi^+} = \qty{135}{\mega\electronvolt/c^2}$,
$m_{\mu^+} = \qty{105}{\mega\electronvolt/c^2}$, and $m_{\nu_\mu} = 0$. But this energy is lost
through the kinetic energy of the muon and muon-neutrino.

The momentum before is just the momentum of the pion
\begin{align*}
    p_i = p_{\pi} = 0
\end{align*}
since it is startionary. Afterward the momentum is split between the muon and neutrino
\begin{align*}
    p_f = p_{\mu} + p_{\nu_\mu}
\end{align*}
where energy and momentum is conserved:
\begin{align*}
    \vb{p}_\mu &= -\vb{p}_{\nu} \\
    m_{\pi} c^2 &= E_{\mu} + E_{\nu_\mu} 
\end{align*}

\paragraph{4-momentum conservation}
\begin{align*}
    p_{before} &= p_{after} \\
    p_{\pi} &= p_{\mu} + p_{\nu_\mu}
\end{align*}
since the massless particle has no momentum from the energy momentum relation
\begin{align*}
    p_\nu &= p_\pi - p_\mu \\
    p_\nu^2 &= (p_\pi - p_\mu)^2 \\
    &= p_\pi^2 - 2p_\pi p_\mu + p_\mu^2 \\
    0 &= m_\pi^2 c^2 + m_\mu^2 c^2 - 2\frac{m_\pi c^2}{c} \frac{E_\mu}{c} \\
    2E_\mu m_\pi &= (m_\pi^2 + m_\mu^2) c^2 \\
    E_\mu &= \frac{m_\pi^2 + m_\mu^2 - m_\nu^2}{2m_\pi} c^2 \\
\end{align*}

Another way is finding
\begin{align*}
    p_\pi = p_\mu + p_\nu
\end{align*}
rewritten as
\begin{align*}
    p_\mu = p_\pi - p_\nu
\end{align*}
squaring both sides gives
\begin{align*}
    p_\mu^2 = p_\pi^2 - 2p_\pi p_\nu + p_\nu^2
\end{align*}
and since $p_\nu^2 = 0$ we have
\begin{align*}
    p_\mu^2 = p_\pi^2 - 2p_\pi p_\nu
\end{align*}
which implies
\begin{align*}
    m_\mu^2 c^2 = m_\pi^2 c^2 - 2m_\pi E_\nu
\end{align*}
the Planck relation tells us
\begin{align*}
    E_\nu = \abs{\vb{p}_\nu} c = \abs{\vb{p}_\mu} c
\end{align*}
thus
\begin{align*}
    2 m_\pi \abs{\vb{p}_\mu} c = (m_\pi^2 - m_\mu^2) c^2
\end{align*}
and
\begin{align*}
    \abs{\vb{p}_\mu} = \frac{m_\pi^2 - m_\mu^2}{2m_\pi} c
\end{align*}

\paragraph{Scattering experiments}

\begin{itemize}
    \item Head-on collision: (LHC)
    \item Fixed target collision: Beam of protons hitting a target (e.g. Carbon) (SLAC)
\end{itemize}  
From momentum conservation, the head-on collision is more energy efficient as it loses the minimum
amount of energy. The created particle is at rest, thus the energy is the rest energy. But the Fixed
target collision has a higher energy loss since the particle loses energy since the created particle
has kinetic energy.

e.g. The Anti-proton Discovery is due to the Bevatron colliding two protons to create an anti-proton
\begin{align*}
    p + p \rightarrow p + p + p + \bar p
\end{align*}

HW HINT: $E_{cm} < E_{fixed}$
\end{document}