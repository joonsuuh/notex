\documentclass[../main.tex]{subfiles}

\graphicspath{{../images/}}

\begin{document}
\pagestyle{fancy}
\lhead{Lecture 15: 3/26}
\chead{Chapter 8}
\rhead{PHYS 472}

\section*{Chapter 14: Plasmons, Polaritons, and Polarons}
\addcontentsline{toc}{section}{Chapter 14: Plasmons, Polaritons, and Polarons}

\subsection*{E\&M Stuff} 
In E\&M, we extensively study two fields: the electric field $\vb E$ and the magnetic field $\vb B$.
We also have a vector
\begin{align*}
    \vb D = \vb E + 4\pi \vb P
\end{align*} 
where $\vb P$ is the polarization vector, and $\vb D$ is the displacement vector. In a static field,
we see that the divergence of the electric field
\begin{align*}
    \div \vb E = \frac{\rho}{\epsilon_0}
\end{align*}
is equivalent to the ratio of the charge density $\rho$ and the permittivity of free space
$\epsilon_0$. We also know that the curl
\begin{align*}
    \curl \vb E = 0
\end{align*}
is zero as the electric field can be expressed as the gradient of a scalar(Hemholtz) potential. 
Looking at the displacement vector,
\begin{align*}
    \div \vb D = 4\pi \rho_f
\end{align*}
where in CGS units we define
\begin{align*}
    \vb D = \epsilon \vb E
\end{align*}
where the dielectric function $\epsilon(\omega, \vb K)$ has a dependence on frequency and wave
vector which makes it a difficult problem to solve. 
\paragraph*{Plasmon} 
The total charge density
\begin{align*}
    \rho = \rho_{\text{ext}} + \rho_{\text{ind}}
\end{align*}
is the sum of the external charge density and the induced charge density. In CGS units, the 
divergence of the two fields are
\begin{align*}
    \div \vb D &= \rho_{\text{ext}} \\
    \div \vb E &= 4\pi(\rho_{\text{ext}} + \rho_{\text{ind}})\\
        &= 4\pi \rho
\end{align*}
We define the following
\begin{align*}
    D(\vb K) &= \epsilon(\vb K) \vb E(\vb K)
\end{align*}
so the divergence of the electric field and displacement vector are
\begin{align*}
    \div \vb E &= \div [\sum \vb E(\vb K) e^{i\vb K \cdot \vb r}] 
    = 4\pi \sum_K \rho(\vb K) e^{i\vb K \cdot \vb r} 
\end{align*}
and
\begin{align*}
    \div \vb D &= \div[\sum_K \epsilon(\vb K) \vb E(\vb K) e^{i\vb K \cdot \vb r}]
    = 4\pi \sum_K \rho_{\text{ext}}(\vb K) e^{i\vb K \cdot \vb r}
\end{align*}
diving the two equations we find
\begin{align*}
    \epsilon(\vb K) &= \frac{\rho_{\text{ext}}(\vb K)}{\rho(\vb K)}
     = 1 - \frac{\rho_{\text{ind}}}{\rho(\vb K)}
\end{align*}
\paragraph*{Free Electron} In 1D, the EOM of an electron in an electric field is
\begin{align*}
    m \dv[2]{x}{t} = - eE
\end{align*}
where time dependence is harmonic i.e.
\begin{align*}
    x &= x_0 e^{-i\omega t} \\
    \implies -\omega^2 m x_0 &= -eE;\qquad x_0 = \frac{eE}{m\omega^2}
\end{align*}
The polarization, or dipole moment per unit volume of the electron, is
\begin{align*}
    P = -nex_0 = \frac{ne^2E}{m\omega^2}
\end{align*}
where $n$ is the electron density. So the dielectric function is
\begin{align*}
    \epsilon(\omega) = \frac{D}{E} = \frac{E + 4\pi P}{E} = 1 - \frac{4\pi ne^2}{m\omega^2}
\end{align*}
We define the plasma frequency as
\begin{align*}
    \omega_p^2 = \frac{4\pi ne^2}{m} 
\end{align*}
so
\begin{align*}
    \epsilon(\omega) = 1 - \frac{\omega_p^2}{\omega^2}
\end{align*}
\paragraph*{Example} In the background the dielectric constant $\epsilon(\infty)$ then
\begin{align*}
    \epsilon(\omega) &= \epsilon(\infty)\qt[1 - \frac{\bar{\omega}_p^2}{\omega^2}]
\end{align*}
where
\begin{align*}
    \bar{\omega}_p^2 = \frac{4\pi ne^2}{m\epsilon(\infty)}
\end{align*}
\subsection*{Electromagnetic wave} 
From the Poynting vector
\begin{align*}
    \vb S = \vb E \cross \vb B
\end{align*} 
\paragraph*{Aside: 3 Types of Differential Equations}
\begin{itemize}
    \item The wave equation 
    \begin{align*}
        A\laplacian f = \pdv[2]{f}{t}
    \end{align*}
    \item The diffusion equation
    \begin{align*}
        D \laplacian f = \pdv{f}{t}
    \end{align*}
    \item The Poisson equation
    \begin{align*}
        \laplacian f = A
    \end{align*}
\end{itemize}
For EM waves, the wave equation is
\begin{align*}
    \dv[2]{D}{t} = c^2 \laplacian \vb E
\end{align*}
where we have a solution
\begin{align*}
    E \propto e^{i\omega t} e^{i\vb K \cdot \vb r}\qand \vb D = \epsilon \vb E
\end{align*}
so the wave equation tells us the dispersion relation
\begin{align*}
    \omega^2 \epsilon(\omega, \vb K) = c^2 K^2
\end{align*}
This tells us some intersting things
\begin{itemize}
    \item $\epsilon$ is real, $\epsilon > 0$, and for real $K$ and $\omega$ the wave propogates 
    transversely with phase velocity 
    \begin{align*}
        v_p = \frac{c}{\sqrt{\epsilon}}
    \end{align*}
    \item If $\epsilon$ is real and $\epsilon < 0$, then $K$ is imaginary and the wave is damped.
    \item If $\epsilon$ is complex and $\omega$ is real, $\vb K$ is complex and is damped.
\end{itemize}
From the dispersion relation
\begin{align*}
    \epsilon(\omega, \vb K) = 1 - \frac{\omega_p^2}{\omega^2}
\end{align*}
if $\omega < \omega_p$ there is total reflection, and if $\omega > \omega_p$ the material is 
transparent. 
\paragraph*{Metal} In a metal with positve charge density, we apply an electric field to slighlty
displace the electrons and cause them to oscillate. The EOM is
\begin{align*}
    n m \dv[2]{x}{t} = -neE
\end{align*}
This displaces the surface charge density $\sigma = \pm neu$ or a capacitor. Using a gaussian pillbox at the 
two surfaces of the capacitor, we know that
\begin{align*}
    E \cdot S = \frac{\sigma \cdot s}{\epsilon_0} = \frac{\sigma}{\epsilon_0}
\end{align*}
and from Gauss's law
\begin{align*}
    E = 4\pi n u e
\end{align*}
so the wave equation is
\begin{align*}
    nm\dv[2]{u}{t} = -neE = -4\pi ne^2 u \\
    \implies \dv[2]{u}{t} + \omega_p^2 u = 0
\end{align*}
where the frequency is
\begin{align*}
    \omega_p = \sqrt{\frac{4\pi ne^2}{m}}
\end{align*}
We can approximately find that for $10^{23}$ electrons per cubic centimeter(Avogadro's number) we 
get a frequency of roughly $10^{16}$ Hz, and the energy is roughly
\begin{align*}
    \hbar \omega_p \approx \frac{10^{-34} \cdot 10^{16}}{10^{-19}} = \qty{1}{eV}
\end{align*}
Experimentally we find that the plasmon energy is roughly $\qty{10}{eV}$ since the frequency is
$10^{16}$ Hz. 

\end{document}
