\documentclass[../main.tex]{subfiles}

\graphicspath{{../images/}}

\begin{document}
\pagestyle{fancy}
\lhead{Lecture 7: 2/6}
\chead{Chapter 5}
\rhead{PHYS 472}

\section*{Chapter 5: Phonon properties} 
\addcontentsline{toc}{section}{Chapter 5: Phonon properties}

The heat capacity is in general
\begin{align*}
    C_v = \qt(\pdv{U}{T})_v
\end{align*}
The proportionaly of heat capacity in different materials
\begin{itemize}
    \item Metal: $C_v \propto I + Q T^3$
    \item Insulator: $C_v \propto T^3$
\end{itemize}

The Energy of an $N$ particle system is $E = N k_B T$. and the heat capacity is
\begin{align*}
    C_v = \pdv{E}{T} = \omega k_B
\end{align*}
The total energy is
\begin{align*}
    U_{tot} = \sum{k, p} \hbar \omega_{k,p} \langle n_{k,p} \rangle
\end{align*}
where $\langle n_{k,p} \rangle$ is the Bose-Einstein distribution
\begin{align*}
    \langle n_\omega \rangle = \frac{1}{\exp(\frac{\hbar \omega}{k_B T}) - 1}
\end{align*}
where there is no chemical potential. At low temperatures the constant goes to
\begin{align*}
    \exp(\frac{\hbar \omega}{k_B T})
\end{align*}
or a boltzmann distribution.
\begin{align*}
    U_{tot} \sum_{k,p} \frac{\hbar \omega_{k,p}}{\exp(\frac{\hbar \omega_{k,p}}{k_B T}) - 1}
\end{align*}
or into an integral
\begin{align*}
    \int \dd{k} f(\omega)  \to \int \dd{\omega} f(\omega) \frac{1}{\dv{\omega}{k}}
\end{align*}
We can compute this numerically from the phonon dispersion relation. For the Heat capacity of a
solid, this $T^3$ term is in the order of 3 meV. We only need to take the acoustic modes into 
account for finding $\dv{\omega}{k}$ which approximately a constant $C$, so
\begin{align*}
    \int_0^{a} \dd{\omega} \frac{\hbar\omega^2}{\exp(\frac{\hbar \omega}{k_B T}) - 1}
\end{align*}
where we can simplify using the substitution 
\begin{align*}
    x = \frac{\hbar \omega }{k_B T}
\end{align*}
to change the integral from $k$ space to $\omega$ space and the integral becomes
\begin{align*}
    T^2 \int_0^\infty \dd{x} \frac{x}{\exp(x) - 1}
\end{align*}
where the $T^2$ term comes from substituting for $\omega$ twice. So the total energy is
\begin{align*}
    U_{tot} = \sum_{p} \int \dd{\omega} D_p(\omega) \frac{\hbar \omega}{\exp(\frac{\hbar \omega}{k_B T}) - 1}
\end{align*}
where $D_p(\omega)$ is the density of states. Using the substition for $x$ we get
\begin{align*}
    U_{tot} = \sum_p \int \omega \dd{\omega} D_p(\omega) \frac{x}{e^x - 1}
\end{align*}
and the heat capacity is
\begin{align*}
    C_v = \pdv{U_{tot}}{T}
    = k_B \sum_p \int \dd{\omega} D_p(\omega) \frac{x^2 e^x}{(e^x - 1)^2}
\end{align*}
The density of states (DOS) is given by
\begin{align*}
    D(\omega) = \dv{N}{\omega}
\end{align*}
where $N$ is the number of states. For a 3D phonon gas, the total allowed states is
\begin{align*}
    N = \frac{\frac{4}{3} \pi k^3}{\qt(\frac{2\pi}{L})^3}
\end{align*}
Which is equivalent to the volume of a sphere for each unit volume. So the DOS is 
\begin{align*}
    D(\omega) = \dv{N}{\omega} = \dv{N}{k} \dv{k}{\omega} = \frac{V}{2\pi^2} \frac{k^2}{1}
\end{align*}
and the heat capacity is
\begin{align*}
    C_v \propto k_B \sum_p \in \dd{\omega} \frac{k^2V}{2\pi^2} \frac{1}{\dv{\omega}{k}} \frac{x^2 e^x}{(e^x - 1)^2}
\end{align*}
and since $\omega = v k$ and $\dv{\omega}{k} = v$ we get
\begin{align*}
    C_v \propto \sum_p \int_0^{\omega_D} \dd{\omega} \frac{V}{1} \frac{\omega^2}{v^3} \frac{x^2 e^x}{(e^x - 1)^2}
\end{align*}
so we get the number of states
\begin{align*}
    N = \int_0^{\omega_D} \dd{\omega} D(\omega)
\end{align*}
where
\begin{align*}
    \omega_D^2 = \frac{6\pi^2 N}{V}
\end{align*}
the total energy is
\begin{align*}
    U_{tot} = \int_0^{\omega_D} \dd{\omega} \frac{V\omega^2}{2 T^2 v^3} \frac{\hbar \omega}{\exp(\frac{\hbar \omega}{k_B T}) - 1}
\end{align*}
and substituting for $x$ we get
\begin{align*}
    \propto T^4 \int_0^{x_D} \dd{x} x^2 \frac{x}{e^x - 1}
\end{align*}
where we have four $x$ terms that are substituted thus the $T^4$ term. We get Debye's law for the
heat capacity
\begin{align*}
    U_{tot} \propto T^4 f(x_D) \qquad C_v = \pdv{U}{T} \propto T^3
\end{align*}
for low temperature $T \to 0$, $x_D \to \infty$ so the $f(x_D) \to 1$ which will give us Debye's
law. Some constants: $\omega_D$ is the Debye frequency and the Debye temperature is
\begin{align*}
    \theta_D = \frac{\hbar \omega_D}{k_B}
\end{align*}

\paragraph*{Einstein Model}
\begin{align*}
  D(\omega) = N \delta(\omega - \omega_0)
\end{align*}
so we get a simple expression for the total energy
\begin{align*}
    U_{tot} \propto \frac{\hbar \omega_o}{\exp(\frac{\hbar \omega_o}{k_B T}) - 1}
\end{align*}
and the heat capacity is
\begin{align*}
    C_v = \pdv{U}{T}\eval_{T \to 0}
    \propto \frac{1}{T} \frac{\exp(\frac{\hbar \omega_o}{k_B T})}{\qt(\exp(\frac{\hbar \omega_o}{k_B T}) - 1)^2} 
    \to T\exp(\frac{\hbar\omega}{k_B T})
\end{align*}
For the einstein model we get a wrong number because we assumed the density of states is a delta functions
,but it reality it is a constant.

\end{document}