\documentclass[../main.tex]{subfiles}

\graphicspath{{../images/}}

\begin{document}
\pagestyle{fancy}
\lhead{Lecture 14: 2/29}
\chead{Chapter 8}
\rhead{PHYS 472}

\section*{Chapter 8: Semiconductor Crystals}
\addcontentsline{toc}{section}{Chapter 8: Semiconductor Crystals}

Orders of Magnitude:
\begin{itemize}
    \item $\rho \sim \qty{e14}{\ohm\cm}$: Insulator resistivity
    \item $\rho \approx \num{e-2} \sim \qty{e9}{ohm*\cm}$: Semiconductor resistivity
    \item $v_g = \pdv{E}{k}$ is the group velocity
    \item The intrinsic carrier concentration is dependent on
    \begin{align*}
        \frac{E_g}{k_B T}
    \end{align*}
    or the number of carriers
    \begin{align*}
        n_e \propto e^{\frac{-E_g}{k_B T}}
    \end{align*}
    so for a gap energy of $E_g = \qty{0.5}{eV}$ at room temperature $k_B T \approx \qty{0.03}{eV}$
    \item $\num{e-19} \sim \qty{e-20}{e/cm^3}$ is the roughly the number of carriers in a semiconductor.
\end{itemize}
\paragraph*{Doping} Doping silicon with phosphorus (V)--- or n doping--- introduces an extra electron
while doping with boron (III)--- or p doping--- introduces a hole.

\paragraph*{Defect level} Finding this level where the effective mass is almost infinite thus
zero curvature. For Phosphorus doped silicon, the defect level is close to the conduction band
because the extra electron is weakly bound to the phosphorus atom(donor level). The electron is freely moving
in the conduction band which relates to the bound state of the hydrogen atom. The rydberg energy
is roughly
\begin{align*}
    \Delta E \propto \frac{m}{\epsilon^2}
\end{align*}
for the hydrogen atom this energy is roughly $\qty{10}{eV}$, and for silicon the effective mass is
\begin{align*}
    m_e^* \approx 0.1 m_e
\end{align*}
and the dielectric constant is roughly
\begin{align*}
    \epsilon \approx 10
\end{align*}
so 
\begin{align*}
    \Delta E \approx \frac{1}{1000} E_{\text{Ryd}} \approx \qty{10}{meV}
\end{align*}
which is much smaller than the boltzmann factor at room temperature $\qty{25}{meV}$. This allows for
self-doping where the electron can be thermally excited to the conduction band. For a (III)
doped silicon, the defect level is closer to the lower band and the electron jumps up to the level
—or hole jumps down—otherwise known as shallow doping (acceptor level). If the state is closer to
the middle of the band gap, it is called deep.

\paragraph*{Hole} The group velocity for the electron in the conduction band is the slope or 
\begin{align*}
    \pdv{E}{k} < 0
\end{align*}
and for the hole the slope is positive on the left:
\begin{align*}
    \text{hole} > 0
\end{align*}
so the hole has a negative sign in the energy $-E_h$ so the lower hole will move upward like a
bubble in a liquid.
\paragraph*{Motion of electron under electric field} The equation is
\begin{align*}
    v_g = \pdv{\omega}{k} = \frac{1}{\hbar} \grad_{\vb k} \omega (\vb k) \qquad E(k) = \hbar \omega k
\end{align*}
and phase velocity
\begin{align*}
    v_p = \frac{\omega}{k} 
\end{align*}
so a small change in energy is
\begin{align*}
    \delta \epsilon = \dv{\epsilon}{k} \delta k \\
    V_g = \frac{1}{\hbar} \dv{\epsilon}{k}
\end{align*}
so
\begin{align*}
    \delta \epsilon &= - e E V_g \delta t \\
    &= \hbar v_g \delta k \\
    \implies \hbar \dv{k}{\epsilon} &= -e E \\
    \implies \dv{\hbar k}{t} &= -e E = F = -e E - \frac{e}{c} \vb c \cross \vb B
\end{align*}
where $\hbar k$ is the crystal momentum. 
\end{document}