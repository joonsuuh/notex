\documentclass[../main.tex]{subfiles}

\graphicspath{{../images/}}

% sum commands

\begin{document}

\setcounter{section}{5}
\begin{center}
    \addcontentsline{toc}{section}{Homework 5}
    \section*{Homework 5}
    \subsection*{Due 2/27 4pm}
\end{center}
\hrule \vspace{10px}

\paragraph*{1.} 3 Advantages of Counting mRNA Molecules to Measure Gene Expression:
\begin{itemize}
    \item Discrete Numbers: We can quantify the `exact' number of mRNA moleucules that contribute to
    protein expression. We can also compare this to the total number of mRNA molecules to get a better
    picture of how much stuff is expressing the protein.
    \item More stuff: We can account for a parts of the cell that are low in intensity but still contribute to
    the overall expression of the protein. 
    \item We are not counting empty space where there are no mRNA molecules. The intensity
    recording method would count regions of the cell(that do not have any mRNA molecules) as a data
    point which could skew the results.
\end{itemize}
\paragraph*{2.} 3-5 Potential Sources of Error
\begin{itemize}
    \item Is the field of view (FOV) representative of the entire cell? The location of where this
    image is taken with respect to the cell could be a major source of error; we could make a
    mistake by taking a picture of the same small region across multiple cells, and this would not
    accurately represent the entire cell.
    \item Is the labeling process accurate? Does it account for all the mRNA molcules that relate to
    protein expression, and can we be sure that it doens't also label other unrelated molecules 
    which have nothing to do with protein expression? Also 
    \item How do we know what counts as a single molecule in relation to a fluorescing spot? There
    are spots of different intensities and sizes, so it could be possible that a large bright spot
    is actually multiple mRNA molecules.
    \item How do we know that the mRNA molecules are not degrading over time? That is, there is a
    finite time for the mRNA to fluoresce before it degrades, and we could be missing data.
\end{itemize}
\paragraph*{3.} Most problematic: The labeling procees is the most problematic because there is so 
much biological complexity that we can't account for. Unless we have a perfect labeling process,
there can be errors from not accounting for all the mRNA molecules and also accidentally labeling other
molecules that are not related to protein expression. 

\paragraph*{4.} Lets say we have $N$ total number of particles in a volume $V$. The probability of
finding $n_o$ particles in a volume $v_o$ can be given by the binomial distribution:
\begin{align*}
    P(n_o) = \frac{N!}{n_o!(N-n_o)!}f^{n_o}(1-f)^{N-n_o}
\end{align*}
where the frequency
\begin{align*}
    f = \frac{\lambda}{N} = \frac{rT}{N} 
\end{align*}
is the ratio of the average number of particles $\lambda$ to the total number $N$, and $r$ is the
rate of particles entering this volume over a time $T$. After some mathy stuff:
\begin{align*}
    P(n_o) &= \frac{N(N - 1)\dots(N - n_o + 1) \cancel{(N - n_o)!}}{n_o! \cancel{(N-n_o)!}}
        \qt(\frac{\lambda}{N})^{n_o} \frac{\qt(1-\frac{\lambda}{N})^{N}}{\qt(1-\frac{\lambda}{N})^{n_o}} \\
        &= \frac{N(N - 1)\dots(N - n_o + 1)}{N^{n_o}} \qt(1-\frac{\lambda}{N})^{-n_o}
            \frac{\lambda^{n_o}}{n_o!} \qt(1-\frac{\lambda}{N})^{N} 
\end{align*}
and for large number of total particles $N \to \infty$, we have two terms that go to 1:
\begin{align*}
    \frac{N(N - 1)\dots(N - n_o + 1)}{N^{n_o}} &= \frac{N}{N} \frac{N-1}{N} \dots \frac{N-n_o+1}{N} \\
        &= 1 \qt(1 - \frac{1}{N}) \dots \qt(1 - \frac{n_o-1}{N}) \approx 1
\end{align*}
and
\begin{align*}
    \qt(1 - \frac{\lambda}{N})^{-n_o} \to 1
\end{align*}
And from the limit definition of the exponential function:
\begin{align*}
    \lim_{N \to \infty} \qt(1 + \frac{-\lambda}{N})^{N} \to e^{-\lambda}
\end{align*}
So we finally get
\begin{align*}
    P(n_o) = \frac{\lambda^{n_o}}{n_o!}e^{-\lambda}
\end{align*}
thus obeying Poisson statistics.

\paragraph*{5.} This `anomaly' is perhaps due to the fact that the molecule counts are centered 
around the detected nucleus center. In Problem 4, we assumed that this test volume
was a randomly chosen volume, and the researchers at Fancy University have chosen a test volume that
is dependent on focusing around nucleus centers for each molecule count. This method would disregard
volumes that are not related to a nucleus, so we have added a bias in our method of data analysis. 
If we were to exclude this automatic nucleus centering i.e. we define this $\qty{100}{\micro m^3}$
cylinder randomly in our FOV, we may get something closer to Poissonian noise.

\end{document}