\documentclass[../main.tex]{subfiles}

\graphicspath{{../images/}}

% sum commands

\begin{document}

\setcounter{section}{1}
\begin{center}
    \addcontentsline{toc}{section}{Homework 2}
    \section*{Homework 2}
    \subsection*{Due 2/6 12pm}
\end{center}
\hrule \vspace{10px}

\paragraph*{1.}
\begin{align*}
    P(r | \lambda) = \exp(-\lambda) \frac{\lambda^r}{r!}
\end{align*}
(a) Taking the log of the likelihood function:
\begin{align*}
    L(\lambda) = \ln P(r | \lambda) = -\lambda + r \ln \lambda - \ln r!
\end{align*}
finding the maximum by taking the derivative with respect to $\lambda$ and setting it to zero:
\begin{align*}
    \frac{dL}{d\lambda} = -1 + \frac{r}{\lambda} = 0 \implies \hat \lambda = r
\end{align*}
so the maximum likelihood estimate for $\lambda$ is $\hat{\lambda} = r$.

(b) Given the derivative with respect to the function $\ln \lambda$:
\begin{align*}
    \dv{(\ln{\lambda})} u^n = nu^n, \qquad \dv{(\ln{\lambda})} \ln \lambda = 1
\end{align*}
we can find the curvature of the log likelihood function:
\begin{align*}
    \dv{(\ln \lambda)} L(\lambda) &= -\lambda + r = 0 \implies \hat \lambda = r \\
    \dv[2]{(\ln \lambda)} L(\lambda) &= -\lambda = k
\end{align*}
For a normal distribution with width $\sigma$, the curvature is $k = -1/\sigma^2$. So the width is
approximately
\begin{align*}
    \sigma \propto \frac{1}{\sqrt{-k}} = \frac{1}{\sqrt{\lambda}}
\end{align*}
and the 95\% confidence interval at the MLE is approximately
\begin{align*}
    \hat \lambda \pm 2\sigma = r \pm \frac{2}{\sqrt{\hat \lambda}}
\end{align*}
(c) Given the new Poisson distribution
\begin{align*}
    P(r | \lambda) = \exp(-(\lambda + b)) \frac{(\lambda + b)^r}{r!}
\end{align*}
the log likelihood function is
\begin{align*}
    L(\lambda) = -(\lambda + b) + r \ln(\lambda + b) - \ln r!
\end{align*}
and the maximum likelihood estimate for $\lambda$ is
\begin{align*}
    \frac{dL}{d\lambda} = -1 + \frac{r}{\lambda + b} = 0 \implies \hat \lambda = r - b
\end{align*}

\end{document}
