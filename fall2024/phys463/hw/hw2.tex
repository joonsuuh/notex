\documentclass[../main.tex]{subfiles}

\graphicspath{{../images/}}

\begin{document}
\pagestyle{fancy}
\lhead{Homework 2}
\chead{Junseo Shin}
\rhead{PHYS 463}

\setcounter{section}{2}

\paragraph{1.}
\begin{itemize}
    \item [(a)] Given 
    \begin{align*}
        \Omega(E, E + \delta E) &= \frac{N!}{(N/2 - E/(2\mu H))!(N/2 + E/(2\mu H))!} \frac{\delta E}{2\mu H}
    \end{align*}
    and using $\beta = \pdv{\ln\Omega}{E}$ \& stirling approx $\ln N! = N\ln N - N$, so taking the log of the DoS:
    \begin{align*}
        \ln \Omega(E) = \ln N! &- \ln (N/2 - E/(2\mu H))! - \ln (N/2 + E/(2\mu H))! + \ln \frac{\delta E}{2\mu H} \\
        = \ln N! & \color{draculagreen} - (N/2 - E/(2\mu H)) \ln(N/2 - E/(2\mu H)) + (N/2 - \cancel{E/(2\mu H)}) \\
        &- (N/2 + E/(2\mu H)) \ln(N/2 + E/(2\mu H)) + (N/2 + \cancel{E/(2\mu H)}) + \ln \frac{\delta E}{2\mu H} \\
        = \ln N! + N &- (N/2 - E/(2\mu H))\ln(N/2 - E/(2\mu H)) \\
        &- (N/2 + E/(2\mu H))\ln(N/2 + E/(2\mu H)) + \ln \frac{\delta E}{2\mu H} 
    \end{align*}
    Finally taking the partial derivative
    \begin{align*}
        \pdv{\ln \Omega}{E} &= \frac{1}{2\mu H}\ln(N/2 - E/(2\mu H)) - \cancel{\frac{N/2 - E/(2\mu H)}{N/2 - E/(2\mu H)}} \qt(-\frac{1}{2\mu H})\\
        &- \frac{1}{2\mu H}\ln(N/2 + E/(2\mu H)) - \frac{1}{2\mu H} \\
        \beta &= \frac{1}{2\mu H} \qt[\ln(N/2 - E/(2\mu H)) - \ln(N/2 + E/(2\mu H))] \\
        &= \frac{1}{2\mu H} \ln(\frac{N/2 - E/(2\mu H)}{N/2 + E/(2\mu H)}) \\
        &= \frac{1}{2\mu H} \ln(\frac{N/2}{N/2} \frac{1 - E/(N\mu H)}{1 + E/(N\mu H)}) \\
        &= \frac{1}{2\mu H} \ln(\frac{1 - E/(N\mu H)}{1 + E/(N\mu H)}) \\
    \end{align*}
    or using the inverse hyperbolic tangent 
    \begin{align*}
        \arctanh(x) = \frac{1}{2} \ln(\frac{1 + x}{1 - x})
    \end{align*}
    So we set $x = -E/(N\mu H)$
    \begin{align*}
        \beta &= \frac{1}{\mu H} \arctanh(-E/(N\mu H)) 
    \end{align*}
    Thus using $\beta = 1/kT$ and the odd function $\tanh(-x) = -\tanh(x)$
    \begin{align*}
        \frac{\mu H}{kT} &= \arctanh(-\frac{E}{N\mu H}) \\
        \implies \tanh(\frac{\mu H}{kT}) &= -\frac{E}{N\mu H}
    \end{align*}
    and finally
    \begin{align*}
        \boxed{
            E = -N\mu H \tanh(\frac{\mu H}{kT})
        }
    \end{align*}
    \item [(b)] For $T = -T_0 < 0$ the sign of the hyperbolic tangent is negative
    \begin{align*}
        E = -N \mu H \tanh(-\frac{\mu H}{kT_0}) = N \mu H \tanh(\frac{\mu H}{kT_0})
    \end{align*}
    $\implies \boxed{E > 0}$
    \item [(c)] The total magnetic moment is proportional to the difference of parallel and antiparallel spins
    \begin{align*}
        M = \mu(n_1 - n_2) = \mu (n_1 - (N - n_1)) = \mu (2n_1 - N)
    \end{align*}
    and from the previous HW 1 we know that
    \begin{align*}
        n_1 = \frac{1}{2} \qt(N - \frac{E}{\mu H}) 
    \end{align*}
    thus
    \begin{align*}
        M = \mu \qt[(N - \frac{E}{\mu H}) - N] = -\frac{E}{H} 
    \end{align*}
    and therefore
    \begin{align*}
        \boxed{
            M (H, T) = N \mu \tanh(\frac{\mu H}{kT})
        }
    \end{align*}
\end{itemize}

\newpage
\paragraph{2.}
\begin{itemize}
    \item [(a)]  Once again the numnber of ways is
    \begin{align*}
        \Omega(N, n_+) = \frac{N!}{n_+!(N - n_+)!}
    \end{align*}
    And the total length is
    \begin{align*}
        \ell = (n_+ - n_-) d = (2n_+ - N)d \implies n_+ = \frac{1}{2}(N + \ell/d)
    \end{align*}
    so
    \begin{align*}
        \boxed{
            \Omega(\ell) = \frac{N!}{\qt[\frac{1}{2}(N + \ell/d)]! \qt[\frac{1}{2}(N - \ell/d)]!}
        }
    \end{align*}
    \item [(b)] And using
    \begin{align*}
        x = \frac{\ell}{Nd} \implies \ell = xNd
    \end{align*}
    we can equate the two expressions above:
    \begin{align*}
        xNd = (2n_+ - N)d \implies xN = 2n_+ - N \implies n_+ = \frac{N}{2}(1 + x)
    \end{align*}
    So
    \begin{align*}
        \Omega = \frac{N!}{\frac{N}{2}(1 + x)!(N - \frac{N}{2}(1 + x))!} = \frac{N!}{\qt[\frac{N}{2}(1 + x)]!\qt[\frac{N}{2}(1 - x)]!}
    \end{align*}
    and using stirling's approximation
    \begin{align*}
        \ln \Omega = N \ln N - \cancel{N} &- \frac{N}{2}(1 + x) \ln(\frac{N}{2}(1 + x)) + \cancel{\frac{N}{2}(1 + x)} \\
        &- \frac{N}{2}(1 - x) \ln(\frac{N}{2}(1 - x)) + \cancel{\frac{N}{2}(1 - x)} \\
        = N \ln N &- \frac{N}{2}(1 + x) \ln(\frac{N}{2}(1 + x)) - \frac{N}{2}(1 - x) \ln(\frac{N}{2}(1 - x))
    \end{align*}
    and from
    \begin{align*}
        -\qt(\pdv{S}{\ell})_E &= - k \qt(\pdv{\ln \Omega}{\ell})_E \\
        \implies S &= k \ln \Omega
    \end{align*}
    or
    \begin{align*}
        \boxed{
            S = k \qt[N \ln N - \frac{N}{2}(1 + x) \ln(\frac{N}{2}(1 + x)) - \frac{N}{2}(1 - x) \ln(\frac{N}{2}(1 - x))]
        }
    \end{align*}
    \newpage
    \item [(c)] Using
    \begin{align*}
        \pdv{S}{\ell} = \frac{S}{\partial x} \pdv{x}{\ell} = \pdv{S}{x} \frac{1}{Nd} 
    \end{align*}
    So the force relation is
    \begin{align*}
        \frac{F}{T} = -\pdv{S}{\ell} &= -\pdv{S}{x} \pdv{x}{\ell} \\
        = -k(
            0 &- \frac{N}{2} \qt[
                \ln(\frac{N}{2}(1 + x)) + \cancel{\frac{1 + x}{1 + x} \frac{2}{N}}
            ] \\ &- \frac{N}{2} \qt[
                -\ln(\frac{N}{2}(1 - x)) + \cancel{\frac{1 - x}{1 - x} \frac{2}{N} (-1)}
            ]
        ) \pdv{x}{\ell} \\
        = k \frac{N}{2} &\qt(\ln \frac{1 + x}{1 - x}) \pdv{x}{\ell} 
    \end{align*}
    and since for small $x$ we can Taylor expand (or see graphically)
    \begin{align*}
        \ln (1 + x) \approx x \quad \ln (1 - x) \approx -x
        \implies \ln \frac{1 + x}{1 - x} \approx 2x
    \end{align*}
    So
    \begin{align*}
        \frac{F}{T} = \frac{kN}{2} (2x) \frac{1}{Nd} = \frac{kx}{d}
    \end{align*}
    and finally $x = \ell/(Nd)$ thus
    \begin{align*}
        \boxed{
            F \approx \frac{kT\ell}{Nd^2}
        }
    \end{align*}
\end{itemize}

\newpage
\paragraph{3.}
From the 2nd law of thermodynamics the total entropy $S_T$ of the thermally isolated system must increase i.e.
\begin{align*}
    \Delta S_T = \Delta S + \Delta S' \geq 0
\end{align*}
and since system $A$ absorbs heat $Q$ from system $A'$ the change in entropy of system $A'$ is
\begin{align*}
    \Delta S' = -\frac{Q}{T'}
\end{align*}
thus
\begin{align*}
    \Delta S + \Delta S' = \Delta S - \frac{Q}{T'} \geq 0
\end{align*}
which implies
\begin{align*}
    \boxed{
        \Delta S \geq \frac{Q}{T'}
    }
\end{align*}

\newpage
\paragraph{4.} A glass bulb with air at room temp and 1 atm is placed in to a chamber with helium at 1 atm.
Since the glass bulb is only permeable to helium, so over time, the pressure of helium outside the bulb will have to equal the
partial pressure of helium inside the bulb. Thus the final pressure inside the bulb after equilibrium is
\begin{align*}
    \boxed{P_0 + P_\text{Helium out} = \qty{2}{atm}}
\end{align*}

\newpage
\paragraph{5.} $m_c = 750$ g copper calorimeter can contains $m_w = 200$ g of water in equilibrium at $T_i = 293$ K.
$m_\text{ice} = 30$ g of ice at $T_\text{ice} = 273$ K is placed in the calorimeter and enclosed in a heat-insulating shield.
\begin{itemize}
    \item [(a)] Given
    \begin{align*}
        c_w &= \qty{4.18}{J/gK} \quad c_c = \qty{0.418}{J/gk} 
    \end{align*}
    After the ice melts and reaches equilibrium the final temperature of the water and calorimeter must be equal
    \begin{align*}
        \Delta Q &= Q_{ice} + Q_\text{melted ice} + Q_{water} + Q_{copper} = 0 \\
        &= m_\text{ice} L_f + m_\text{ice} c_w (T_f - T_\text{ice}) + m_w c_w (T_f - T_i) + m_c c_c (T_f - T_i) \\
        &= \qty{30}{g} \times \qty{333}{J/g} + \qty{30}{g} \times \qty{4.18}{J/gK} (T_f - \qty{273}{K}) \\
            &+ \qty{200}{g} \times \qty{4.18}{J/gK} (T_f - \qty{293}{K}) + \qty{750}{g} \times \qty{0.418}{J/gK} (T_f - \qty{293}{K}) \\
        &= 9990 + 125.4 (T_f - 273) + 836 (T_f - 293) + 313.5 (T_f - 293) \\
        \implies T_f &= \qty{283}{K}
    \end{align*}
    \item [(b)] The total entropy for ice melting at $T = 273$ K is 
    \begin{align*}
        \Delta S_\text{ice} = \frac{Q_\text{ice}}{T_\text{ice}} = \frac{m_\text{ice} L_f}{T_\text{ice}}
    \end{align*}
    and for three other processes
    \begin{align*}
        \Delta S_a = \int_{T_0}^{T_f} \frac{m_a c_a}{T} dT = mc \ln \frac{T_f}{T_i}
    \end{align*}
    so
    \begin{align*}
        \Delta S &= \frac{m_\text{ice} L_f}{273} + m_\text{ice} c_w \ln \frac{283}{273} + m_w c_w \ln \frac{283}{293} + m_c c_c \ln \frac{283}{293} \\
        \Delta S &= \qty{1.19}{J/K}
    \end{align*}
    \item [(c)] The work required to bring the water back to $T_i = 293$ K i.e. $\Delta T = T_f - T_i = 293 - 283 = 10$ K
    \begin{align*}
        W &= \Delta Q \\
        &= (m_w + m_i) c_w (T_f - T_i) + m_c c_c (T_f - T_i) \\
        &= 230 (4.18) 10 + 750 (0.418) 10 \\
        W &= \qty{12749}{J}
    \end{align*}
\end{itemize}

\end{document}