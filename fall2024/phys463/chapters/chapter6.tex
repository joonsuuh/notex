\documentclass[../main.tex]{subfiles}

\graphicspath{{../images/}}

\begin{document}
\pagestyle{fancy}
\lhead{Lecture 13: 10/10/24}
\chead{Chapter 6.1-6.2}
\rhead{PHYS 463}

\section{Stat Mech Results and Methods}

Our Return to th stat mech part\dots with systems $A$ and heat reservoir $A'$ where
\begin{align*}
    A \ll A'
\end{align*}

\paragraph{} What is the prob of finding system $A$ in a ny particular microstate $r$ with energy $E_r$?
\begin{align*}
    E_r + E' = E^{(0)}, \implies E' = E^{(0)} - E_r
\end{align*}
And from the DoS the number of states in $A'$ is
\begin{align*}
    \Omega'(E^{(0)} - E_r) 
\end{align*}
or the Multiplicity of $A'$ given $E_r$. The prob $P_r$ has a proportionality
\begin{align*}
    P_r \propto C' \Omega'(E^{(0)} - E_r)
\end{align*}
Since $A \ll A'$ and $E_r \ll E^{(0)}$ we can take the log and Taylor expand
\begin{align*}
    \ln\Omega'(E^{(0)} - E_r) = \ln\Omega'(E^{(0)}) - \pdv{\ln\Omega'}{E'}\eval_{E^{(0)}} E_r
\end{align*}
where the derivative is the thermodynamic beta
\begin{align*}
    \pdv{\ln\Omega'}{E'}\eval_{E^{(0)}} = \frac{1}{kT} = \beta
\end{align*}
which is independent of $E_r$. So taking the exponential again\dots
\begin{align*}
    \Omega'(E^{(0)} - E_r) = \Omega'(E^{(0)})e^{-\beta E_r} = C e^{-\beta E_r}
\end{align*}
where the $\Omega'(E^{(0)})$ is a constant, i.e.
\begin{align*}
    P_r = C e^{-\beta E_r}
\end{align*}
This must be normalized by
\begin{align*}
    \sum P_r = 1
\end{align*}
or
\begin{align*}
    C = \frac{1}{\sum e^{-\beta E_r}}
\end{align*}
where the ``Partition Function'' is
\begin{align*}
    Z \equiv \sum_r e^{-\beta E_r}
\end{align*}
coined by Planck (1920) as ``Zustandsumme'' or ``Sum over all states''.

The probability is
\begin{align*}
    P_r = \frac{e^{-\beta E_r}}{Z}
\end{align*}
Where we have a ``Boltzmann funtion'' $e^{-\beta E_r}$ and $P_r$ is the cannonincal distribution.

\newpage
\lhead{Lecture 14: 10/15/24}
\chead{Chapter 6.4-6.6}
The probability of $A$ having energy $E$ is given by
\begin{align*}
    P(E) = \frac{\Omega{E} e^{-\beta E}}{Z} 
\end{align*}
where the partition function $Z$ is incredibly useful for
\begin{itemize}
    \item Average energy: From $P_r, E_r$
    \begin{align*}
        \bar E = \sum_r P_r E_r = \frac{\sum_r e^{-\beta E_r} E_r}{Z}, \quad Z = \sum_r e^{-\beta E_r}
    \end{align*}
    Using the mathematically useful fact
    \begin{align*}
        \pdv{Z}{\beta} = -\sum_r E_r e^{-\beta E_r}
    \end{align*}
    So we can get
    \begin{align*}
        \bar E = -\frac{1}{Z} \pdv{Z}{\beta}
    \end{align*}
    which is just equivalent to
    \begin{align*}
        \boxed{
            \bar E = -\pdv{\beta}(\ln Z)
        }
    \end{align*}
    The variance of the change in energy is (NOTE: the square is inside because $(\overline{\Delta E})^2 = 0$ )
    \begin{align*}
        \overline{(\Delta E)^2} = \overline{(E - \bar E)^2} = \overline{E^2} - \bar E^2
    \end{align*}
    where
    \begin{align*}
        \overline{E^2} = \sum_r P_r E_r^2 = \frac{\sum_r e^{-\beta E_r} E_r^2}{Z}
    \end{align*}
    The top part is equivalent to the second derivative of $Z$
    \begin{align*}
        \pdv[2]{Z}{\beta} = \sum_r e^{-\beta E_r} E_r^2 
    \end{align*}
    So
    \begin{align*}
        \overline{E^2} = \frac{1}{Z^2} \qt(\pdv{Z}{\beta})^2 = \pdv{\beta}(\frac{1}{Z} \pdv{Z}{\beta}) + \frac{1}{Z} \qt(\pdv{Z}{\beta})^2 = -\pdv{\bar E}{\beta} + \bar E^2
    \end{align*}
    where we get second part of the variance from above
    \begin{align*}
        \bar E^2 = \frac{1}{Z^2} \qt(\pdv{Z}{\beta})^2
    \end{align*}
    Thus
    \begin{align*}
        \overline{(\Delta E)^2} = -\pdv{\bar E}{\beta} = \pdv[2]{\ln Z}{\beta}
    \end{align*}
    \item Change in Work:
    \begin{align*}
        \dbar W &= \frac{\sum e^{\-beta E_r} \qt(\pdv{E_r}{x} dx)}{Z} \\
        &= \frac{1}{\beta} \pdv{\ln Z}{x} dx \\
        \dbar W &= \bar X dx, \quad \boxed{\bar X = \frac{1}{\beta} \pdv{\ln Z}{x}}
    \end{align*}
\end{itemize}

\chead{Chapter 6.3}
\paragraph{Example:} A spin-$\frac{1}{2}$ particle (or a two-level system) in a magnetic field $B$ with magnetic moment $\mu$ so the two states are
\begin{itemize}
    \item $\mu B \to \ket{+}$
    \item $-\mu B \to \ket{-}$
\end{itemize}
The partition function is
\begin{align*}
    Z = \sum_r e^{-\beta E_r} = e^{\beta \mu B} + e^{-\beta \mu B} = 2\cosh(\beta \mu B)
\end{align*}
Thus
\begin{align*}
    \bar E = -\pdv{\ln Z}{\beta} = -\frac{1}{Z} \pdv{Z}{\beta} = -\mu B \frac{\sinh(\beta \mu B)}{\cosh(\beta \mu B)} = -\mu B \tanh(\beta \mu B)
\end{align*}
And the temperature limits
\begin{align*}
    T \to 0, \bar E = -\mu B \\
    T \to \infty, \bar E = 0
\end{align*}
\paragraph{Example:} Harmonic Oscillator
\begin{align*}
    E = \qt(n + \frac{1}{2}) \hbar \omega
\end{align*}
Thus the partition function is
\begin{align*}
    Z &= \sum e^{-\beta E_r} = \sum_{n = 0}^\infty e^{-\beta (n + 1/2) \hbar \omega} \\
    &= e^{\frac{1}{2} \beta \hbar \omega} \sum \qt(
        e^{-\beta \hbar \omega}
    )^n
\end{align*}
We can simplify the summation using a geometric series
\begin{align*}
    \sum_n x^2 = 1 + x + x^2 + \cdots = \frac{1}{1 - x}
\end{align*}
Therefore
\begin{align*}
    Z = \frac{e^{-\frac{1}{2} \beta \hbar \omega}}{1 - e^{-\beta \hbar \omega}}
\end{align*}
and
\begin{align*}
    \bar E = -\pdv{\ln Z}{\beta} = \frac{1}{2} 
    \hbar \omega+ \frac{\hbar \omega}{e^{\beta \hbar \omega} - 1}
\end{align*}
Looking at the temperature limits
\begin{itemize}
    \item $T \to 0, \beta \to \infty \quad \bar E = \frac{1}{2} \hbar \omega$
    \item $T \to \infty, \beta \to 0 \quad \bar E \to \infty$
\end{itemize}

\newpage
\lhead{Lecture 15: 10/17/24}
\chead{Chapter 6.7-6.9 and Chapter 7.2-7.3 (also 9.9-9.10)}

\paragraph{Example:} Particle in a Box: the solution in 1D is
\begin{align*}
    E_n = \frac{n^2 \pi^2 \hbar^2}{2mL^2}
\end{align*}
where we get this from the Schr\"odinger equation wavefunction for a infinite well
\begin{align*}
    \psi(x) = A \sin(\frac{n_x\pi x}{L_x}),\quad k_x = \frac{n_x\pi}{L_x}
\end{align*}
with boundary conditions $\psi(0) = \psi(L) = 0$. For the 3D box we remember
\begin{align*}
    \psi = A \sin(k_x x) \sin(k_y y) \sin(k_z z)
\end{align*}
where the momentum is $p = \hbar k$ and the energy is
\begin{align*}
    E = \frac{p^2}{2m} = \frac{\hbar^2 k^2}{2m} = \frac{\hbar^2}{2m} \qt(k_x^2 + k_y^2 + k_z^2)
\end{align*}

\paragraph{} In $k$-space the volume of a point (microstate) is 
\begin{align*}
    k_x k_y k_z = \frac{\pi^3}{L_x L_y L_z} = \frac{\pi^3}{V}
\end{align*}
We assume the box is large, so in the range $k \to k + dk$. The volume of the ``orange peel'' in this range divided by the volume in $k$-space is
\begin{align*}
    \Omega(k) = \frac{1}{8} \frac{4\pi k^2 dk}{\pi^3 / V} = \frac{V}{2\pi^2} k^2 dk
\end{align*}
Note: the $k$-space is spherical since the vector basis is
\begin{align*}
    k = \sqrt{k_x^2 + k_y^2 + k_z^2}
\end{align*}
and we only deal with positive $k_i$ i.e. the positve $x,y,z$ octant in the Cartesian $k$-space.

The partition function is then
\begin{align*}
    Z &= \int_0^\infty \Omega(k) e^{-\beta \frac{\hbar^2 k^2}{2m}} dk = \frac{V}{2\pi^2} \int_0^\infty k^2 e^{-\beta \frac{\hbar^2 k^2}{2m}} dk \\
    &= \frac{V}{2\pi^2} \frac{\sqrt{\pi}}{4} \qt(\frac{2m}{\beta\hbar})^{3/2} 
\end{align*}
or
\begin{align*}
    \boxed{
        Z = \qt(\frac{2m}{\hbar^2\pi})^{3/2} \beta^{-3/2} V
    }
\end{align*}

\paragraph{Multiple Particles:} The system
\begin{align*}
    A^{(0)} &= A + A', \qqtext{For} A^{(0)}, \quad E_{r,s}^{(0)} = E_r + E_s'
\end{align*}
and
\begin{align*}
    A' \text{ has } E_s, \quad Z^{(0)} = \sum_{r,s} e^{-\beta E_{r,s}^{(0)}}
\end{align*}
The partition function
\begin{align*}
    Z^{(0)} &= \sum_{r,s} e^{-\beta(E_r + E_s')} \\
    &= \sum_r e^{-\beta E_r} \sum_s e^{-\beta E_s'} = Z Z'
\end{align*}
So for $N$ particles
\begin{align*}
    Z_N = \qt[
        \qt(\frac{2m}{\hbar^2\pi})^{3/2} \beta^{-3/2} V
    ]^N
\end{align*}
The average Energy is now
\begin{align*}
    \bar E = -\pdv{\ln Z_N}{\beta} = \frac{3}{2} N kT
\end{align*}

\paragraph{Density of States} For average pressure is equivalent to the generalized force so
\begin{align*}
    \bar P &= \frac{1}{\beta} \pdv{(\ln Z)}{V} = \frac{1}{\beta} N \frac{1}{V} \\
    &= \frac{NkT}{V}
\end{align*}
thus
\begin{align*}
    \bar P V = NkT 
\end{align*}

\paragraph{Connection to Thermo} For $Z(\beta, x)$ the differential is mathematically stated by
\begin{align*}
    d\ln Z(\beta, x) = \pdv{\ln Z}{\beta} d\beta + \pdv{\ln Z}{x} dx 
\end{align*}
or 
\begin{align*}
    d \ln Z &= -\bar E d\beta + \beta \dbar W \\
    &= (\beta \dbar W - d(\bar E \beta)) + \beta d \bar E \\
    d(\ln Z + \bar E \beta) &= \beta(\dbar W + d\bar E) = \beta dQ \\
    &= \frac{1}{k} dS
\end{align*}
Thus the entropy of an ideal gas system is
\begin{align*}
    S = k(\ln Z + \bar E \beta)
\end{align*}
so using $\bar E = \frac{3}{2} N kT$ we get
\begin{align*}
    S = Nk \qt[
        \frac{3}{2} \ln(\frac{2m}{\hbar^2\pi}) - \frac{3}{2} \ln(\beta) + \ln(V) + \frac{3}{2}
    ]
\end{align*}
where
\begin{align*}
    \frac{3}{2} \ln \beta = \frac{3}{2} \ln k + \frac{3}{2} \ln T
\end{align*}
so
\begin{align*}
    S = Nk \qt(
        \ln V + \frac{3}{2} \ln T + \sigma_0
    ), \quad \sigma_0 = \frac{3}{2} \ln(\frac{2m}{\hbar^2\pi}) + \frac{5}{2}
\end{align*}
\paragraph{Two Issues:}
\begin{itemize}
    \item The second law is violated: $T \to 0$ which implies $S \to -\infty$
    \item Gibbs paradox: $S > S' + S''$
\end{itemize}
\end{document}