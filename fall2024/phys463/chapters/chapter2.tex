\documentclass[../main.tex]{subfiles}

\graphicspath{{../images/}}

\begin{document}
\pagestyle{fancy}
\lhead{Lecture 2: 8/29/24}
\chead{Chapter 2}
\rhead{PHYS 463}

\section{Statistical description of systems of particles}
\barh \vspace*{1em}

\paragraph*{Essential ingredients:}
\begin{enumerate}
    \item state of the system:
    \begin{itemize}
        \item single spin-1/2 particle. $\uparrow, \downarrow$
        \item a bunch of spin-1/2 particles. $\uparrow\uparrow\downarrow\dots$
        \item a simple 1D Harmonic Oscillator: $E = (n + 1/2) \hbar \omega$, with states $\ket{n}$
        \item a bunch of 1D HO: $\ket{n_1, n_2, \dots, n_N}$
    \end{itemize}
    \item Statistical ensemble: Instead of a simple experiments, we consider an exsemble of many exps.
    \item Basic postulate about a priori probabilities (relative prob of finding the system in any of its accessible states)
    \item Calculate probabilities
\end{enumerate}

\paragraph*{Example:} 3 spin-1/2
\begin{table} [ht]
    \centering
    \begin{tabular}{c|c|c|c|c}
        State & Spin & Energy & $\Omega(E)$ & $y_k = \uparrow, \downarrow$ \\
        \hline
        $\uparrow\uparrow\uparrow$ & 3/2 & $-3\mu H$ & 1 & $\Omega(-\mu H, \uparrow)$ \\
        $\uparrow\uparrow\downarrow$ & 1/2 & $-\mu H$ & 3 \\
        $\uparrow\downarrow\uparrow$ & & & \\
        $\downarrow\uparrow\uparrow$ & & & \\
        $\uparrow\downarrow\downarrow$ & -1/2 & $\mu H$ & 3 \\
        $\downarrow\uparrow\downarrow$ & & & \\
        $\downarrow\downarrow\uparrow$ & & & \\
        $\downarrow\downarrow\downarrow$ & -3/2 & $3\mu H$ & 1
    \end{tabular}
    \caption{Energy levels of 3 spin-1/2 particles}
    \label{tab:3spin}
\end{table}

System: \emph{isolated}: energy cannot change
\emph{equilibrium}: prob of finding the system in any one accessible state is constant in time

\paragraph*{A fundamental postulate:}
\[\boxed{\textrm{An isolated system in equilirbium is equally likely to be in any of its accessible states}}\]
In calculating probabilities, e.g., isolated system with energy in range $[E, E+\delta E]$

$\Omega(E)$: total number of states of the system in this range

$\Omega(E,y_k)$: in this energy range and some other property $y_k$
where the probability of having this property is
\[P(y_k) = \frac{\Omega(E,y_k)}{\Omega(E)}\]

\paragraph*{Density of states (DOS)}
\[\Omega(E) = w(E) \dd E, \quad w(E) \sim E\]
where $w(E)$ is the density of states.
\end{document}
