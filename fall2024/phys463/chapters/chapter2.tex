\documentclass[../main.tex]{subfiles}

\graphicspath{{../images/}}

\begin{document}
\pagestyle{fancy}
\lhead{Lecture 2: 8/29/24}
\chead{Chapter 2}
\rhead{PHYS 463}

\section{Statistical description of systems of particles}
\barh \vspace*{1em}

\subsection{Statistical formulation}
\paragraph*{Essential ingredients:}
\begin{enumerate}
    \item state of the system:
    \begin{itemize}
        \item single spin-1/2 particle. $\uparrow, \downarrow$
        \item a bunch of spin-1/2 particles. $\uparrow\uparrow\downarrow\dots$
        \item a simple 1D Harmonic Oscillator: $E = (n + 1/2) \hbar \omega$, with states $\ket{n}$
        \item a bunch of 1D HO: $\ket{n_1, n_2, \dots, n_N}$
    \end{itemize}
    \item Statistical ensemble: Instead of a simple experiments, we consider an exsemble of many exps.
    \item Basic postulate about a priori probabilities (relative prob of finding the system in any of its accessible states)
    \item Calculate probabilities
\end{enumerate}

\paragraph*{Example:} 3 spin-1/2
\begin{table} [ht]
    \begin{center}
    \begin{tabular}{c|c|c|c|c}
        State & Spin & Energy & $\Omega(E)$ & $y_k = \uparrow, \downarrow$ \\
        \hline
        $\uparrow\uparrow\uparrow$ & 3/2 & $-3\mu H$ & 1 & $\Omega(-\mu H, \uparrow)$ \\
        $\uparrow\uparrow\downarrow$ & 1/2 & $-\mu H$ & 3 \\
        $\uparrow\downarrow\uparrow$ & & & \\
        $\downarrow\uparrow\uparrow$ & & & \\
        $\uparrow\downarrow\downarrow$ & -1/2 & $\mu H$ & 3 \\
        $\downarrow\uparrow\downarrow$ & & & \\
        $\downarrow\downarrow\uparrow$ & & & \\
        $\downarrow\downarrow\downarrow$ & -3/2 & $3\mu H$ & 1
    \end{tabular}
    \end{center}
    \caption{Energy levels of 3 spin-1/2 particles}
\end{table}

System: \emph{isolated}: energy cannot change
\emph{equilibrium}: prob of finding the system in any one accessible state is constant in time

\paragraph*{A fundamental postulate:}
\[\boxed{\textrm{An isolated system in equilirbium is equally likely to be in any of its accessible states}}\]
In calculating probabilities, e.g., isolated system with energy in range $[E, E+\delta E]$

$\Omega(E)$: total number of states of the system in this range

$\Omega(E,y_k)$: in this energy range and some other property $y_k$
where the probability of having this property is
\[P(y_k) = \frac{\Omega(E,y_k)}{\Omega(E)}\]

\paragraph*{Density of states (DOS)}
\[\Omega(E) = w(E) \dd E, \quad w(E) \sim E\]
where $w(E)$ is the density of states.

\newpage
\lhead{Lecture 3: 9/3/24}
\chead{Chapter 2.6-2.8}
\subsection{Interactions between macroscopic systems} 

\paragraph*{} In general: specify some macroscopic measureable paremeters $x_1, x_2, \dots x_n$

\begin{itemize}
    \item Micostate: A particula quantum state: $\gamma$ of the system with energy $E_r$
    \[E_r = E_r(x_1, x_2,\dots, x_n)\]
    \item Macrostate (Macroscopic state): Specify external parameters and any other conditions, and includes all the possible microstates—e.g., from Table above the macrostate of $-\mu H$ has 3 microstates.
    ``Microstate'' is one particular specific state consistern with the macrostate.
\end{itemize}

\paragraph{} Consider two macro systems $A, A'$; they can interact with each other to exchange energy.

Q: what are the different ways to exchange $E$? HEAT, WORK.
e.g. If $A, A'$ are in a box seperated by a wall, then the wall moving due to pressure exchanges energy as work. If the wall cannot move, then there is no work exchanged.
[insert image of two boxes with a wall]
\paragraph*{Two Cases:}
\begin{itemize}
    \item  \emph{thermal interaction}: If all the external parameters are fixed
    \[\Delta E = Q, \quad \Delta E' = Q'\]
    where $Q, Q'$ are the heat absorbed by each macrosystem, and the energy of the whole system is unchanged, i.e.,
    \[\Delta E + \Delta E' = 0 \implies Q + Q' = 0, \quad Q = -Q'\]
    \item \emph{mechanical interaction} (thermal isolation): no heat exchange ``adiabatic''. I do work, negative work is done!
\end{itemize}
\paragraph*{Example:} Beaker of water, $A$, and a wheel attached to a pulley with a weight, $A'$ (2.7 Example 2). The work done by the pulley decreases the energy of system $A'$ by $ws$ (weight times distance).

\paragraph*{In general} energy can be exchanged both as Heat and Work.
\[Q \equiv \Delta E - \mathcal{W}\]
where $\mathcal{W}$ is the work done to the system. And
\[W = \mathcal{W}\]
is the workd done by the system, i.e.,
\[Q \equiv \Delta \bar E + W\]

\paragraph*{Case of small amounts interaction:} Infinitesimal changes
\[\dbar Q = d \bar E + \dbar W \]
where the bar through the differential indicates the process as path dependent.

\paragraph*{Worksheet}
\begin{enumerate}
    \item [(1)] For the infinitesimal quantity
    \[dG = \alpha dx + \beta \frac{x}{y} dy\]
    it is path dependent:
    \begin{itemize}
        \item $(1,1) \to (1,2) \to (2,2)$:
        \begin{align*}
            G &= \int_{(1,1)}^{(1,2)} dG + \int_{(1,2)}^{(2,2)} dG \\
            &= \cancel{\alpha x\eval_{(1,1)}^{(1,2)}} + \beta \ln y\eval_{(1,1)}^{(1,2)} + \alpha x\eval_{(1,2)}^{(2,2)} + \cancel{\beta \ln y\eval_{(1,2)}^{(2,2)}} \\
            &= \alpha + \beta \ln 2
        \end{align*}
        \item $(1,1) \to (2,1) \to (2,2)$:
        \begin{align*}
            G &= \alpha + 2\beta \ln 2
        \end{align*}
        So the path is dependent or $dG$ is an inexact differential.
    \end{itemize}
    \item [(2)] Is the following exact?
    \[dF = \frac{a}{x} dx + \frac{b}{y} dy \]
\end{enumerate}

\newpage
\lhead{Lecture 4: 9/5/24}
\chead{Chapter 2.9-2.11}
\paragraph*{General interaction process:} energy is exchanged both as heat and work
\[Q = \delta E + W\]
where $Q$ is the heat added to the system (positive $\Delta E$ adds energy)
and $W$ is the work done by the system

\paragraph*{Very very small work/heat:} infinitesimal
\[\dbar Q = d\bar E + \dbar W\]
where $d$ is an exact differential (path independent) and $\dbar$ is a inexact differential (path dependent).

\paragraph*{Math:} multivariable differential

A differential form is exact if its equal to the general differential $dF$ for some function $F(x,y)$

e.g. \(A(x,y) dx + B(x,y) dy = dF(x,y)\)

From last times worksheet:
\[\frac{a}{x} dx + \frac{b}{y} dy = d (a \ln x + b \ln y)\]

\paragraph*{How to check if its exact?} Assume $F$ exists:
\[dF(x,y): \qqtext{is exact} \iff \quad \qt(\pdv{A}{y}_x) = \qt(\pdv{B}{x})_y\]
where $\iff$ means iff or if and only if. e.g. from the worksheet:
\[dG = a dx + b \frac{x}{y} dy, \quad A = a, \quad B = b\frac{x}{y} \]
so
\[\pdv{A}{y} = 0, \quad \pdv{B}{x} = \frac{b}{y}\]
thus it is inexact.

\paragraph{Quasi-static process:} A system interacts with other systems in a process that is so slow that $A$ remains arbitrarily close to equilibrium at all stages!

e.g. a piston pushing very slowly in a cylinder; when the system is not in equilibrium, then the ideal gas law $pV = nRT$ does not hold.

``relaxational time $\tau$'': time system requires to reach equilibrium if it experiences a sudden change.

Recall we denote the external parameters of an isolated system
\[x_1, x_2, \dots, x_x \]
and the energy of a microstate $r$
\[E_r = E_r(x_1, x_2, \dots, x_n )\]
When we start to change the external parameter, energy of state $r$ will change:
\[x_\alpha \to x_\alpha + d x_\alpha \]
and the change in energy is
\[dE_r = \sum_{\alpha = 1}^n \pdv{E_r}{x_\alpha} dx_\alpha\]
Now in isolated case $\dbar Q = 0$ so
\begin{align*}
    dE_r &+ \dbar W_r = 0 \\
    \implies \dbar W_r &= -dE_r = \sum_{\alpha = 1}^n \qt(-\pdv{E_r}{x_\alpha}) dx_\alpha
\end{align*}
where
\[X_{\alpha, r} = -\pdv{E_r}{x_\alpha} \]
is the ``generalized force''--- e.g. if $x$ is a distance, then $X$ is a force; if $x$ is a volume, then $X$ is a pressure.

NOTE all discussion above are for : state $r$

Consider an ensemble: in a quasi static process, $X_{\alpha, r}$ has definite value, so
\[\dbar W = \sum_\alpha \bar X_{\alpha, r}dx_\alpha\]
where $\bar X_{\alpha, r}$ is mean of the generalized force.

\paragraph*{Example:} Cylindrical chamber in state $r$ (height $s$, circular area $A$, pressure $P_r$) with a piston pushing in $ds$

Force on the piston: $P_r A$

Volume: $V = A S$

Thus work done is 
\begin{align*}
    \dbar W = F d s &= (P_r A) d s\\
    &= P_r dV
\end{align*}
and 
\begin{align*}
    dE_r = -\dbar W_r = -P_r dV, \quad 
    P_r = -\pdv{E_r}{V}
\end{align*}

\paragraph*{Worksheet}
\begin{enumerate}
    \item The mean pressure $p$ of thermally insulated gas vaires with volume $V$ by
    \[pV^\gamma = K \]
    where $K$ and $\gamma$ are constants. Find work from $p_i, V_i$ to $p_f, V_f$.
    \begin{align*}
        \int \dbar W &= \int_{V_i}^{V_f} p dV \\
        &= \int_{V_i}^{V_f} \frac{K}{V^\gamma} dV \\
        W &= \frac{KV^{1 - \gamma}}{1 - \gamma} \eval_{V_i}^{V_f}
    \end{align*}
    And since $p_i V_i^\gamma = p_f V_f^\gamma = K$, then
    \begin{align*}
        W &= \frac{K}{1 - \gamma} \qt(V_f^{1 - \gamma} - V_i^{1 - \gamma}) \\
        &= \frac{1}{1 - \gamma} \qt(p_f V_f - p_i V_i)
    \end{align*}
\end{enumerate}
\end{document}
