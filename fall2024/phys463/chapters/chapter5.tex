\documentclass[../main.tex]{subfiles}

\graphicspath{{../images/}}

\begin{document}
\pagestyle{fancy}
\lhead{Lecture 10: 9/26/24}
\chead{Chapter 5.5-6}
\rhead{PHYS 463}

\section{Simple Applications of macroscopic thermodynamics}

\subsection{General relationship of thermodynamics}

Fundamental thermodynamic relation for a \textit{quasi-static process}:
\begin{align*}
    dS = \frac{\dbar Q}{T}
\end{align*}
where
\begin{align*}
    \dbar Q = dE + \dbar W = dE + pdV
\end{align*}
The only external parameter of change is $V$
\begin{align*}
    \implies dE = TdS - pdV
\end{align*}
This specifies certain relationship between $T,S,p,V$ i.e. $S$ \& $V$ are independent variables
\begin{align*}
    E = E(S,V)
\end{align*}
So we have a pure mathematical relationship
\begin{align*}
    dE = \qt(\pdv{E}{S})_V dS + \qt(\pdv{E}{V})_S dV
\end{align*}
where
\begin{align*}
    \begin{cases}
        T = \qt(\pdv{E}{S})_V \\
        -p = \qt(\pdv{E}{V})_S
    \end{cases}
\end{align*}
which we already know! Because $dE$ is an exact differential
\begin{align*}
    \qt(\pdv{T}{V})_S = -\qt(\pdv{p}{S})_V
\end{align*}
this is known as the first Maxwell relation \href{https://en.wikipedia.org/wiki/Maxwell_relations#/media/File:Thermodynamic_map.svg}{(wiki)}.

\paragraph{How about} $S, P$?

From our favorite starting point
\begin{align*}
    dE = TdS - pdV 
\end{align*}
we need to change $dV \to dp$ so from chain rule
\begin{align*}
    d(pV) = p dV + V dp \implies pdV = d(pV) - V dp 
\end{align*}
so
\begin{align*}
    dE = TdS - d(pV) + V dp
\end{align*}
or
\begin{align*}
    d(E + pV) = TdS + Vdp
\end{align*}
lets call this new parameter $H = E + pV$ the \textbf{enthalpy} i.e.
\begin{align*}
    H = H(S,p)
\end{align*}
So
\begin{align*}
    \begin{cases}
        T = \qt(\pdv{H}{S})_p \\
        V = \qt(\pdv{H}{p})_S
    \end{cases}
\end{align*}
where $dH$ is an exact differential
\begin{align*}
    \qt(\pdv{T}{p})_S = \qt(\pdv{V}{S})_p
\end{align*}
or the second Maxwell relation!

\paragraph{Worksheet}
We can derive the Helmholtz free energy $F = F(T,V)$ by starting with
\begin{align*}
    d(TS) = TdS + SdT \implies TdS = d(TS) - SdT
\end{align*}
so
\begin{align*}
    dE &= d(TS) -SdT - pdV \\
    d(E - TS) &= -SdT - pdV
\end{align*}
\begin{enumerate}
    \item Thus the Hemholtz free energy $F \equiv E - TS$ so
    \begin{align*}
        dF &= dE - (TdS + SdT) \\
        &= TdS - pdV - Tds - SdT \\
        &= -SdT - pdV
    \end{align*}
    \item So $F = F(T,V)$ the we know that
    \begin{align*}
        \begin{cases}
            -S = \qt(\pdv{F}{T})_V \\
            -p = \qt(\pdv{F}{V})_T
        \end{cases}
    \end{align*}
    and $dF$ is an exact differential
    \begin{align*}
        \qt(\pdv{S}{V})_T = \qt(\pdv{p}{T})_V
    \end{align*}
\end{enumerate}
Finally for independent parameters $T,p$:
\begin{align*}
    dE = TdS - pdV 
\end{align*}
we need to change $dV \to dp$ so from chain rule
\begin{align*}
    d(pV) = p dV + V dp \implies pdV = d(pV) - V dp 
\end{align*}
so also using $TdS = d(TS) - S dT$
\begin{align*}
    dE &= (d(TS) - S dT) - (d(pV) - V dp) \\
    d(E - TS + pV) &= -S dT + V dP
\end{align*}
where $G = E - TS + pV$ is the Gibbs free energy $G = G(T,p)$
\begin{align*}
    \begin{cases}
        -S = \qt(\pdv{G}{T})_p \\
        V = \qt(\pdv{G}{p})_T
    \end{cases}
\end{align*}
and $dG$ is an exact differential
\begin{align*}
    \qt(\pdv{V}{T})_p = -\qt(\pdv{S}{p})_T
\end{align*}
\paragraph{Summary of Maxwell relations}
\begin{align*}
    \qt(\pdv{T}{V})_S = -\qt(\pdv{p}{S})_V \\
    \qt(\pdv{T}{p})_S = \qt(\pdv{V}{S})_p \\
    \qt(\pdv{S}{V})_T = \qt(\pdv{p}{T})_V \\
    \qt(\pdv{V}{T})_p = -\qt(\pdv{S}{p})_T
\end{align*}
or in box form
\table
\end{document}

