\documentclass[../main.tex]{subfiles}

\graphicspath{{../images/}}

\begin{document}
\pagestyle{fancy}

\lhead{Lecture 16: 10/24/24}
\chead{Chapter 7.3,7.5-7.7}
\rhead{PHYS 463}

\section{Simple Applications of Stat Mech}
\subsection{Gibbs Paradox}
From the last lecture the Gibbs paradox $S > S' + S''$ is puzzling\dots

\paragraph{(indistinguishable)} If the particles are identical we can keep track double counting with
\begin{align*}
    Z_N = \frac{Z_1^N}{N!}
\end{align*}
And from the log of the partition function
\begin{align*}
    \ln Z_N &= N\ln Z_1 - \ln N! \qusing \ln N! = N \ln N - N \\
    &= N \ln Z_1 - N\ln N + N
\end{align*}
NOTE: This does not affect $\bar E, \bar P$ as they are still
\begin{align*}
    \bar E = \frac{3}{2} N kT, \quad \bar P = \frac{NkT}{V}
\end{align*}
The entropy is recaculated as
\begin{align*}
    S = k (\ln Z + \beta E)
\end{align*}
Using
\begin{align*}
    Z_1 = \qt(\frac{2m}{\hbar^2 \pi})^{3/2} \beta^{-3/2} V
\end{align*}
we have the entropy
\begin{align*}
    S = kN \qt[
        \ln\frac{V}{N} + \frac{3}{2} \ln T + \sigma_0
    ], \quad \sigma_0 = \sigma + 1 = \frac{3}{2}\ln\qt(\frac{2\pi mk}{h^2})^{3/2} + \frac{5}{2}
\end{align*}

\subsection{Equipartition Theorem} Using the Boltzmann function

Consider some systems described by generalized coordinates $q_k, p_k$ with energies
\begin{align*}
    E = E(q_1, \dots, q_N, p_1, \dots, p_N)
\end{align*}
\begin{itemize}
    \item Assumption 1: The total energy is additive
    \begin{align*}
        E = \epsilon_i(p_i) + E(q_1, \dots, q_N, p_1, \dots, \text{no } p_i, \dots, p_N)
    \end{align*}
    \item Assumption 2: function $\epsilon_i$ is quasi-staic in $p_i$ or usually the energy is quadratic i.e.
    \begin{align*}
        \epsilon_i(p_i) = bp_i^2
    \end{align*}
\end{itemize}
The average value of $\epsilon_i$ is
\begin{align*}
    \overline{\epsilon_i} &= \frac{1}{Z} \int \epsilon_i e^{-\beta E} dq dp \\
    &= \frac{\int_{-\infty}^\infty e^{-\beta E(q_1,\dots, p_N)} \epsilon_i dq_1, \dots, dp_N}{\int_{-\infty}^\infty e^{-\beta E(q_1,\dots, p_N)} dq_1, \dots, dp_N}
\end{align*}
From the first assumption we know that the energy is additive so
\begin{align*}
    \overline{\epsilon_i} &= \frac{\int_{-\infty}^\infty e^{-\beta \epsilon_i} \epsilon_i dp_i \cancel{\int e^{-\beta E'} dq_1, \dots, dp_N}}{\int_{-\infty}^\infty e^{-\beta \epsilon_i} dp_i \cancel{\int e^{-\beta E'} dq_1, \dots, dp_N}} \\
    &= -\pdv{\beta} \ln(\int e^{-\beta E} dp_i)
\end{align*}
Now using the second assumption the intgral becomes
\begin{align*}
    \int e^{-\beta \epsilon_i} dp_i = \int e^{-\beta bp_i^2} dp_i
\end{align*}
With a change of variables
\begin{align*}
    y = \sqrt{\beta} p_i, \quad dy = \sqrt{\beta} dp_i
\end{align*}
the integral becomes
\begin{align*}
    = \frac{1}{\sqrt{\beta}} \int_{-\infty}^\infty e^{-y^2} dy = \sqrt{\frac{\pi}{\beta}}
\end{align*}
which is independent of $\beta$ so
\begin{align*}
    \int e^{-\beta \epsilon_i} dp_i = C \beta^{-1/2} 
\end{align*}
where $C$ is a constant. Thus
\begin{align*}
    \overline{\epsilon_i} = -\pdv{\beta} \ln(C \beta^{-1/2}) = \frac{1}{2\beta} = \frac{1}{2} kT
\end{align*}

\paragraph{Worksheet}
\begin{itemize}
    \item [1.] Use the equipartition theorem to determine the molar heat capacity at constant volume of a monoatomic gas: Given
    \begin{align*}
        \bar \epsilon = \frac{1}{2} kT \qqtext{for} q_x, q_y, q_z \implies \bar E = \frac{3}{2} N kT
    \end{align*}
    so the molar heat capacity is
    \begin{align*}
        c_V = \pdv{\bar E}{T} = \frac{3}{2} N k \implies c_p = \frac{c_V}{n} = \frac{3}{2} R ,\quad R = \frac{N}{n} k = N_A k
    \end{align*}
    \item [2.] A small particle undergoing Brownian motion in a liquid. The particle is in equilibrium with a bath at temp T.
    Use the equipartition theorem to determine the velocity dispersion
    \begin{align*}
        \bar E_x &= \frac{1}{2} m \overline{v_x}^2 = \frac{1}{2} kT \\
        \implies \overline{v_x}^2 &= \frac{2\bar E_x}{m} = \frac{kT}{m}
    \end{align*}
\end{itemize}

\subsection{Specifc heat of solids}
In 3D the energy is
\begin{align*}
    E = \sum_{i=1}^{3N} \qt[
        \frac{p_i^2}{2m} + \frac{1}{2} m k_i^2 q_i^2
    ]
\end{align*}
where we have three dimensions as well as a kinetic and potential dimension (6N degrees of freedom).
From the equipartition theorem the average energy is
\begin{align*}
    \bar E = 3N \qt(\frac{1}{2} kT \cdot 2) = 3N kT
\end{align*}
The molar heat capacity is roughly
\begin{align*}
    c_p = \frac{c_V}{n} = \frac{3Nk}{n} = 3R
\end{align*}
The molar heat capacity of a solids at $T = 300$ K are
% list of values for c_p of Ag, S, Zn, Al, C
\begin{align*}
    c_p = \begin{cases}
        25.35 \text{ J/mol K} & \text{Ag} \\
        22.75 \text{ J/mol K} & \text{S} \\
        25.39 \text{ J/mol K} & \text{Zn} \\
        24.20 \text{ J/mol K} & \text{Al} \\
        6.01 \text{ J/mol K} & \text{C}
    \end{cases}
\end{align*}

\paragraph{Einstein's Solids:} All atoms have the same spring constant $\omega = \sqrt{k/m}$
From the partition function, the average energy in 3D is
\begin{align*}
    \bar E = 3N \hbar \omega \qt(
        \frac{1}{2} + \frac{1}{e^{\beta \hbar \omega} - 1}
    )
\end{align*}

\end{document}