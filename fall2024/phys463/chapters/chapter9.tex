\documentclass[../main.tex]{subfiles}

\graphicspath{{../images/}}

\begin{document}
\pagestyle{fancy}

\lhead{Lecture 21: 11/12/24}
\chead{Chapter 9.1-9.2}
\rhead{PHYS 463}

\section{Quantum Statistics}

\subsection{Identical particles and symmetry}
From Gibbs' paradox
\begin{align*}
    Z = = \frac{Z_1^N}{N!}
\end{align*}
where we have indistinguishable particles.

\begin{itemize}
    \item Classical particles: $A, B, C, \dots$ where we have distinguishable particles
    \item Quantum particles: $A, A, A, \dots$ which are indistinguishable\dots but we also have two types of quantum particles
    \begin{itemize}
        \item Bosons: Integer spin, symmetric total wave function $\Psi$
        e.g. photons, gamma rays
        \item Fermions: Half-integer spin, antisymmetric total wave function $\Psi \to$ pauli exclusion principle, 
        e.g. electrons
    \end{itemize}
\end{itemize}

\paragraph{Worksheet}
\begin{itemize}
    \item [1.] Assume 2 partilces and each particle can be in one of three possible staes,
    \begin{align*}
        r = 1, 2, 3
    \end{align*}
    \begin{itemize}
        \item [(1)] Maxwell-Boltzmann statistics (classical particle) total number of available states
        \begin{align*}
            \Omega = 3^2 = 9
        \end{align*}
        \item [(2)] Bose-Einstein statistics (bosons) total number of available states
        \begin{align*}
            \Omega = 3 + 3 = 6 
        \end{align*}
        \item [(3)] Fermi-Dirac statistics (fermions) we take away the same states occupations
        \begin{align*}
            \Omega = 6 - 3 = 3
        \end{align*}
    \end{itemize}
\end{itemize}

\subsection{Formulation of quantum statistical problem}
Consider a gas of particles in volume $V$ at temperature $T$.
\begin{itemize}
    \item $\epsilon_r$: is the energy of a particle in state $r$
    \item $n_r$: \# of particles in state $r$
    \item $R$: specify all possible states of the whole system
\end{itemize}
So the total energy of the system is
\begin{align*}
    E_R = n_1 \epsilon_1 + n_2 \epsilon_2 + \dots = \sum_r n_r \epsilon_r
\end{align*}
where $\sum_r n_r = N$. The partition function is
\begin{align*}
    Z = \sum_R e^{-\beta E_R} = \sum_R e^{-\beta \sum_r n_r \epsilon_r}
\end{align*}
Since the probability of having $\{n_1, n_2, \dots, n_r, \dots\}$ state is
\begin{align*}
    \frac{e^{-\beta E_R}}{Z}
\end{align*}
for a state $R$, the mean number of particles in states $S$ is
\begin{align*}
    \bar n_S = \frac{\sum_R n_S e^{-\beta E_R}}{Z}
\end{align*}
or
\begin{align*}
    = \frac{1}{Z} \sum_R \qt(-\frac{1}{\beta} \pdv{Z}{\epsilon_S})
\end{align*}

\newpage
\lhead{Lecture 22: 11/14/24}
\chead{Chapter 9.3}
\begin{itemize}
    \item Bose-Einstein Statistics (BE)
    \begin{align*}
        \sum n_R = N
    \end{align*}
    \item Photon statistics: no restriction of particle number
    \item Fermi-Dirac Statistics (FD): for $n_r = 0, 1$
\end{itemize}
Using the mulitplication math thing
\begin{align*}
    e^{-\beta (n_1 \epsilon_1 + n_2 \epsilon_2 + \dots)} = e^{-\beta n_s \epsilon_s} e^{-\beta n_1 \epsilon_1 + \dots}
\end{align*}
where the second term doesn't have a $n_s$ term. So the mean number of particles in state $S$ is
\begin{align*}
    \bar n_S &= \frac{1}{Z} \sum_R n_S e^{-\beta E_R} \\ 
    &= \frac{1}{Z} \sum_R n_S e^{-\beta n_s \epsilon_s} e^{-\beta n_1 \epsilon_1 + \dots} \\
    &= \frac{
        \sum_{R} \qt(
            n_s e^{-\beta n_s \epsilon_s} e^{-\beta n_1 \epsilon_1 + \dots}
        )
    }
    {
        \sum_R \qt(
            e^{-\beta n_s \epsilon_s} e^{-\beta n_1 \epsilon_1 + \dots}
        )
    } \\
    &= \frac{
        \sum_{n_s} \qt(
            n_s e^{-\beta n_s \epsilon_s} \sum_{n_1, n_2, \dots}^{(S)} e^{-\beta n_1 \epsilon_1 + \dots}
        )
    }
    {
        \sum_{n_s} \qt(
            e^{-\beta n_s \epsilon_s} \sum_{n_1, n_2, \dots}^{(S)} e^{-\beta n_1 \epsilon_1 + \dots}
        )
    }
\end{align*}
\paragraph{Photon statistics} No restriction on \# of particles $\implies$ the sum $\sum_{n_1, n_2, \dots}$ is always infinite,
so the second term cancels out
\begin{align*}
    \bar n_S = \frac{
        \sum_{n_s} n_s e^{-\beta n_s \epsilon_s}
    }
    {
        \sum_{n_s} e^{-\beta n_s \epsilon_s}
    } -\frac{1}{\beta} \pdv{\epsilon_s} \ln(\sum_{n_s = 0}^\infty e^{-\beta n_s \epsilon_s})
\end{align*}
and using the geometric series
\begin{align*}
    &= -\frac{1}{\beta} \pdv{\epsilon_s} \ln \frac{1}{1 - e^{-\beta \epsilon_s}} \\
    &= \frac{1}{\beta} \pdv{\epsilon_s} \ln(1 - e^{-\beta \epsilon_s}) \\
    &= \frac{1}{\beta} \frac{\beta e^{-\beta \epsilon_s}}{1 - e^{-\beta \epsilon_s}} \\
    &= \frac{1}{e^{\beta \epsilon_s} - 1}
\end{align*}

\paragraph{Fermi-Dirac statistics} For $n_r = 0, 1$ (the easier one)
\begin{align*}
    \bar n_S &= \frac{
        0 + e^{-\beta \epsilon_s} Z_S (N - 1)
    }
    {
        Z_S (N) + e^{-\beta \epsilon_S} Z_S (N - 1)
    } \\
    &= \frac{1}{
        \qt(
            \frac{Z_S (N)}{Z_S (N - 1)} e^{-\beta \epsilon_s} + 1
        )
    }
\end{align*}
where the $Z_S$ ommits the $n_s$ term
\begin{align*}
    Z_S (N) \equiv \sum_{n_1, n_2, \dots}^{(S)} e^{-\beta n_1 \epsilon_1 + \dots}
\end{align*}
To relate $Z_S (N)$ and $Z_S (N - 1)$ for large $N$:
\begin{align*}
    \ln Z_S (N-1) = \ln Z_S (N) - \pdv{\ln Z_S (N)}{N} \cdot 1
\end{align*}
where the $\pdv{\ln Z_S (N)}{N} = \alpha_S$ so
\begin{gather*}
    Z_S (N - 1) = Z_S (N) e^{-\alpha_S} \\
    \implies \frac{Z_S (N)}{Z_S (N - 1)} = e^{\alpha_S}
\end{gather*}
ASSUMPTION: Since the sum of $Z_S$ includes \emph{many} states,
$\alpha_S$ cannot does not depend too much on $S$, so we assume a constant
\begin{align*}
    \alpha_S = \alpha = \pdv{\ln Z}{N}
\end{align*}
So the mean number of particles in state $S$ is
\begin{align*}
    \bar n_S = \frac{1}{e^{\alpha + \beta \epsilon_s} + 1}
\end{align*}
Since we know the relation
\begin{align*}
    F = -kT \ln Z \implies \pdv{F}{N} = -kT \pdv{\ln Z}{N} = \mu \implies \alpha = -\beta \mu
\end{align*}
So we get the Fermi-Dirac distribution
\begin{align*}
    \bar n_S = \frac{1}{e^{\beta (\epsilon_s - \mu)} + 1}
\end{align*}

\paragraph{Worksheet} Bose-Einstein stats using
\begin{align*}
    \frac{Z_S(N)}{Z_S(N-1)} = e^{\alpha}
\end{align*}
The average number of particles in state $S$ is
\begin{align*}
    \bar n_S &= \frac{
        0 + e^{-\beta \epsilon_s} Z_S (N - 1) + 2 e^{-\beta \epsilon_s} Z_S (N - 2) + \dots
    }
    {
        Z_S (N) + e^{-\beta \epsilon_s} Z_S (N - 1) + e^{-2\beta \epsilon_s} Z_S (N - 2) + \dots
    }
\end{align*}
taking out a $Z_S (N)$ term for each e.g.
\begin{align*}
    e^{-\beta \epsilon_s} Z_S (N - 1) = Z_S(N) \qt(e^{-\beta \epsilon_s} \frac{Z_S(N - 1)}{Z_S(N)}) = Z_S(N) e^{-\beta \epsilon_s}e^{-\alpha}
\end{align*}
and for the next term
\begin{align*}
    e^{-2\beta \epsilon_s} Z_S (N - 2) &= Z_S(N) e^{-2\beta \epsilon_s} \frac{Z_S(N - 2)}{Z_S(N)} \\
    &= Z_S(N) e^{-2\beta \epsilon_s} \frac{Z(N - 2)}{Z(N - 1)} e^{-\alpha} \\
    &= Z_S(N) e^{-2\beta \epsilon_s} e^{-2\alpha}
\end{align*}
So
\begin{align*}
    \bar n_S = \frac{
        Z_S(N) \qt(
            0 + e^{-\beta \epsilon_s} e^{-\alpha} + e^{-2\beta \epsilon_s} e^{-2\alpha} + \dots
        )
    }
    {
        Z_S(N) \qt(
            1 + e^{-\beta \epsilon_s} e^{-\alpha} + e^{-2\beta \epsilon_s} e^{-2\alpha} + \dots
        )
    }
\end{align*}

\newpage
\lhead{Lecture 23: 11/19/24}
\chead{Chapter 9.4-9.7}
From last time
\begin{itemize}
    \item Photon Statistics (Boson):
    \begin{align*}
        \bar n_s = \frac{1}{e^{\beta \epsilon_s} - 1}
    \end{align*}
    \item Bose-Einstein Statistics:
    \begin{align*}
        \bar n_s = \frac{1}{e^{\beta (\epsilon_s - \mu)} - 1}
    \end{align*}
    \item Fermi-Dirac Statistics: 
    \begin{align*}
        \bar n_s = \frac{1}{e^{\beta (\epsilon_s - \mu)} + 1}
    \end{align*}
\end{itemize}
Today: Partition function for quantum statistics\dots
\begin{align*}
    Z = \sum_R e^{-\beta (n_1 \epsilon_1 + n_2 \epsilon_2 + \dots)}
\end{align*}
where for BE and FD, $\sum n_r = N$

\subsection{Photon statistics}
\begin{align*}
    Z &= \sum_{n_1, n_2, \dots} e^{-\beta n_1 \epsilon_1 + \dots} \\
    &= \underbrace{\sum_{n_1 = 0}^\infty e^{-\beta n_1 \epsilon_1}}_{1 + e^{-\beta \epsilon_1} + e^{-2\beta \epsilon_1} + \dots} \sum_{n_2 = 0}^\infty e^{-\beta n_2 \epsilon_2} \dots \\
    &= \frac{1}{1 - e^{-\beta \epsilon_1}} \frac{1}{1 - e^{-\beta \epsilon_2}} \dots \\
\end{align*}
So the log of the partition function is
\begin{align*}
    \ln Z &= \sum_r \ln \frac{1}{1 - e^{-\beta \epsilon_r}} \\
    &= -\sum_r \ln (1 - e^{-\beta \epsilon_r})
\end{align*}
The mean number of particles in one state $\epsilon_S$ is
\begin{align*}
    \bar n_S &= -\frac{1}{\beta} \pdv{\ln Z}{\epsilon_S} \\
    &= \frac{1}{\beta} \frac{- (-\beta) e^{-\beta \epsilon_S}}{1 - e^{-\beta \epsilon_S}} \\ 
    &= \frac{e^{-\beta \epsilon_S}}{1 - e^{-\beta \epsilon_S}} \\
    & = \frac{1}{e^{\beta \epsilon_S} - 1}
\end{align*}

\subsection{Bose-Einstein statistics}
The partition function for BE:
\begin{align*}
    Z = \sum_R = e^{-\beta (n_1 \epsilon_1 + n_2 \epsilon_2 + \dots)}
\end{align*}
where $\sum_r n_r = N$ so $Z(N')$ has a rapidly increasing with $N'$ which is a variable.

$Z(N') e^{-\alpha N'}$ has a sharp maximum, so if we choose $\alpha$, this maximum happens at
$N = N'$. First we define a Grand Partition function
\begin{align*}
    \mathcal{Z} \equiv \sum_{N'} Z(N') e^{-\alpha N'}
\end{align*}
so
\begin{align*}
    \mathcal{Z} &= \sum_R e^{-\beta(n_1 \epsilon_1 + n_2 \epsilon_2 + \dots)} e^{-\alpha (n_1 + n_2 + \dots)} \\
    &= \sum_{n_1 = 0}^\infty e^{-\beta(n_1 \epsilon_1) - \alpha n_1} \sum_{n_2 = 0}^\infty e^{-\beta(n_2 \epsilon_2) - \alpha n_2} \dots \\
    &= \frac{1}{1 - e^{-\beta \epsilon_1} e^{-\alpha}} \frac{1}{1 - e^{-\beta \epsilon_2} e^{-\alpha}} \dots
\end{align*}
where
\begin{align*}
    \ln \mathcal{Z} = - \sum_r \ln(1 - e^{-(\alpha + \beta e_r)})
\end{align*}
And using the taylor series approximation $\ln Z = \alpha N + \ln \mathcal{Z}$, and the maximum condition
\begin{align*}
    \pdv{\ln(Z(N') e^{-\alpha N'})}{N'}\eval_{N' = N} = 0
\end{align*}
and
\begin{align*}
    \pdv{N} \ln Z - \alpha = 0 \implies \alpha = \alpha(N)
\end{align*}
So we get
\begin{align*}
    N + \pdv{\ln\mathcal{Z}}{\alpha} = 0 \implies \pdv{\ln{Z(N)}}{\alpha} = 0
\end{align*}

\paragraph{Worksheet} From BE
\begin{align*}
    \ln (Z) = -\beta \mu N - \sum_R \ln(1 - e^{-\beta (\epsilon_r - \mu)})
\end{align*}
\begin{itemize}
    \item [1.] Determine $\bar n_S$ for BE
    \begin{align*}
        \bar n_S &= \frac{1}{e^{\beta (\epsilon_S - \mu)} - 1} \\
        &= \frac{1}{\beta} \frac{-\beta e^{-\beta (\epsilon_S - \mu)}}{1 - e^{-\beta (\epsilon_S - \mu)}} \\
        &= \frac{e^{-\beta (\epsilon_S - \mu)}}{1 - e^{-\beta (\epsilon_S - \mu)}} \\
        &= \frac{1}{e^{\beta (\epsilon_S - \mu)} - 1}
    \end{align*}
    \item [2.] Partition function for Fermions (Fermi-Dirac statistics):
    Since each $n_r$ can be 0 or 1, the partition function is
    \begin{align*}
        \mathcal{Z} &= \sum_{n_1 = 0}^1 e^{-\beta n_1 \epsilon_1 - \alpha n_1} \sum_{n_2 = 0}^1 e^{-\beta n_2 \epsilon_2 - \alpha n_2} \dots \\
        &= \qt(1 + e^{-\beta \epsilon_1 - \alpha}) \qt(1 + e^{-\beta \epsilon_2 - \alpha}) \dots
    \end{align*}
    so 
    \begin{align*}
        \ln\mathcal{Z} &= \sum_r \ln(1 + e^{-\beta \epsilon_r - \alpha})
    \end{align*}
    Thus the partition function is
    \begin{align*}
        \ln Z &= \alpha N + \ln \mathcal{Z} \\
        &= \alpha N + \sum_r \ln(1 + e^{-\beta \epsilon_r - \alpha})
    \end{align*}
\end{itemize}
\end{document}