\documentclass[../main.tex]{subfiles}

\graphicspath{{../images/}}

\begin{document}
\pagestyle{fancy}
\lhead{Lecture 9: 9/24/24}
\chead{Chapter 4.1,6}
\rhead{PHYS 463}

\section{Macroscopic Parameters and their measurement}

\subsection{Work \& internal energy}
From the first law of thermodynamics we always talk about
\begin{align*}
    Q = \Delta \bar E + W
\end{align*}
Given a system, work is \textit{easy} measure i.e. we integrate
\begin{align*}
    W = -\int p \dd V
\end{align*}
\paragraph{Measure of internal energy}
\begin{itemize}
    \item Thermal isolation case: $Q = 0$
    \begin{align*}
        \Delta \bar E = \bar E_b - \bar E_b = -W_{ab} = \int_a^b \dbar W
    \end{align*}
    e.g. a thermally isolated piston goes from state $a$ to $b$.
\end{itemize}

\subsection{Heat}

The heat absorbed by a system going from macrostate $a$ to $b$ is simply
\begin{align*}
    Q_{ab} = (\bar E_b - \bar E_a) + W_{ab}
\end{align*}
\paragraph{Example} A superconducting circuit $A$ is connected to the circuit $B$ with a resistor.

Adding $\qty{20}{\micro W}$ of heat to the system: we actuallly are doing work on a resistor.

\subsubsection*{Method of Mixers (Comparison Method)}

Bring system $A$ into contact with system $B$ that has a known relation between its internal energy and some parameters (T).
\begin{align*}
    Q_A = \Delta \bar E_B = -Q_B
\end{align*}
e.g. system $A$ is submerged in water $B$ and we can measure the change in internal energy of water quite easily.

\subsection{Entropy} 

We define entropy $S$
\begin{align*}
    dS = \frac{\dbar Q}{T}
\end{align*}
and \textbf{Absolute entropy} from the 3rd law
\begin{align*}
    T \to 0,\quad S \to S_0
\end{align*}

\paragraph{Example:} Tin

Two structures of a solid:
\begin{enumerate}
    \item White tin---a metal $\to$ stable $> 298$ K
    \item Grey tin---semiconductor $\to$ stable $< 298$ K
\end{enumerate}
Thus it requires some amount of heat $Q$ to transform from grey to white tin.
\begin{itemize}
    \item Case 1: a mole of white tin from $T=0 \to T_0$ with specific heat $C^{(w)}(T)$
    \begin{align*}
        S^{(w)} (T_0) = S^{(w)}(T = 0) + \int_0^{T_0} \frac{C^{(w)}(T)}{T} \dd T
    \end{align*}
    \item Case 2: Grey tin from $0$ K $\to T_0$ and then it transforms to white tin quasi-statically.
    It absorbs heat $Q$ and the entropy change is
    \begin{align*}
        S^{(w)}(T_0) = S^{(g)}(T=0) + \int_0^{T_0} \frac{C^{(g)} (T)}{T} dT + \frac{Q}{T_0}
    \end{align*}
    where
    \begin{align*}
        S^{(g)}(T=0) = S^{(w)}(T=0) = S_0
    \end{align*}
\end{itemize}

\begin{figure*}[ht]
    \centering
    \includegraphics[width=0.3\linewidth]{/moleoftin.png}
    \caption{Mole of Tin (DALL-E 3)}
    \label{fig:moleoftin}
\end{figure*}

\end{document}