\documentclass[../main.tex]{subfiles}

\graphicspath{{../images/}}

\begin{document}
\pagestyle{fancy}
\lhead{Homework 7}
\chead{Junseo Shin}
\rhead{PHYS 421}

\setcounter{section}{7}

% 4.15, 4.17, 4.19, 4.37
\paragraph{4.15} Thick spherical shell of inner radius $a$, outer radius $b$, with polarization
\begin{align*}
    \vb P (\vb r) = \frac{k}{r} \vu r    
\end{align*}
\begin{itemize}
    \item [(a)] E-field in all regions using bound charges: The volume bound charge is
    \begin{align*}
        \rho_b &= -\div \vb P = - \frac{1}{r^2} \pdv{r}(r^2 \frac{k}{r}) = -\frac{k}{r^2}
    \end{align*}
    and the surface bound charges are ($\vu n = -\vu r$ at $r = a$)
    \begin{align*}
        \sigma_b = \vb P \vdot \vu n = \begin{cases}
            -\dfrac{k}{a} & r = a \\
            \dfrac{k}{b} & r = b
        \end{cases}
    \end{align*}
    Using Gauss's Law
    \begin{align*}
        \oint \vb E \vdot \dd \vb a &= \frac{Q_\text{enc}}{\epsilon_0}
    \end{align*}
    \begin{itemize}
        \item [(i)] $r < a$: $Q_\text{enc} = 0$, so $\vb E = 0$
        \item [(ii)] $a < r < b$: The enclosed charge is the inner surface charge plus the volume charge:
        \begin{align*}
            Q_\text{enc} &= \oint_S \sigma_b \dd{\vb a} + \int_V \rho_b \dd{\tau} \\
            &= \int -\frac{k}{a} a^2 \sin\theta \dd{\theta} \dd{\phi} + \int -\frac{k}{r^2} r^2 \sin\theta \dd{r} \dd{\theta} \dd{\phi} \\
            &= -4\pi k a - 4\pi k (r - a) = -4\pi k r
        \end{align*}
        So using Gauss's Law
        \begin{align*}
            \oint \vb E \vdot \dd{\vb a} &= \frac{Q_\text{enc}}{\epsilon_0} \\
            \abs{\vb E} \oint \dd{a} &= -\frac{4\pi k r}{\epsilon_0} \\
            \abs{\vb E} 4\pi r^2 &= -\frac{4\pi k r}{\epsilon_0} \\
            \abs{\vb E} &= -\frac{k}{\epsilon_0 r} 
        \end{align*}
        or 
        \begin{align*}
            \vb E = -\frac{k}{\epsilon_0 r} \vu r
        \end{align*}
        \item [(iii)] $r > b$: The total enclosed charge of a dielectric is zero (from last HW 4.14), so $\vb E = 0$
    \end{itemize}
    \item [(b)] Using
    \begin{align*} \tag{4.23}\label{eq:4.23}
        \oint D \vdot \dd{\vb a} = Q_\text{free}
    \end{align*}
    and 
    \begin{align*}
        \vb D = \epsilon_0 \vb E + \vb P
    \end{align*}
    the total free enclosed charge is zero, so
    \begin{align*}
        \vb D = 0
    \end{align*}
    Thus
    \begin{align*}
        \epsilon_0 \vb E + \vb P = 0 
    \end{align*}
    or
    \begin{align*}
        \vb E = -\frac{1}{\epsilon_0} \vb P = \begin{cases}
            0 & r < a \qand r > b \\
            -\dfrac{k}{\epsilon_0 r} \vu r & a < r < b
        \end{cases}
    \end{align*}
\end{itemize}

\newpage
\paragraph{4.17} Bar electret from Prob. 4.11 has $\rho_b = 0$: From divergence theorem
\begin{align*}\tag{4.22}
    \int_V (\div \vb D) \dd{\tau} = \oint_S \vb D \vdot \dd{\vb a} = Q_\text{free} = 0 \implies \div \vb D = 0
\end{align*}
So, the field lines for $\vb D$ are closed loops as shown in Fig.

\newpage
\paragraph{4.19}
\end{document}