\documentclass[12pt]{article}

%
%Margin - 1 inch on all sides
%
\usepackage[letterpaper]{geometry}
\usepackage{times}
\geometry{top=1.0in, bottom=1.0in, left=1.0in, right=1.0in}

%
%Doublespacing
%
\usepackage{setspace}
\doublespacing

%
%Rotating tables (e.g. sideways when too long)
%
\usepackage{rotating}


%
%Fancy-header package to modify header/page numbering (insert last name)
%
\usepackage{fancyhdr}
\pagestyle{fancy}
\lhead{} 
\chead{} 
\rhead{Shin \thepage} 
\lfoot{} 
\cfoot{} 
\rfoot{} 
\renewcommand{\headrulewidth}{0pt} 
\renewcommand{\footrulewidth}{0pt} 
%To make sure we actually have header 0.5in away from top edge
%12pt is one-sixth of an inch. Subtract this from 0.5in to get headsep value
\setlength\headsep{0.333in}
\setlength\headheight{15pt}

%
%Works cited environment
%(to start, use \begin{workscited...}, each entry preceded by \bibent)
% - from Ryan Alcock's MLA style file
%
\newcommand{\bibent}{\noindent \hangindent 40pt}
\newenvironment{workscited}{\newpage \begin{center} Works Cited \end{center}}{\newpage }

% EXTRA PACKAGES
\usepackage{hyperref}

%
%Begin document
%
\begin{document}
\begin{flushleft}

%%%%First page name, class, etc
Junseo Shin\\
Professor Lyndsie Schultz\\
EDUC 301C: American School\\
\today{}\\


%%%%Title
\begin{center}
    End of Course Reflection: What Should Education Look Like?
\end{center}


%%%%Changes paragraph indentation to 0.5in
\setlength{\parindent}{0.5in}
%%%%Begin body of paper here

The American education system is one of the sole reasons why my family and I have moved 
from South Korea to the United States. After learning about the history of American schooling
and the current state of education in the United States, here is what I believe education should
look like in the future.

Education should be a public good, but not a pure public good. David F. Labaree states two goals of
education as a public good: democratic equality and social efficiency (42). I believe that the
future of education should focus on these two goals. In addition, the social mobility model of
education as a private goal should be eliminated.

To align the future of education with the democratic equality goal, we must instill common values
and a sense of social responsibility in students from a young age until the end of their education
i.e. high school or college. This can be achieved by incorporating civic education into the
curriculum at all levels of schooling. For example, we should change the purpose of school field
trips from visiting historical sites to local community centers and service projects. Although this
may not work for younger students, it can be easily implemented in middle and high schools. 
In addition, we can encourage civic engagement by requiring students to actively care for their
classroom and classmates by setting time aside for the school to clean up and organize the school. 
Furthermore, students should be taught to serve lunch to their classmates and teachers
and clean up after lunch.

Although this idea borrows from East Asian schools such as the Japanese school system, which have 
implemented the responsibility of the students to care for the school through required activities 
such as cleaning, this relates to Dewey's idea of education as a curriculum reformer
who believed that students should be ``trained…into membership, saturated with the spirit of
service, and provided with the instruments of effective self-direction,'' through hands-on,
experiential learning (Fraser 207). Although these changes may seem foreign to the American
education system, we can reframe it using Dewey's pragmatic philosophy and move
education towards a more democratic and socially responsible institution with these changes.

To fund the model of education, I believe that all schools should be funded equally. This
does not mean that all schools should receive the same amount of money, but rather that all schools
should be funded based on the resources the students need. This would call for the local and state
tax dollars to be the primary source of funding and any excess funds raised by the local county
should be redistributed to other schools within the state or region.

Furthermore, there should be a shift to incentivize schools to spend more public funding on the
students and teachers rather than administrative costs. According to the National Center for
Education Statistics, the number of public school teachers has increased by 8.7\% from 2000 to 2019,
and the number of public school administrators has increased by 98\% during the same period.
This could be an indication that schools are spending more money on administrative costs rather than
on the students and teachers. To combat this, there should be a reform on teacher salaries and a
reduction of administrative salaries. Moreover, the growth of a teacher's salary should be tied to
the growth of the student's performance and education which would incentivize teachers to readily
engage with their students and provide a better education. From the taxpayer's perspective,
this would highlight the social efficiency goal of education because there is a direct connection
between the money spent on education and the quality of education.

When education becomes meritocratic and stresses a graded hierarchy to propagate social mobility,
it becomes privatized and creates a divide between the privileged and marginalized. It widens the 
gap between people who can afford ``better education'' through private schools, tutoring services,
etc. So we should eliminate the graded hierarchy and implement a pass or fail system. This
takes away the need for students to compete with each other and invite a more collaborative
environment in the classroom. In addition, we can readily implement LREs
(Least Restrictive Environment) to integrate students with disabilities into the general education
classroom (Bicard 324). This will emphasize a classroom with equal treatment and access to students
of all abilities and backgrounds. Although not every student can be integrated into the classroom,
schools must provide the necessary resources and support for students with
disabilities who can't be integrated into the general education classroom rather than putting the
full responsibility of education on the parents. Finally, this would necessitate private schools to
be abolished because private schools separate students based on their socio-economic status which 
contradicts the goals of implementing a least restrictive environment with a diverse and inclusive
student body.

Education should be a public good that truly provides meaningful education and social growth
for all students. As America seems to be moving towards a more divided society from the radical
socio-political climate to the wealth gap across the country, it is imperative that we first tackle
educational debt and strive for a more democratic and socially aware education system because it
is the education of the next generation that will shape the future. 






%%%%Title
% \begin{center}
% Notes
% \end{center}


% \setlength{\parindent}{0.5in}

% 1. Danhof includes “Delaware, Maryland, all states north of the Potomac and Ohio rivers, Missouri, and states to its north” when referring to the northern states (11).


% 2. For the purposes of this paper,“science” is defined as it was in nineteenthcentury agriculture: conducting experiments and engaging in research.


% 3. Please note that any direct quotes from the nineteenth century texts are writtenin their original form, which may contain grammar mistakes according to twenty-first century grammar rules.

%%%%Works cited
\begin{workscited}

\bibent
Labaree, David F.
``Public Goods, Private Goods: The American Struggle over Educational
Goals.'' \textit{American Educational Research Journal}, vol. 34, no. 1, 1997, pp. 39–81. JSTOR,
\href{https://doi.org/10.2307/1163342}{https://doi.org/10.2307/1163342}.

\bibent
Fraser, J.W. (2014). Rights and Opportunities in American Education, 1965-1980.
\textit{The School in the United States}. Routledge.

\bibent
Bicard, S.C. \& Heward, W.L. (2010). Educational Equality for Students with Disabilities.
In J.A. Banks \& C.A. McGee Banks (Eds.), \textit{Multicultural} Education (pp. 315-341).
John Wiley \& Sons, Inc.

\bibent
National Center for Education Statistics.
\textit{Digest of Education Statistics, 2021: Table 213.10.} U.S. Department of Education,
Institute of Education Sciences, 2021,
\href{https://nces.ed.gov/programs/digest/d21/tables/dt21_213.10.asp?current=yes}{https://nces.ed.gov/programs/digest/d21/tables/dt21\_213.10.asp?current=yes}.


\end{workscited}



\end{flushleft}
\end{document}