\documentclass[../main.tex]{subfiles}

\graphicspath{{../images/}}

\begin{document}

\pagestyle{fancy}
\lhead{Lecture 1: 1/13/25}
\chead{BOP: Chapter 2}
\rhead{MATH 310}

%%%%% 240 notes %%%%%
% \newpage

% \section*{240 Lecture Notes*}

% \subsection{Propositional Logic}

% Proposition $=$ statement that has a \underline{true value} (T or F)
% \begin{quote}
%     $p = $ ``$1 + 1 = 2$'': T

%     $q = $ ``St. Louis is the capial of MO'': F
% \end{quote}

% \paragraph{\underline{Negation}} NOT $\quad \neg$

% \begin{quote}
%     $\neg p =$ ``not p'' or ``$p$ is false''
% \end{quote}

% Or in a truth table:
% \begin{table}[ht]
%     \centering
%     \begin{tabular}{c|c}
%         $p$ & $\neg p$ \\
%         \hline
%         T & F \\
%         F & T
%     \end{tabular}
% \end{table}

% e.g. from before:

% \begin{itemize}
%     \item $\neg p = $ ``$1 + 1 \neq 2$'': F
%     \item $\neg q = $ ``St. Louis is \emph{not} the capital of MO'': T
% \end{itemize}

% \paragraph{\underline{Conjunction}:} AND $\quad \land$

% \begin{quote}
%     $p \land q =$ ``$p$ and $q$'' or ``both $p$ and $q$ are true''
% \end{quote}

% Or in a truth table:
% \begin{table}[ht]
%     \centering
%     \begin{tabular}{c|c|c}
%         $p$ & $q$ & $p \land q$ \\
%         \hline
%         T & T & T \\
%         T & F & F \\
%         F & T & F \\
%         F & F & F
%     \end{tabular}
% \end{table}

% e.g.
% \begin{quote}
%     $p = $ ``Alan Turing was born in England'': T

%     $q = $ ``Alan Turing was born in 1912'': T

%     $p \land q = $ ``Alan Turing was born in England in 1912'': T
% \end{quote}

% \paragraph{\underline{Disjunction}} OR $\quad \lor$

% \begin{quote}
%     $p \lor q =$ ``$p$ or $q$'' or ``$p$ is true or $q$ is true (or both)'' (inclusive)
% \end{quote}

% As a truth table:
% \begin{table}[ht]
%     \centering
%     \begin{tabular}{c|c|c}
%         $p$ & $q$ & $p \lor q$ \\
%         \hline
%         T & T & T \\
%         T & F & T \\
%         F & T & T \\
%         F & F & F
%     \end{tabular}
% \end{table}

% e.g.
% \begin{quote}
%     $p = $ ``2 is a prime number'': T

%     $q = $ ``The Blues will win the Stanley Cup this year''

%     $p \lor q = $ T (since $p$ is true, we can't determine the truth value without knowing $q$)
% \end{quote}

% To wrap this around our head listen to
% \href{https://youtu.be/LjdCFat9rjI?si=1ZF7paZ8Hdt_-iI7}{Conjunction Junction} - Schoolhouse Rock.

% \paragraph{\underline{Exclusive OR}} XOR $\quad \oplus$

% \begin{quote}
%     $p \oplus q =$ ``$p$ x-or $q$'' 
%     or ``$p$ or $q$ is true but not both''
% \end{quote}

% As a truth table:
% \begin{table}[ht]
%     \centering
%     \begin{tabular}{c|c|c}
%         $p$ & $q$ & $p \oplus q$ \\
%         \hline
%         T & T & F \\
%         T & F & T \\
%         F & T & T \\
%         F & F & F
%     \end{tabular}
% \end{table}

% Rather than using T or F we can use bits (1 or 0) to represent the truth values such that
% \begin{align*}
%     1 \oplus 1 = 2 \equiv 0 (\text{mod } 2)
% \end{align*}

% \paragraph{\underline{Multiple Propositions}}

% \begin{quote}
%     $p \land q \land r =$ ``All of $p, q, r$ are true''

%     $p \lor q \lor r =$ ``At least one of $p, q, r$ is true''
% \end{quote}

% For the truth table:
% \begin{table}[ht]
%     \centering
%     \begin{tabular}{c|c|c|c|c}
%         $p$ & $q$ & $r$ & $p \land q \land r$ & $p \lor q \lor r$ \\
%         \hline
%         T & T & T & T & T \\
%         T & T & F & F & T \\
%         T & F & T & F & T \\
%         T & F & F & F & T \\
%         F & T & T & F & T \\
%         F & T & F & F & T \\
%         F & F & T & F & T \\
%         F & F & F & F & F
%     \end{tabular}
% \end{table}

% which can be generalized to $n$ propositions.

% \paragraph{\underline{Truth Tables for Compound propositions}}

% \begin{quote}
%     $(p \lor q) \land (\neg q \lor r)$
% \end{quote}

% So filling out the truth table:
% \begin{table}[ht]
%     \centering
%     \begin{tabular}{c|c|c|c|c|c|c}
%         $p$ & $q$ & $r$ & $\neg q$ & $p \lor q$ & $\neg q \lor r$ & $(p \lor q) \land (\neg q \lor r)$ \\
%         \hline
%         T & T & T & F & T & T & T \\
%         T & T & F & F & T & F & F \\
%         T & F & T & T & T & T & T \\
%         T & F & F & T & T & T & T \\
%         F & T & T & F & T & T & T \\
%         F & T & F & F & T & F & F \\
%         F & F & T & T & F & T & F \\
%         F & F & F & T & F & T & F
%     \end{tabular}
% \end{table}

% We don't always have to construct truth tables, especially when we are given the truth values of the
% propositions---e.g.
% \begin{align*}
%     (\neg p \land q) \lor (q \land \neg r) \qquad p = T \; q = F \; r = T \\
%     (\neg T \land F) \lor (F \land \neg T) = (F \land F) \lor (F \land F) = F \lor F = F
% \end{align*}

\section{The Foundations: Logic and Proofs}

Consider the following argument:
\begin{quotation}
    i eat chocolate if i am depressed

    i am not depressed

    therefore i am not eating chocolate
\end{quotation}
Obviously, the logic is flawed\dots but how do we write this in a more formal way?

\subsection{Propositional Logic}

A \emph{statement} is a senetence or mathematical expression that is either \emph{true} or
\emph{false}---e.g.
\begin{itemize}
    \item $P:$ The number 3 is odd
    \item $Q:$ The number 6 is even
    \item $R:$ The number 4 is odd
\end{itemize}

\subsection*{\underline{Not a statement}}
\begin{itemize}
    \item $x > 2$ (the true value depends on $x$)
    \item $x = 2, \; t + 4q = 17$
\end{itemize}

\subsection*{\underline{Combining statements}}

Given statements $P$ and $Q$:
\begin{itemize}
    \item ``$P$ and $Q$'' is a statement ($P \land Q$)
    \item ``$P$ or $Q$'' is a statement ($P \lor Q$)
\end{itemize}

We can construct a truth table to represent the truth values of $P \land Q$ and $P \lor Q$:
\begin{table}[ht]
    \centering
    \begin{tabular}{c|c|c}
        $P$ & $Q$ & $P \land Q$ \\
        \hline
        T & T & T \\
        T & F & F \\
        F & T & F \\
        F & F & F
    \end{tabular}
    % add vertical separator
    \hspace{2cm}
    \begin{tabular}{c|c|c}
        $P$ & $Q$ & $P \lor Q$ \\
        \hline
        T & T & T \\
        T & F & T \\
        F & T & T \\
        F & F & F
    \end{tabular}
    \caption{Truth tables for conjunction ($\land$) and disjunction ($\lor$)}
\end{table}

\subsection*{\underline{Conditional Statements}}

The expression:
\begin{quote}
    If $P$, then $Q$ (or $P \Rightarrow Q$)
\end{quote}
is a \emph{conditional statement}.
% insert truth table for conditional statements
\begin{table}[ht]
    \centering
    \begin{tabular}{c|c|c}
        $P$ & $Q$ & $P \Rightarrow Q$ \\
        \hline
        T & T & T \\
        T & F & F \\
        F & T & T \\
        F & F & T
    \end{tabular}
    \caption{Truth table for conditional statements}
\end{table}

\paragraph{Example:}
\begin{quote}
    $P(n)$: The integer $n$ is odd

    $Q(n)$: The integer $n^2$ is odd
\end{quote}
$P(n)$ and $Q(n)$ are not statements, but they are \emph{predicates} (statements once $n$ is
determined). So the conditional statement is

\begin{quote}
    $P(n) \Rightarrow Q(n)$: If the integer $n$ is odd, then the integer $n^2$ is odd
\end{quote}

\subsection*{\underline{Proving a statement of the form $P \Rightarrow Q$}}

\begin{enumerate}
    \item Direct proof: Assume $P$ is true and ``prove'' that $Q$ is also true
\end{enumerate}


\newpage
\lhead{Lecture 2: 1/15/25 by Jesus!}

Example: Let's construct a truth table for $(P \lor Q) \Rightarrow R$
\begin{table}[ht]
    \centering
    \begin{tabular}{c|c|c|c|c}
        $P$ & $Q$ & $R$ & $P \lor Q$ & $(P \lor Q) \Rightarrow R$ \\
        \hline
        T & T & T & T & T \\
        T & F & T & T & T \\
        T & F & F & T & F \\
        F & T & T & T & T \\
        F & F & T & F & T \\
        F & F & F & F & T
    \end{tabular}
    \caption{Truth table for $(P \lor Q) \Rightarrow R$}
\end{table}

Where we want to prove
\begin{quote}
    If $n$ is odd, then $n^2$ is odd.
\end{quote}
The first proposition is symbolically $O(n): n$ is odd, and the conditional statement is
\begin{align*}
    O(n) \Rightarrow O(n^2)
\end{align*}
\paragraph{Def}

First we define and integer $n$ odd if $n = 2k + 1$ for some integer $k$. An integer is even if
$n = 2k$ for some integer $k$. 

\begin{remark}
    The set of integers
    \begin{align*}
        \mathbb{Z} = \{ \dots, -3, -2, -1, 0, 1, 2, 3, \dots \}
    \end{align*}
    where $k$ is an integer is denoted as $k \in \mathbb{Z}$.
\end{remark}


\paragraph{Proof}

Suppose $n$ is odd. So by definition, $n = 2k + 1$ for some $k \in \mathbb{Z}$.

\begin{align*}
    \implies n^2 = (2k + 1) (2k + 1) = 4k^2 + 4k + 1 = 2(2k^2 + 2k) + 1
\end{align*}

Since $2k^2 + 2k$ is an integer, we have that $n^2$ is in fact odd. $\square$

\paragraph{Another Example} (Because students love examples)
Suppose $x$ and $y$ are positive numbers. Prove that if $x < y$ then $x^2 < y^2$.

\paragraph{Sol}

Suppose $x$ and $y$ are positive real numbers and further suppose that $x < y$. A fundamental
property of $<$ on the real numbers is that if $a < b$ and $c > 0$, then $a \cdot c < b \cdot c$
since if 
\begin{align*}
    a < b \implies 0 < b - a
\end{align*}
and the product of the two positve numbers is positive, i.e.
\begin{align*}
    0 < c(b - a) = c b - c a
\end{align*}
Which now implies $ca < cb$. In this case, if $a = x, b = y, c = x$, then
\begin{align*}
    x^2 = x \cdot x < x \cdot y
\end{align*}
Now if we swap and use $c = y$, we have
\begin{align*}
    x \cdot y < y \cdot y = y^2
\end{align*}
Concatenating the two inequalities, we find that
\begin{align*}
    x^2 = x \cdot x < x \cdot y < y \cdot y = y^2
\end{align*}
Because $x$ and $y$ were arbitrary positive numbers, the conclusion holds. $\square$

\subsection{Logical Equivalence}

Two statements are \emph{logically equivalent} if they have the same truth value, e.g. $x$ \& $y$
are real numbers
\begin{quote}
    $P: \; x \cdot y = 0$

    $Q: \; x = 0$ or $y = 0$
\end{quote}
are equivalence since they are either both T or both F.

\paragraph{}

If $P$ and $Q$ are equivalent we say $P$ if and only if $Q$ and we write
\begin{align*}
    P \iff Q \qqtext{or} P \equiv Q
\end{align*}
which is a \emph{biconditional statement}. Note that $P$ \& $Q$ are predicates but $P \iff Q$ is a
statement.

\paragraph{Example} $P, Q,$ and $R$ are statements
\begin{align*}
    ((P \lor Q) \Rightarrow R) \iff ((P \Rightarrow R)) \land (Q \Rightarrow R)
\end{align*}
% truth table
\begin{table}[ht]
    \centering
    \begin{tabular}{c|c|c|c|c|c|c|c}
        $P$ & $Q$ & $R$ & $P \lor Q$ & $P \Rightarrow R$ & $Q \Rightarrow R$ & $(P \lor Q) \Rightarrow R$  & $(P \Rightarrow R) \land (Q \Rightarrow R)$ \\
        \hline
        T & T & T & T & T & T & T & T \\
        T & T & F & T & F & F & F & F \\
        T & F & T & T & T & T & T & T \\
        T & F & F & T & F & T & F & F \\
    \end{tabular}
    \caption{Truth table}
\end{table}

\newpage
\lhead{Lecture 3: 1/17/25 by Jesus (Sanchez)!}

\paragraph{\underline{Contrapositive}}

The \emph{contrapositive} state is

\begin{quote}
    If not $Q$, then not $P$
\end{quote}

\paragraph{Claim}

The statement $P \Rightarrow Q$ and its contrapositive $\neg Q \Rightarrow \neg P$ are logically 
equivalent.

\paragraph{Proof}

For fun watch the YouTube video \href{https://youtu.be/QcLfb0PhfO0?si=ngcYSId-1LFhW5fA}{Not Knot}

\begin{table*}[h]
    \centering
    \begin{tabular}{c|c|c|c|c|c}
        $P$ & $Q$ & $P \Rightarrow Q$ & $\neg Q$ & $\neg P$ & $\neg Q \Rightarrow \neg P$ \\
        \hline
        T & T & T & F & F & T \\
        T & F & F & T & F & F \\
        F & T & T & F & T & T \\
        F & F & T & T & T & T
    \end{tabular}
    \caption{Truth table proof}
\end{table*}

\paragraph{Remark} 

A proof of a condition statement by proving the contrapositive is called a \emph{contrapositive proof}.

\paragraph{Example}

Let's prove the statement
\begin{quote}
    Suppose $x$ is a real number. If $x^2 + 5x < 0$, then $x <0$
\end{quote}
using a contrapositive proof.

\paragraph{Proof}

\begin{align*}
    P:& \; x^2 + 5x < 0 \\
    Q:& \; x < 0
\end{align*}
So $\neg Q \Rightarrow \neg P$ is
\begin{quote}
    If $x \geq 0$, then $x^2 + 5x \geq 0$
\end{quote}
Suppose $x$ is a real number satisfying $x \geq 0$. Then $5x \geq 0$ \& $x^2 \geq 0$. Thus
\begin{align*}
    x^2 + 5x \geq 0
\end{align*}
Because $x \geq 0$ was arbitrary, we have $\neg Q \Rightarrow \neg P$.

\paragraph{\underline{Converse}}

$Q \Rightarrow P$ is called the \emph{converse} of $P \Rightarrow Q$.

\paragraph{Example}

\begin{quote}
    $P$: $f$ is differentiable at $x = 0$

    $Q$: $f$ is continuous at $x = 0$
\end{quote}

As an example, $f = |x|$ is continuous at $x = 0$ but not differentiable at $x = 0$---so here
\begin{quotation}
    $P \Rightarrow Q$ is true, but

    $Q \Rightarrow P$ is false
\end{quotation}
Another example is
\begin{quotation}
    $P$: $A$ is an invertible $2 \times 2$ matrix

    $Q$: $\det A \neq 0$
\end{quotation}

\newpage
\paragraph{\underline{Negation \& Quantifiers}}

\paragraph{Example}

Let $m$ and $n$ be integers. If 4 divides the product $mn$ (results in an integer),
then 4 divides $m$ or 4 divides $n$.

\begin{itemize}
    \item Converse: If 4 divides $m$ or 4 divides $n$, then 4 divides $mn$
    \item Contrapositive: If 4 does not divide $m$ and 4 does not divide $n$, then 4 does not divide $mn$
\end{itemize}

This statement is False!

\paragraph{Proof}

If $m = n = 2$, then 4 divides $mn = 4$. But 4 does \emph{not} divide $m$ or $n$, thus the 
statement is F. $\square$

The \emph{negation} of a statement $P$ is the statement whose truth values are opposite for those of
$P$ and is denoted as $\neg P$.

\paragraph{Claim} Let $P$ and $Q$ be statements.

The negation of the conditional statement $P \Rightarrow Q$ is $P \land (\neg Q)$.

\paragraph{Proof}
    We check that $\neg(P \Rightarrow Q)$ and $P \land (\neg Q)$ are logically equivalent with a
    truth table.

\begin{table}[ht]
    \centering
    \begin{tabular}{c|c|c|c|c|c}
        $P$ & $Q$ & $P \Rightarrow Q$ & $\neg(P \Rightarrow Q)$ & $\neg Q$ & $P \land (\neg Q)$ \\
        \hline
        T & T & T & F & F & F \\
        T & F & F & T & T & T \\
        F & T & T & F & F & F \\
        F & F & T & F & T & F
    \end{tabular}
    \caption{Truth table for negation of a conditional statement}
\end{table}

\paragraph{Discussion}

Let $P$ and $Q$ be statements and negate $P \lor Q$, and find what it is equivalent to.

\begin{table}[ht]
    \centering
    \begin{tabular}{c|c|c|c|c}
        $P$ & $Q$ & $P \lor Q$ & $\neg(P \lor Q)$ & $\neg P \land \neg Q$ \\
        \hline
        T & T & T & F & F \\
        T & F & T & F & F \\
        F & T & T & F & F \\
        F & F & F & T & T
    \end{tabular}
    \caption{Truth table for negation of a disjunction}
\end{table}

So the two statements are logically equivalent $\neg(P \lor Q) \Longleftrightarrow \neg P \land \neg Q$.
This is one of De Morgan's Laws:

\begin{table}[ht]
    \centering
    \begin{tabular}{c}
        $\neg(P \lor Q) \Longleftrightarrow \neg P \land \neg Q$ \\
        $\neg(P \land Q) \Longleftrightarrow \neg P \lor \neg Q$
    \end{tabular}
    \caption{De Morgan's Laws}
\end{table}

\newpage
\paragraph{Example}

Every nonempty subset of $\mathbb{N}$ has a smallest element.

\paragraph{Notation} $\mathbb{N} = \{0, 1, 2, 3, \dots\}$ is the set of natural numbers.

\paragraph{Definition} The symbols $\forall$ and $\exists$ are called \emph{quantifiers}.

\begin{itemize}
    \item $\forall$ stands for ``for all'' or ``for every''
    \item $\exists$ stands for ``there exists'' or ``there is''
\end{itemize}

thus we write the above statement as logical mathematical symbols is

\begin{quote}
    $\forall X \subset \mathbb{N}$ with $X \neq \phi$, $\exists x_0 \in X$ such that
    $x_0 \leq x \quad \forall x \in X$
\end{quote}

\newpage
\lhead{Lecture 4: 1/22/25}

\paragraph{HW NOTES}

\begin{align*}
    (P \Leftrightarrow Q) \equiv [ (P \Rightarrow Q) \land (Q \Rightarrow P) ]
\end{align*}

Show both $P \Rightarrow Q$ and $Q \Rightarrow P$ are true.

\paragraph{Example}

Negate the statemen:

\begin{quote}
    The integers 5 \underline{and} 9 are both odd.
\end{quote}

Using De Morgan's Laws $\neg(P \land Q) \equiv \neg P \lor \neg Q$ we can rewrite the statement as

\begin{quote}
    Either 5 is even or 9 is even.
\end{quote}

Let $A$ be a set and $a \in A$.

\begin{itemize}
    \item $\forall a \in A, P(a)$: means $P(a)$ is true \underline{for every} element of set $A$.
    \item $\exists a \in A, P(a)$: means $P(a)$ is true \underline{for some} element of set $A$.
    \item $\neg (\forall a \in A, P(a)) \equiv \exists a \in A, \neg P(a)$
    \item $\neg (\exists a \in A, P(a)) \equiv \forall a \in A, \neg P(a)$
\end{itemize}

\paragraph{WARNING}:

\begin{itemize}
    \item $\neg (a \in A) \equiv a \notin A$ is not the same as
    \item $\neg (\forall a \in A) \equiv \exists a \in A$
\end{itemize}

\paragraph{Example}
Let $C(x)$: $x$ has taken calculus ($x$ is a 310 student).

\begin{quote}
    $G(x,y):\quad x > y \qquad (x, y \in \mathbb{R})$ 

    $P(x): \quad x$ is prime $\qquad (x \in \mathbb{N} = \{0, 1, 2, \dots\})$
\end{quote}

\begin{enumerate}
    \item $\forall x, C(x)$ as a statement: Every 310 student has taken calculus
    
    \underline{Negation}: There is some 310 student who has not taken calculus, or

    \item $\exists x, C(x)$
    
    \item Negate $\forall x \in \mathbb{N}, \neg P(x)$

    \begin{quote}
        Statement: Every natural number is not prime.
    
        \underline{Negation}: $\exists x \in \mathbb{N}, P(x)$---There exist a natural number that is
        prime.
    \end{quote}
    
    \item Negate $\exists x \in \mathbb{R}, G(x, 2)$
    
    \begin{quote}
        Statement: There exists a real number greater than 2.

        \underline{Negation}: $\forall x \in \mathbb{R}, \neg G(x, 2)$---Every real number is less
        than or equal to 2. $\iff$
    \end{quote}
\end{enumerate}

\paragraph{Example} Negate the following statements:

\begin{enumerate}
    \item For all $X \subseteq \mathbb{N}$, there exists an integer $n$ such that $|X| = n$.

    Symbolically: $\forall X \subseteq \mathbb{N}\quad \exists n \in \mathbb{Z},\quad |X| = n$. Where $|X|$
    is ``the number of elements in the set $X$, cardinality of $X$''.

    e.g.
    \begin{itemize}
        \item $X = \{1, 2, 3\}$ then $|X| = 3$
        \item All even natural numbers $X = \{0, 2, 4, 6, 8, \dots\}$
        
        then $|X| = \infty$,
        so $\cancel{\exists}$ an integer $n$ such that $|X| = n$.
    \end{itemize}

    Thus the negatation
    $\exists X \subseteq \mathbb{N}\quad \forall n \in \mathbb{Z},\quad |X| \neq n$
    shows that the statement is \underline{false}.
    \item There exists $x \in \mathbb{Z}$ such that for all $n \in \mathbb{Z},\quad x \neq n + 2$.
    
    Symbolically: $\exists x \in \mathbb{Z}\quad \forall n \in \mathbb{Z},\quad x \neq n + 2$.

    Negation: $\forall x \in \mathbb{Z}\enspace \exists n \in \mathbb{Z},\quad x = n + 2$.

    which is \underline{true}.
    \item For all $x \in \mathbb{R}$, there exists $y \in \mathbb{R}$ such that $y^3 = x$.
    
    \dots this is \underline{true}
    \item There exists $x \in \mathbb{Z}$ such that for all $n \in \mathbb{Z},\quad x \neq n + 2$.
    
    \dots this is \underline{false}.
\end{enumerate}

% \begin{itemize}
%     \item $\Leftrightarrow$ vs. $\iff$
%     \item $\Rightarrow$ vs. $\implies$
%     \item $\neg$ vs. $\sim$
% \end{itemize}

\newpage
\lhead{Lecture 5: 1/24/25}
\chead{BOP: Chapter 1}

\paragraph{Example} True or False; Negate

\begin{enumerate}
    \item For all $x \in \R$, there exists $y \in \R$ such that $y^2 = x$
    \begin{align*}
        \forall x \in \R \enspace \exists y \in \R, \enspace y^2 = x
    \end{align*}
    Negation: $\exists x \in \R \enspace \forall y \in \R, \enspace y^2 \neq x$
    \begin{quote}
        There exists $x \in \R$ so that for all $y \in \R$, $y^2 \neq x$
    \end{quote}
    The original statement is \underline{false}:
    \begin{quote}
        Let $x = -1$. Then $y^2 \neq -1 \enspace \forall y \in \R$
    \end{quote}
    
    \item For all $x \in \R$, there exists $y \in \R$ such that $y^3 = x$.
    \begin{align*}
        \forall x \in \R \enspace \exists y \in \R, \enspace y^3 = x
    \end{align*}
    Negation: $\exists x \in \R \enspace \forall y \in \R, \enspace y^3 \neq x$

    The original statement is \underline{true} because every real number has a cube root.
\end{enumerate}

\paragraph{Definition} A \emph{set} is a collection of objects.

The objects in a set are called \emph{elements}.

\paragraph{Definition} The unique set containing no elements is called the \emph{empty set}, 
denoted by $\emptyset$ or $\varnothing$.

\paragraph{Example}

$ A = \{1, 2, 3, 4, 5, \{6, 7\}\}$

\begin{enumerate}
    \item [(a)] $1 \in A$ (1 is an element of $A$) T
    \item [(b)] $\{1\} \in A \qquad$ F
    \item [(c)] $1 \subseteq A \qquad$ F
    \item [(d)] $\{1\} \subseteq A \qquad$ F
    \item [(e)] $\{6, 7\} \subseteq A \qquad$ F
    \item [(e)'] $\{\{6,7\}\} \subseteq A \qquad$ T
    \item [(f)] $\{4, 5\} \subseteq A \qquad$ T
    \item [(g)] $|A| = 6 \qquad$ T
    \item [(h)] $\emptyset \in A \qquad$ F
\end{enumerate}

\paragraph{Set-builder notation} used to describe sets when its difficult to list all elements.

\paragraph{Example} Even integers $\{\dots, -4, -2, 0, 2, 4, \dots\}$

\begin{align*}
    = \{2k \; | \;  k \in \Z\} = \{2k : k \in \Z\}
\end{align*}

\paragraph{Example} The set of rational numbers

\begin{align*}
    \Q :=\qt{\frac{P}{q} \; | \; p, q \in \Z, \; q \neq 0}
\end{align*}

The set of \emph{irrational numbers} is set of all real numbers that are not rational.

\paragraph{Remark} $\N \subsetneq \Z \subsetneq \Q \subsetneq \R$

\paragraph{Example} Write in set-builder notation:

\begin{enumerate}
    \item \(\qt{\dots, \frac{1}{27}, \frac{1}{9}, 1, 3, 9, 27 \dots}\)
    \begin{align*}
        = \{3^k \; | \; k \in \Z\}
    \end{align*}
    \item The set of odd integers
    \begin{align*}
        \{2k + 1 \; | \; k \in \Z\}
    \end{align*}
    \item $(-\infty, 3] = \{x \in \R \; | \; x \leq x\}$
\end{enumerate}

\paragraph{Definition} Let $A$ and $B$ be sets.

\begin{itemize}
    \item \emph{Union}: $A \cup B := \{x \mid x \in A \lor x \in B\}$

    \item \emph{Intersection}: $A \cap B := \{x \mid x \in A \land x \in B\}$
    
    Definition: The sets $A$ and $B$ are \emph{disjoint} if $A \cap B = \emptyset$. $\quad \varnothing$

    \item \emph{Set-difference}: $A - B = A \setminus B := \{x \in A \mid x \notin B\}$
    
    \item The \emph{compliment} of $A$ in a set $U$ is
    $A^c = \overline{A} := \{x \in U \mid x \notin a\}$

    \item \emph{Cartesian product}:
    \begin{align*}
        A \cross B := \{(a, b) \mid a \in A, b \in B\}
    \end{align*}
    (e.g. $\R \cross \R = \R^2$)
\end{itemize}

\paragraph{T/F}
\begin{enumerate}
    \item $A \cross B = B \cross A \qquad$

    F: $A = \{1\}, \; B = \{2\}$, so $A \cross B = \{(1, 2)\}$ but $B \cross A = \{(2, 1)\}$
    \item If $|A| = 2$ and $|B| = 3$, then $|A \cross B| = 6 \qquad$ T
    \item $\R \subseteq \R^2 \qquad$ F
    \item [4'.] $\R \cross \{O\} = \R^2 \qquad$ T
\end{enumerate}

\newpage
\lhead{Lecture 6: 1/27/25}

\paragraph{Example} Write out the sets by listing all elements:

\begin{enumerate}
    \item 
    \(
        \{x \in \R \mid \cos(x) = 0, \, 0 \leq x \leq 2\pi\}
    \)
    \begin{align*}
        = \qt{
            \frac{\pi}{2}, \frac{3\pi}{2}
        }
    \end{align*}
    \item 
    \(
        \{x \in \R \mid \sin(x) = 0, \, 0 \leq x \leq 2\pi\}
    \)
    \begin{align*}
        = \qt{
            0, \pi, 2\pi
        }
    \end{align*}

    \item
    \(
        \{
            m \mid m \in \N,\, m^2 < 10
        \}
    \)
    \begin{align*}
        = \{1, 2, 3, 0\}
    \end{align*}
\end{enumerate}

\paragraph{Example} Compute the following sets:

\begin{enumerate}
    \item 
    \(
        \bigcup\limits_{n \in \N} \qt[
            \frac{1}{n +1} ,\, n + 1 
        ]
        = \color{draculagreen} (0, \infty)
    \)

    Looking at a few of our favorite natural numbers\dots
    \begin{itemize}
        \item $n = 4$: $\qt[\frac{1}{5}, 5]$
        \item $n = 0$: $[1, 1] = \{1\}$
        \item $n = 2$: $\qt[\frac{1}{3}, 3]$
    \end{itemize}
    So the union of all these sets is $\qt(0, \infty)$.

    \item \( \bigcap\limits_{n \in \N} \qt[\frac{1}{n + 1} ,\, n + 1] 
    = \color{draculagreen} \{1\} \)

    The intersection of all these sets is when $n = 0$ because that is when the two values are equal
    to each other.
\end{enumerate}

\paragraph{Claim} Let $A, B,$ and $C$ be sets.

If $B \subseteq C$, then $A \cross B \subseteq A \cross C$.

\begin{proof}
    Let $(a, b) \in A \cross B$. By definition of the Cartesian product, $a \in A$ and $b \in B$.
    Since $B \subseteq C$, $b \in C$. Thus, $(a, b) \in A \cross C$.
\end{proof}

\paragraph{Claim} For all sets $A$ and $B$, $(A \cup B)^c = A^c \cap B^c$.

\begin{proof}
    $(\subseteq)$ Let $x \in (A \cup B)^c$. 

    This implies $x \notin A$ and $x \notin B$. Thus, $x \in A^c$ and $x \in B^c$ so
    $x \in A^c \cap B^c$.

    \paragraph{}
    $(\supseteq)$ Let $x \in A^c \cap B^c$, so $x \notin A$ and $x \notin B$.

    This implies $x$ is not in $A$ or $B$. Thus, $x \notin A \cup B$ so $(A \cup B)^c$.
\end{proof}

\paragraph{Claim} $\Z = \{25 a + 24b \mid a, b \in \Z\}$.

\begin{proof}
    $(\supseteq)$ This is obvious, since $25a +24b \in \Z$ for all $a, b \in \Z$.

    \paragraph{}
    $(\subseteq)$ Let $k \in \Z$. Set $a = k$ and $b = -k$. Then $25 a + 24b = 25k - 24k = k$.
\end{proof}

To get to the forwards proof we can test a few values of $k$ to find anything:
\begin{itemize}
    \item $k = 0$: $25(0) + 24(0) = 0$
    \item $k = 1$: $25(1) + 24(-1) = 1$
    \item $k = 2$: $25(2) + 24(-2) = 2$\dots so we can see the pattern
    \item $k = 25k + 24(-k)$
\end{itemize}

\newpage
\lhead{Lecture 7: 1/29/25}

\subsection{Proof by Contradiction}
\paragraph{Example}

Suppose $A, B$, and $C$ are nonempty sets

\begin{quote}
    T/F: If $A \cross B = A \cross C$ then $B = C$ \\
    \color{draculagreen} True

    Note: $A = \emptyset$
    \begin{align*}
        A \cross B = \emptyset = A \cross C
    \end{align*}
    for all $B, C$
\end{quote}

\begin{proof}
    $(B \subseteq C)$ Let $b \in B$. \cancel{Suppose $a \in A$}
    Since $A = \emptyset$, there is an element $a \in A$. Then
    \begin{align*}
        (a,b) \in A \cross B
    \end{align*}
    Since $A \cross B = A \cross C$, we know
    \begin{align*}
        (a,b) \in A \cross C
    \end{align*}
    By definition of the Cartesian product, $b \in C$. This proves $B \subseteq C$

    \paragraph{}
    $(C \subseteq B)$ By similar reasoning (with the roles of $B$ and $C$ reversed),
    we can show $C \subseteq B$.
\end{proof}

\paragraph{Example} Prove that if $a, b \in \Z$, then $a^2 \neq 4b +2$.

\paragraph{Ideas}

\begin{enumerate}
    \item Cases: $a$ is odd vs. $a$ is even
    \begin{align*}
        a = 2k 
    \end{align*}
    \item looking at all the squares
    \begin{align*}
        c_0 = 0^2 = 0, \quad c_1 = 1^2 = 1, \quad 2^2 = 4, \quad 3^2 = 9, \quad 4^2 = 16, \quad 5^2 = 25 \dots
    \end{align*}
    which can be written as
    \begin{align*}
        c_n = c_{n - 1} + (n - 1)(2k + 1)
    \end{align*}
    \item Claim: The prod of odd numbers is odd and the prod of even numbers is even.
    \begin{align*}
        a^2 = 4b + 2 = 2(2b + 1)
    \end{align*}
    So if $a^2$ is even $\Rightarrow$ $a$ is even: $a = 2k$
    \begin{align*}
        (2k)^2 &= 2(2b + 1) \\
        \implies 4k^2 &= 2 (2b + 1) \\
        \implies 2k^2 &= 2b + 1 
    \end{align*}
    where the RHS is odd but the LHS is even, which is a contradiction.
\end{enumerate}
\begin{proof}
    (Contradiction) Asume there exist $a, b \in \Z$ such that $a^2 = 4b + 2$.
    Then $a^2$ is even, so $a$ is even. Write $a = 2k$ for some $k \in \Z$.

    Then
    \begin{align*}
        (2k)^2 = 4b + 2 \implies 2k^2 = 2b + 1
    \end{align*}
    THe LHS of the equation is even, while the RHS is odd. This is a contradiction.
\end{proof}

Suppose you want to prove statement $P$\dots

\paragraph{Proof by Contradiction Steps}

\begin{enumerate}
    \item Assume $\neg P$
    \item Show that $\neg P$ implies that there is some statement $C$ so that
    $C \land \neg P$ (Contradiction)
    \item $\neg P$ is False $\Leftrightarrow$ $P$ is True
\end{enumerate}

\paragraph{Proposition} The number $\sqrt{2}$ is irrational.

\paragraph{Ideas}
\begin{align*}
    \sqrt{2} = \frac{a}{b} &\implies \sqrt{2} b = a \\
                           &\implies 2b^2 = a^2
\end{align*}
$a^2$ is even $\implies a = 2k$
\begin{align*}
    &\implies 2b^2 = (2k)^2 \\
    &\implies b^2 = 2k^2
\end{align*}
$b^2$ is even $\implies b = 2l$
\begin{proof}
    (Contradiction) Assume $\sqrt{2}$ is rational. Thus there are integers $a, b \in \Z$, $b \neq 0$
    so that $\sqrt{2} = \frac{a}{b}$. We can assume $a$ and $b$ have no common factors---that is,
    there is no positive integer greater than 1 that divides both $a$ and $b$. Now,
    \begin{align*}
        \sqrt{2} = \frac{a}{b} \implies \sqrt{2} b = 2 \implies 2b^2 = a^2
    \end{align*}
    So $a^2$ is even, and thus $a$ is even. Write $a = 2k$ for some $k \in \Z$. Then the equation
    becomes:
    \begin{align*}
        2b^2 = (2k)^2 \implies 2b^2 = 2k,
    \end{align*}
    so $b^2$ is even, and thus $b$ is also even. 

    Thus both $a$ and $b$ are even, which contradicts our earlier assumption that they have no
    common factors.
\end{proof}

\paragraph{Example}

\begin{enumerate}
    \item Prove there is no integer $x$ such that 
    \begin{align*}
        x^2 = 5 \qand x^2 = 9
    \end{align*}
    
    \item Suppose $a, b$ are nonzero. Prove that if $ab$ is irrational, then $a$ is irrational or 
    $b$ is irrational.
\end{enumerate}

\newpage
\lhead{Lecture 8: 1/31/25} 

\paragraph{Example} Prove that for any integer $n$,
\begin{align*}
    n^2 = 4k \qor n^2 = 4k + 1 \qqtext{for some} k \in \Z
\end{align*}

\begin{proof}
    If $n$ is even, then $n = 2m$ for some $m \in Z$. Then $n^2 = (2m)^2 = 4m^2$ so
    if $k = m^2$, the claim holds.

    If $n$ is odd, $n = 2m +1$ for some $m \in \Z$.
    Then $n^2 = (2m + 1)^2 = 4m^2 + 4m + 1 = 4(m^2 + m) + 1$.
    If $k = m^2 + m$, the claim holds.
\end{proof}

\begin{definition}
    Given $a, b \in \Z$, we say $a$ divides $b (a \mid b)$ if $b = a k$ for some $k \in \Z$.
\end{definition}

\paragraph{Example} $2 \mid 12,\, 3 \mid  27,\, 3 \,\cancel{\mid}\, 10 $

\paragraph{Example} Let $a, b, c \in \Z$ Prove that if $a \mid b$ and $b \mid c$ then $a \mid c$.

\begin{proof}
    Since $a \mid b$ and $b \mid c$, there exists $k, l \in \Z$ such that $b = ak$ and $c = bl$. 
    Thus $c = (ak) l = a(kl)$. Since $kl \in \Z$, $a \mid c$.
\end{proof}

\paragraph{Recall} $\N = \{0 , 1, 2, 3, \dots\} \subseteq \Z$

\emph{Well-ordering Principle}: Every nonempty subset of $\N$ has a smallest element.

\begin{theorem}
    Division Algorithm: Let $a, b \in \Z$ with $b > 0$. \underline{There exists} \underbar{unique}
    integers $q$ and $r$ such
    that:
    \begin{align*}
        a = qb + r,\quad 0 \leq r < b.
    \end{align*}
\end{theorem}

\begin{proof}
    Let $a, b \in Z$ with $b > 0$.

    Consider the set 
    \begin{align*}
        A = \{
            a - xb \mid x \in \Z,\, a - xb \geq 0
        \}
    \end{align*}

    The set $A$ is nonempty: If $a \geq 0$, then $a \in A$. If $a < 0$, then $a - ab \in A$ 
    since $a - ab = a (1 - b)$ where $b > 0 \Rightarrow b \geq 1 \Rightarrow (1 - b) \leq 0$.

    By the well-ordering principle, $A$ has a smallest element, call it $r$. Since $r \in A$,
    there exists $q \in \Z$ such that $r = a - qb.$ Thus $a = qb + r$.

    Since $r \in A,\, r \geq 0$. We want to show $r < b$. If not, $r \geq b$ and:
    \begin{align*}
        r - b = a - qb - b = a - (q + 1)b \geq 0
    \end{align*}
    so $r - b \in A$. This contradicts our choice of $r$ as the smallest element of $A$,
    so $r < b$.

    To prove $r$ and $q$ are unique let $q_1, r_1 \in \Z$ such that $a = q_1 b + r_1$ and
    $0 \leq r_ 1 \leq b.$ We have:
    \begin{align*}
        0 = a - a &= (qb + r) - (q_1 b + r_1) \\
                  &= (q - q_1) b + (r - r_1) \\
        \implies  r - r_1 &=  (q_1 - q) b
    \end{align*}
    We may assume $r \geq r_1$, so $r - r_1 \geq 0$. 

    Further, $r < b$ so $r - r_1 < b$. But $r - r_1 = (q_1 - q) b$ implies that
    $r - r_1 \geq b$, because $r - r_1$ is a multiple of b. Thus $ 0 \leq r - r_1 < b$,
    so since $r - r_1$ is a multiple of $b$, it must be zero. Thus $r = r_1$ and thus
    $0 = (q_1 - q)b \Rightarrow q_1 = q$.
\end{proof}

\paragraph{Example} If $5 \,\cancel{|}\, n$, then the ones digit of $n^2$ is not 5.
From the division algorithm
\begin{align*}
    n = 5q +r, \quad r \in \{1,2,3,4\}
\end{align*}
Looking at some examples:
\begin{align*}
    n = 5q + 1 \implies n^2 &= 25q^2 + 10q + 1 \\
                            &= 5(5q^2 + 2q) + 1 \\
    n = 5q + 3 \implies n^2 &= 25q^2 + 30q + 9 \\
                            &= 5(5q^2 + 6q + 1) + 4
\end{align*}

\newpage
\lhead{Lecture 9: 2/3/25}
\paragraph{Week 4}

\paragraph{Recall} Given $a, b \in \Z$ $a$ divides $b$ or $a \mid b$. This means there is some
integer $c$ such that $b = ac$.

\paragraph{Warm-up} \textcolor{draculagreen}{T} or F

Let $a,b,m$ be integers and $m \neq 0$. If $ma \mid mb$, then $a \mid b$.

\begin{proof}
    $ma \mid mb$ implies that $mb = mac$ for some integer $c$. Because $m \neq 0$ (dividing both
    side), we get $b = ac$ which is equivalent to $a \mid b$.
\end{proof}

\paragraph{Definition}

For integers $a, b, d$ if $d \mid a$, we say $d$ is a divisor of $a$. If $d \mid a$ and $d \mid b$,
we say $d$ is a common divisor of $a, b$ (with $\abs{d} \leq \abs{a},\, \abs{d} \leq \abs{b}$).

If $d$ is the largest positive integer that divides both $a$ and $b$ we call $d$ the greatest common
divisor of $a, b$. 
\begin{align*}
    d = \gcd(a, b)
\end{align*}

Example: \(\gcd(2,3) = 1, \quad \gcd(9,12) = 3\)

\paragraph{The Euclidean Algorithm}

\begin{quotation}
    Input: $a,b$ positive integers

    Output: $\gamma_n$ positive integer

    Claim: $\gamma_n = \gcd(a,b)$
\end{quotation}
where we repeatedly apply the division algorithm to find the gcd.

Assume $a < b$
\begin{align*}
    b &= q_1 a + \gamma_1 \qqtext{where} 0 \leq \gamma_1 < a \\
    a &= q_2 \gamma_1 + \gamma_2 \qqtext{where} 0 \leq \gamma_2 < \gamma_1 \\
\end{align*}

e.g $\gcd(5817, 1428)$:
\begin{align*}
    \longdivision[0]{5817}{1428}, \quad \longdivision[0]{1428}{105}
\end{align*}
So
\begin{align*}
    a &= 1428, \quad b = 5817 \\
    % long division here 
    5817 &= 4 \cdot 1428 + 105 \\
    1428 &= 13 \cdot 105 + 63 \\
    105 &= 1 \cdot 63 + 42 \\
    63 &= 1 \cdot 42 + 21 \\
    42 &= 2 \cdot 21 + 0
\end{align*}
Thus the claim $\gcd(5817, 1428) = 21$. Or in symobolic form:
\begin{align*}
    \gamma_1 &= q_3 \gamma_2 + \gamma_3 \qqtext{where} 0 \leq \gamma_3 < \gamma_2 \\
    &\vdots \\
    \gamma_{n -2} &= q_n \gamma{n - 1} + \gamma_n \qqtext{where} 0 \leq \gamma_n < \gamma_{n - 1} \\
    \gamma_{n-1} &= q_{n + 1} \gamma_n + 0
\end{align*}

\begin{theorem}
    $\gamma_n = \gcd(a,b)$
\end{theorem}

\begin{proof}
    Step 1: We will prove $\gamma_n \geq \gcd(a,b)$.

    To show $\gamma_n \geq d$, we will prove $d \mid \gamma_n$. By definition,
    $d \mid a$ and $d \mid b$. Because $\gamma_1 = b - q_1 a$, so $d \mid \gamma_1$.
    Because $\gamma_2 = a - q_2 \gamma_1$, so $d \mid \gamma_2$. With the same argument we conclude
    $d \mid \gamma_n$.

    Step 2: We will prove $\gamma_n \leq d = \gcd(a,b)$. We need to show $\gamma_n \mid a$ and 
    $\gamma_n \mid b$. From $\gamma_1 = q_{n+1} \gamma_n$, we get $\gamma_n \mid \gamma_{n-1}$.
    From $\gamma_2 = q_n \gamma_{n - 1} + \gamma_n$, we get $\gamma_n \mid \gamma_2$. With the same
    argument, we get $\gamma_n \mid \gamma_1$ and $\gamma_n \mid \gamma_2$ which implies
    $\gamma_n \mid a$ and $\gamma_n \mid b$.
\end{proof}

\begin{theorem}
    (Bezout's Identity) There exists such integers $s, t$ such that
    \begin{align*}
        \gcd(a, b) = as + bt
    \end{align*}
\end{theorem}
e.g. $\gcd(5,7) = 1$ and from the identity:
\begin{align*}
    1 = 5 \cdot 3 + 7 \cdot (-2)
\end{align*}
From our previous example $\gcd(5817, 1428) = 21$ and from the identity:
\begin{align*}
    21 = 5817 s + 1428 t
\end{align*}
Using the Euclidean Algorithm we can do the following:
\begin{align*}
    21 &= 63 - 42 \\
       &= 63 - (105 - 63) \\
       &= -104 + 2(1428 - 13 \cdot 105) \\
       &= 2 \cdot 1428 - 27 (5817 - 4 \cdot 1428) \\
       &= -27 \cdot 5817 + 110 \cdot 1428
\end{align*}
where $s = -27$ and $t = 110$.
\end{document}