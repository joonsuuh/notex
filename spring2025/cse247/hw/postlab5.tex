\documentclass[../main.tex]{subfiles}

\graphicspath{{../images/}}

\usepackage[noend]{algpseudocode} % for pseudocode
\usepackage[plain]{algorithm} % float environment for algorithms
% preferred pseudocode style
\algrenewcommand{\algorithmicprocedure}{}
\algrenewcommand{\algorithmicthen}{}

% ``do { ... } while (cond)''
\algdef{SE}[DOWHILE]{Do}{doWhile}{\algorithmicdo}[1]{\algorithmicwhile\ #1}%

% ``for (x in y ... z)''
\newcommand{\ForRange}[3]{\For{#1 \textbf{in} #2 \ \ldots \ #3}}
\usepackage{xcolor}
\usepackage{listings}
\definecolor{codegreen}{rgb}{0,0.6,0}
\definecolor{codegray}{rgb}{0.5,0.5,0.5}
\definecolor{codepurple}{rgb}{0.58,0,0.82}
\definecolor{backcolour}{rgb}{0.95,0.95,0.92}

\definecolor{draculabg}      {RGB} {40,   42,   54}
\definecolor{draculacl}      {RGB} {68,   71,   90}
\definecolor{draculafg}      {RGB} {248,  248,  242}
\definecolor{draculacomment} {RGB} {98,   114,  164}
\definecolor{draculacyan}    {RGB} {139,  233,  253}
\definecolor{draculagreen}   {RGB} {80,   250,  123}
\definecolor{draculaorange}  {RGB} {255,  184,  108}
\definecolor{draculapink}    {RGB} {255,  121,  198}
\definecolor{draculapurple}  {RGB} {189,  147,  249}
\definecolor{draculared}     {RGB} {255,  85,   85}
\definecolor{draculayellow}  {RGB} {241,  250,  140}

\lstdefinestyle{mystyle}{
    backgroundcolor=\color{draculabg},   
    commentstyle=\color{draculacomment},
    keywordstyle=\color{draculapink},
    numberstyle=\tiny\color{draculacl},
    stringstyle=\color{draculapurple},
    basicstyle=\ttfamily\footnotesize\color{draculafg},
    breakatwhitespace=false,         
    breaklines=true,                 
    captionpos=b,                    
    keepspaces=true,                 
    numbers=left,                    
    numbersep=5pt,                  
    showspaces=false,                
    showstringspaces=false,
    showtabs=false,                  
    tabsize=2
}
\lstset{style=mystyle} 
\begin{document}
\pagestyle{fancy}
\chead{PostLab 5}
\rhead{Junseo Shin}
\lhead{CSE 247}



\renewcommand{\thefigure}{\arabic{figure}}
\section*{Questions}

\begin{enumerate}
    \item \texttt{updateHeight} takes the maximum of the left and right children heights and adds 1
        to give the height of the current node using the \texttt{height} instance variable. 

    \item \texttt{getBalance} takes the difference of the left and right children heights to give
    the balance factor of the current node using the \texttt{height} instance variable. Returning a
    positive value means the left subtree is taller, 0 means the tree is balanced, and a negative
    value means the right subtree is taller.

    \item \texttt{rebalance} checks the balance factor with \texttt{getBalance} (helper) and
    performs the rotations if it unbalanced based on \texttt{getBalance} where we rotate for 4
    cases: left-left, left-right, right-right, and right-left. To rotate the tree we use helper
    functions \texttt{leftRotate} and \texttt{rightRotate}.

    \item \texttt{rightRotate} rotates the tree (clockwise) by moving the root node to the right
    child and making the left child the new root node. The \texttt{updateHeight} helper function
    recalculates the height of the new root node, and \texttt{setLeft} and \texttt{setRight} helper
    functions fix the pointers from the rotation operation.

    \item In the insertion and removal function we have to update the height of the rood node then
    check if the tree is unbalanced and rebalance it if necessary. The height update
    happens after the insertion or removal operation, and before the rebalancing operation.

    \item If \texttt{rootKey < $v$} we go to the right subtree and look again for the least element
    in the set $>= v$. This is because all the elements in the left subtree will be
    $\texttt{< $v$}$ so we can ignore them.
    
    \item If \texttt{rootKey >= $v$} we might have found the answer (the root),
    but we still need to check the left subtree for a potentially least element in the set $>= v$.
    This is because the left subtree might have a smaller element than the root but still greater
    than $v$.
    
    \item PSEUDOCODE
    \begin{lstlisting}[language=Java]
        // little helper function 
        public T firstAfter(T v) {
            return firstAfterHelper(v, root, null);
        }

        // recursive part
        private T firstAfterHelper(T v, TreeNode<T> root, T bestSoFar) {
            if (root == null) {
                return bestSoFar; 
            }
            
            int comparison = v.compareTo(root.value);
            
            if (comparison > 0) {
                // node < v ==== look in right subtree
                return firstAfterHelper(v, root.right, bestSoFar);
            } else {
                // node >= v ==== might be ans, but look left subtree
                return firstAfterHelper(v, root.left, root.value);
            }
        }
    \end{lstlisting}
\end{enumerate}




\end{document} 