\documentclass[../main.tex]{subfiles}

\graphicspath{{../images/}}

\usepackage[noend]{algpseudocode} % for pseudocode
\usepackage[plain]{algorithm} % float environment for algorithms
% preferred pseudocode style
\algrenewcommand{\algorithmicprocedure}{}
\algrenewcommand{\algorithmicthen}{}

% ``do { ... } while (cond)''
\algdef{SE}[DOWHILE]{Do}{doWhile}{\algorithmicdo}[1]{\algorithmicwhile\ #1}%

% ``for (x in y ... z)''
\newcommand{\ForRange}[3]{\For{#1 \textbf{in} #2 \ \ldots \ #3}}

\begin{document}
\pagestyle{fancy}
\chead{PreLab 5}
\rhead{Junseo Shin}
\lhead{CSE 247}


\renewcommand{\thefigure}{\arabic{figure}}
\section*{Questions}

\begin{enumerate}
    \item One paramenter \texttt{root} of type \texttt{TreeNode<T>} which is used
    to access its children nodes.
        
    It returns the height of the tree rooted at \texttt{root}.

    \item One paramenter \texttt{root} of type \texttt{TreeNode<T>} which is used
    to access its children nodes again.

    It returns the height difference between the left and right subtrees of the
    tree at \texttt{root}.

    \item The \texttt{root} node is an unbalanced subtree that needs to balance
    for the 4 cases: left-left, left-right, right-right, and
    right-left. Then it returns the root node of the balanced subtree.

    \item The \texttt{root} node needs to be rotated to the right which makes its
    left child the new root which is returned.

    \item The max height of a tree's root node is \texttt{height = max(leftHeight, rightHeight) + 1}

    \item A new allocated leaf node has a height of 0, and an empty tree has a height of -1.
    
    \item The helper functions makes it so we can pass the original root node as a parameter
    for the recursive function.

    \item Because the old root node is now the right child of the new root node so the pointer has
    to update to point the other way around now.
\end{enumerate}




\end{document} 